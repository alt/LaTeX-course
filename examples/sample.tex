\documentclass[ngerman,fleqn]{scrartcl}

\usepackage{
  amsmath,
  babel,
  xltxtra
}

\usepackage[math]{blindtext}

\newcommand\test[2][a]{Dieser Befehl gibt das erste Argument: #1 und das zweite: #2 aus.}

\renewcommand\d{\,\mathrm d}

\setmainfont{Linux Libertine}
\setsansfont{Linux Biolinum}
\setmonofont{DejaVu Sans Mono}

\begin{document}
\textit{Dies \textsf{ist} ein \textbf{Test}.}

\[\int_{-\infty}^\infty\kern-.7em \epsilon_{ijk} \d x \neq \sum_{n = 0\atop m = 0}^\aleph \alpha\]

\begin{equation}
\int_{-\infty}^\infty\kern-.7em \epsilon_{ijk} \d x \neq \sum_{n = 0\atop m = 0}^\aleph \alpha
\end{equation}

\begin{align}
a &= b & c &= d &e &= f & g &= h \\
a*b &= x\\
\end{align}



\blinddocument

\end{document}
