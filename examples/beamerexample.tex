%% compile this with pdfTeX
%% view the output with acrobat reader
%% or okular to get alle the nice effects

%% this document will NOT compile if the video and the images
%% are not on your computer – most probably this will
%% not be the case …


\documentclass{beamer}

\usepackage[utf8]{inputenc}
\usepackage[T1]{fontenc}

\mode<presentation>{
  \usetheme{Frankfurt}
  \useoutertheme{infolines}
  \usecolortheme[RGB={0,100,130}]{structure}
  \useinnertheme{rounded}
  \setbeamertemplate{navigation symbols}{} 
}

\title{Eine Testpräsentation}
\author{Arno Trautmann}
\date{\today}

\usepackage{multimedia}

\begin{document}
\frame{\maketitle}
\titlepage
\frame{\tableofcontents}
\section{Titel}
\begin{frame}{Titel}{Untertitel}
Dies ist ein kleiner Frame.
\end{frame}

\begin{frame}{Titel}
Dieser Frame hat keinen Untertitel :(
\end{frame}

\section{Effekte}
\begin{frame}
\includegraphics<1>[width=.8\textwidth]{/home/arno/Desktop/ppem/Seminarvortrag/Bilder/lochblende_bild.jpg}
\transdissolve
\includegraphics<2>[width=.8\textwidth]{/home/arno/Desktop/ppem/Seminarvortrag/Bilder/lochblende.jpg}
\end{frame}

\begin{frame}
\transsplitverticalout
$E = mc^2$
\end{frame}

\begin{frame}{Ein Video!}
{}
{\centering
\movie[autostart,label=vipmen,width=8cm,height=4cm,poster,showcontrols,repeat]{}{/home/matlab/toolbox/vipblks/vipdemos/vipmen.avi}
}
%\hyperlinkmovie[start=5s,duration=7s]{video3}{\beamerbutton{Show the middle stage}}
%\hyperlinkmovie[start=12s,duration=5s]{video3}{\beamerbutton{Show the late stage}}
\end{frame}

\end{document}
