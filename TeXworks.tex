\documentclass{scrartcl}

\usepackage{
  xltxtra
}
\setmainfont{Linux Libertine}
\setmonofont{Linux Biolinum}
\title{Arbeiten mit \TeX works}
\author{Arno Trautmann\thanks{Arno.Trautmann@gmx.de}}

\begin{document}
\maketitle
\begin{abstract}
Dieses Dokument beschreibt ganz kurz den Umgang mit dem Editor \TeX works und die möglichen Einstellungen und Erweiterungen.
\end{abstract}
\TeX works ist vom bekannten Editor \TeX shop inspiriert und richtet sich ebenso wie solcher zunächst an \LaTeX-Anfänger. Aber auch Fortgeschrittene und Profis profitieren von der einfachen und klaren Nutzeroberfläche sowie vor allem der eingebauten sync\TeX-Technologie. Aufgrund der Einfachheit der Oberfläche ist auch eine Beschreibung der Verwendung sehr einfach.

\minisec{Kompilieren}
Zum Erstellen eines \LaTeX-Dokumentes mit Ausgabe benötigt man lediglich folgende Schritte:

\begin{itemize}
\item Öffnen des Editors
\item Anlegen einer neuen Datei oder Öffnen einer bereits vorhandenen
\item Änderungen am Code
\item evtl. Speichern (nicht nötig, da beim Kompilieren automatisch gespeichert wird)
\item Auf die Schaltfläche zum Kompilieren drücken (alternativ Tastenkombination Strg+t für typeset)
\item das fertige pdf wird im eingebauten Viewer in einem neuen Fenster angezeigt
\item Navigation zwischen pdf und tex-Datei ist möglich mittels Rechtsklick auf die entsprechende Stelle oder durck Linksklick bei gedrückter Strg-Taste
\end{itemize}

\minisec{Neue Option zum Kompilieren}

Will man mit einer anderen Maschine oder einem anderen Format als dem eingestellten kompilieren (z.\,B. \XeLaTeX\ statt pdf\LaTeX\ oder lua\TeX\ mit dem plain\TeX-Format o.\,ä.), kann man dies an der Schaltfläche zum Kompilieren einstellen. Ist die gewünschte Maschine bzw. das Format nicht verfügbar, kann es folgendermaßen hinzugefügt werden:

\begin{itemize}
\item[⇒] im Menü auf Edit klicken
\item[⇒] Preferences
\item[⇒] Typesetting
\item bei den \verb|processing tools| kann ein neues Tool hinzugefügt werden (+ Schaltfläche) oder ein bereits vorhandenes editiert werden
\item bei Hinzufügen eines neuen Tools kann ein Name vergeben werden (wird in der Auswahl später angezeigt) und das Programm angegeben werden. Für \XeLaTeX\ nennt man z.\,B. den Namen \verb|XeLaTeX| und das Programm \verb|xelatex|, also das, was in der Kommandozeile aufgerufen wird.
\item weiterhin können Argumente und Optionen übergeben werden, es empfiehlt sich, folgendes anzugeben:\\
\verb|$synctexoption| – ermöglicht sync\TeX-Einsatz\\
\verb|$fullname| – übergibt den Namen des aktuellen Dokumentes
\item danach steht die neue Option zur Verfügung und kann in der Schaltfläche ausgewählt werden.
\end{itemize}



\end{document}