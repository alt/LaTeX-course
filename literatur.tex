\documentclass{scrartcl}

\usepackage{
  array,
  booktabs,
  dtklogos,
  hyperref,
  xltxtra,
  xspace
}


\title{Literaturempfehlungen zum \LaTeX-Kurs}
\subtitle{eine kurze, unvollständige Auflistung\\ (work in progress)}
\author{Arno Trautmann\thanks{arno.trautmann@gmx.de}}

\setmainfont{Arno Pro}
\addtokomafont{disposition}{\fontspec{Arno Pro}}

\newcommand\vorhanden{}
\newcommand\nichtvorhanden{}
\newcommand\präsenz{}

\newcommand\e{(e)\xspace}
\newcommand\w{(w)\xspace}

\begin{filecontents}{literaturbib.bib}

@book{knuth1986texbook,
  title={{The TEXbook}},
  author={Knuth, D.E. and Bibby, D. and Makai, I.},
  year={1986},
  publisher={Addison-Wesley Reading, MA}
}

\end{filecontents}


\begin{document}
\maketitle

\section*{Bücher}
Folgende Bücher (unvollständige Liste) sind als Einführungs- \e oder weiterführende \w Literatur zum Thema \LaTeXTeX\ geeignet und stellen eine persönliche Literaturempfehlung dar. 

\catcode`\•13
\newcommand•[3]{\item[]#1. \emph{#2.} #3}

\begin{itemize}
•{D.E. Knuth}{The \TeX book}{Addison-Wesley Reading, MA, 1986}
•{C. Detig}{Der \LaTeX-Wegweiser}{mitp-Verlag.  2004²}
•{E. Niedermair \& M. Niedermair}{\LaTeX – Das Praxisbuch}{Franzis. 2006³}
•{F. Mittelbach \& M. Goossens}{Der \LaTeX-Begleiter }{Pearson Studium. 2005²}
\subsubsection*{Paketspezifische Bücher: (auch online/mittels \texttt{texdoc} verfügbar)}

\end{itemize}

\begin{tabular}{ll}
  KOMA-Skript\\
  PS-Tricks\\
\end{tabular}


\section*{Dokumentationen im Internet}
\begin{tabular}{ll}
  l2tabu – böse Fehler, die man unbedingt vermeiden sollte. Auf dem eigenen Rechner unter\\ \texttt{texdoc l2tabu}\\
  zu erreichen oder unter\\
  \url{http://tug.ctan.org/tex-archive/info/l2tabu/german/l2tabu.pdf}
  l2kurz – eine kurze Einführung in \LaTeX. Wie l2tabu auf dem Rechner zu finden, außerdem:\\
  \url{http://www.dante.de/CTAN/info/lshort/german/l2kurz.pdf}
  Die gesamte DANTE Edition, die zu einigen Themen gute und umfangreiche Bücher anbietet:\\
  \url{http://www.dante.de/index/Literatur.html}\\
  Für DANTE-Mitglieder gibt es Rabatt, bei Interesse an mich wenden.
\end{tabular}


\end{document}