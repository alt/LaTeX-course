\documentclass{scrartcl}

\usepackage{
  booktabs,
  dtklogos,
  hyperref,
  xltxtra
}


\title{Literaturempfehlungen}
\subtitle{eine kurze, unvollständige Auflistung}
\author{Arno Trautmann\thanks{arno.trautmann@gmx.de}}

\setmainfont{Arno Pro}
\addtokomafont{disposition}{\fontspec{Arno Pro}}

\begin{document}
\maketitle


Folgende Bücher (unvollständige Liste) bieten eine Einführung oder weiterführende Informationen zum Thema \LaTeXTeX:

\begin{tabular}{ll}
  Detig, C. (2004²). Der \LaTeX-Wegweiser. mitp-Verlag. ISBN-13: 978-3826614149\\
  Niedermair, E. \& Niedermair, M. (2006³). LaTeX – Das Praxisbuch. Franzis. ISBN-13: 9783772369308\\
  Mittelbach, F. \& Goossens, M. (2005²). Der LaTeX-Begleiter. Pearson Studium. ISBN-13: 9783827371669\\
  Knuth, D. (1986). Computers \& Typesetting, Volume A, The \TeX{}book. Addison-Wesley. ISBN: 0-201-13447-0\\
\end{tabular}
Paketspezifische Bücher: (auch online verfügbar)
\begin{tabular}{ll}
  KOMA-Skript\\
  PS-Tricks\\
\end{tabular}
Dokumentationen im Internet:
\begin{tabular}{ll}
  l2tabu – böse Fehler, die man unbedingt vermeiden sollte. Auf dem eigenen Rechner unter\\ \texttt{texdoc l2tabu}\\
  zu erreichen oder unter\\
  \url{http://tug.ctan.org/tex-archive/info/l2tabu/german/l2tabu.pdf}
  l2kurz – eine kurze Einführung in \LaTeX. Wie l2tabu auf dem Rechner zu finden, außerdem:\\
  \url{http://www.dante.de/CTAN/info/lshort/german/l2kurz.pdf}
  Die gesamte DANTE Edition, die zu einigen Themen gute und umfangreiche Bücher anbietet:\\
  \url{http://www.dante.de/index/Literatur.html}\\
  Für DANTE-Mitglieder gibt es Rabatt, bei Interesse an mich wenden.
\end{tabular}


\end{document}