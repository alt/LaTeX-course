\documentclass{scrartcl}

\usepackage{
  array,
  booktabs,
  dtklogos,
  hyperref,
  xltxtra,
  xspace
}

\hypersetup{
  colorlinks=true,
  urlcolor=blue
}
\title{Literaturempfehlungen zum \LaTeX-Kurs}
\subtitle{eine kurze, unvollständige Auflistung\\ (work in progress)}
\author{Arno Trautmann\thanks{arno.trautmann@gmx.de}}

\setmainfont{Arno Pro}
\addtokomafont{disposition}{\fontspec{Arno Pro}}

\newcommand\vorhanden{}
\newcommand\nichtvorhanden{}
\newcommand\präsenz{}

\newcommand\e{(e)\xspace}
\newcommand\w{(w)\xspace}

\begin{document}
\hyphenation{Mathe-satz}
\maketitle

\section*{Bücher}
Folgende Bücher (unvollständige Liste) sind als Einführungs- \e oder weiterführende \w Literatur zum Thema \LaTeXTeX\ geeignet und stellen eine persönliche Literaturempfehlung dar. 

\catcode`\•13
\newcommand•[4]{\item[]#1. \emph{#2.} #3 #4}

\begin{itemize}
•{D.E. Knuth}{The \TeX book}{Addison-Wesley Reading, MA, 1986}\w
•{C. Detig}{Der \LaTeX-Wegweiser}{mitp-Verlag.  2004²}\e
•{E. Niedermair \& M. Niedermair}{\LaTeX – Das Praxisbuch}{Franzis. 2006³}\e
•{F. Mittelbach \& M. Goossens}{Der \LaTeX-Begleiter }{Pearson Studium. 2005²} {\e\w}
\end{itemize}
\subsubsection*{Paketspezifische Bücher: (auch online\,/\,mittels \texttt{texdoc} verfügbar)\footnote{Inhaltlich sind die pdf-Versionen identisch mit den Büchern, letztere sind aber typographisch hochwertig gesetzt.}}
\begin{itemize}
•{M. Kohm \& J-U. Morawski}{KOMA-Script}{Lehmanns Media. ³2008}\e
•{H. Voß}{PSTricks}{Lehmanns Media. ⁵2008}\w
\end{itemize}

Die gesamte DANTE-Edition bietet sowohl breite Übersichten über große Themen (z.\,B. den Mathesatz) als auch sehr spezielle Literatur (etwa PSTricks):\\
  \url{http://www.dante.de/index/Literatur.html}\\   Für DANTE-Mitglieder gibt es Rabatt; bei Interesse an mich wenden.

\section*{Dokumentationen im Internet\,/\,mittels \texttt{texdoc}}
\begin{tabular}{ll}
  l2tabu – böse Fehler, die man unbedingt vermeiden sollte. Auf dem eigenen Rechner unter\\ \texttt{texdoc l2tabu}\\
  zu erreichen oder unter\\
  \url{http://tug.ctan.org/tex-archive/info/l2tabu/german/l2tabu.pdf}\\
  l2kurz – eine kurze Einführung in \LaTeX. Wie l2tabu auf dem Rechner zu finden, außerdem:\\
  \url{http://www.dante.de/CTAN/info/lshort/german/l2kurz.pdf}\\
\end{tabular}


\end{document}