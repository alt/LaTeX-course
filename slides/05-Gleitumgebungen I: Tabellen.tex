\let\myoldbar| % as | is needed for tabular headers, we need to restore it’s old meaning later
\makeatletter
\@ifundefined{draftmodeon}{
  \documentclass[german,t]{beamer}
}{
  \documentclass[german,t,draft]{beamer}
}
\makeatother
\mode<presentation>{
  \usetheme{Frankfurt}
  \useoutertheme{infolines}
  \usecolortheme[RGB={0,100,130}]{structure}
  \useinnertheme{rounded}
  \setbeamertemplate{navigation symbols}{} 
}

\newcommand\subtitlei[1]{\def\insertsubtitlei{#1}}
\newcommand\subtitleii[1]{\def\insertsubtitleii{#1}}

\setbeamertemplate{title page}
  {
    \vspace*{1cm}
    \begin{centering}
      {\usebeamerfont{title}\usebeamercolor[bg]{title}
      {\Large \inserttitle}
      \\[.3cm]
      \insertsubtitlei\\
      \large \color{black}\insertsubtitleii
      }
\vspace*{1cm}
    \vbox to 1cm {\hfill \includegraphics[width=.2\textwidth]{ctanlion}}
      \usebeamerfont{subtitle}\usebeamercolor[bg]{subtitle}
      {\color{black}\Large \insertsubtitle}\vfill

    {\hfill \insertdate \hfill}
    \end{centering}

    \vfill\vfill

    \hspace*{-.47cm}
    \usebeamertemplate*{footline}
    \vspace*{-1.15cm}
}

\setbeamercolor*{author in head/foot}{parent=palette tertiary}
\setbeamercolor*{title in head/foot}{parent=palette secondary}
\setbeamercolor*{date in head/foot}{parent=palette primary}

\setbeamercolor*{section in head/foot}{parent=palette tertiary}
\setbeamercolor*{subsection in head/foot}{parent=palette primary}

\defbeamertemplate*{footline}{infolines theme changed}
{
  \leavevmode%
  \hbox{%
  \begin{beamercolorbox}[wd=.3\paperwidth,ht=2.25ex,dp=1ex,center]{author in head/foot}%
      \insertshortauthor
  \end{beamercolorbox}%
  \begin{beamercolorbox}[wd=.4\paperwidth,ht=2.25ex,dp=1ex,center]{title in head/foot}%
    \insertshortdate{}
  \end{beamercolorbox}%
  \begin{beamercolorbox}[wd=.3\paperwidth,ht=2.25ex,dp=1ex,center]{author in head/foot}%
    \insertframenumber{} / \inserttotalframenumber
  \end{beamercolorbox}}%
  \vskip0pt%
}

\defbeamertemplate*{headline}{infolines theme changed}
{
  \leavevmode%
  \hbox{%
  \begin{beamercolorbox}[wd=.3\paperwidth,ht=2.25ex,dp=1.5ex,center]{title in head/foot}%
    \LaTeX-Kurs 2010
  \end{beamercolorbox}%
  \begin{beamercolorbox}[wd=.4\paperwidth,ht=2.25ex,dp=1.2ex,right]{section in head/foot}%
      \coursenumber\ – \coursetitle \hspace*{2em}
  \end{beamercolorbox}%
  \begin{beamercolorbox}[wd=.3\paperwidth,ht=2.25ex,dp=1.2ex,left]{subsection in head/foot}%
    \hspace*{2em}\insertsectionhead
  \end{beamercolorbox}}%
  \vskip0pt%
}

\setbeamersize{text margin left=1em,text margin right=1em}

\def\extractnumber"#1-#2".{#1}
\def\extracttitle"#1-#2".{#2}
\def\coursenumber{\expandafter\extractnumber\jobname.}
\def\coursetitle{\expandafter\extracttitle\jobname.}

\usepackage{
  babel,
  dtklogos,
  moreverb,
  shortvrb,
  xltxtra,
  yfonts
}
\usepackage[final]{showexpl}

\hypersetup{
  colorlinks=true,
  breaklinks=true,
  linkcolor=black, citecolor=black, filecolor=black, menucolor=black,
  urlcolor=blue,
  pdfauthor={Arno L. Trautmann},
}

\MakeShortVerb|

\author{Arno Trautmann}
\institute{Heidelberg}
\title{Einführung in das Textsatzsystem\\[2ex] \Huge \LaTeX}
\subtitlei{Vorlesungsreihe im Sommersemester 2010}
\subtitleii{\textfrak{universitatis:~studii~heydelbergensis:}}
\subtitle{\coursenumber\ – \coursetitle}

\logo{}%\includegraphics[width=7.4em]{unilogo.svg}}

\defaultfontfeatures{Scale=MatchLowercase}
\setmonofont{DejaVu Sans Mono}

\def\lcode{\vspace{.5ex}\par\boxedverbatim}
\def\endlcode{\endboxedverbatim\bigskip\\}
\newenvironment{mydesc}{\begin{tabular}{|>{\columncolor{lightgray}\color{blue}}rl|}\hline}{\\\hline\end{tabular}}

\def\pkg#1{\texttt{#1}}
\def\macro#1{|#1|}

\def\plainTeX{\textsf{plain\TeX}}

\catcode`\⇒=\active
\def⇒{\ensuremath{\Rightarrow}}
\catcode`\⇐=\active
\def⇐{\ensuremath{\Leftarrow}}
\catcode`\…13
\let…\dots


\newcommand\notiz[1]{}
\renewcommand\checkmark{\color{green}\ding{51}}
\newcommand\cross{\color{red}\ding{53}}

\newcommand\pdf[2][]{\bgroup
  \setbeamercolor{background canvas}{bg=}%
  \includepdf[#1]{#2}%
    \egroup
}

\AtBeginDocument{
  \lstset{%
    backgroundcolor=\color[rgb]{.9 .9 .9},
    basicstyle=\ttfamily\small,
    breakindent=0em,
    breaklines=true,
    commentstyle=,
    keywordstyle=,
    identifierstyle=,
    captionpos=b,
    numbers=none,
    frame=tlbr,%shadowbox,
    frameround=tttt,
    pos=r,
    rframe={single},
    explpreset={numbers=none}
  }
}
\makeatletter
\g@addto@macro\beamer@lastminutepatches{ % thanks to Ulrike for this!
  \frame[plain,t]{\titlepage}
  \frame{\centerline{\huge \color[RGB]{0,100,130}Inhalt}\tableofcontents}
}

%—— itemize-hack
\def\outside{o}
\def\inside{i}
\let\insideitemizei\outside
\let\insideitemizeii\outside
\def\altenditemize{
  \if\altlastitem 1%
    \let\altlastitem0%
  \else%
    \end{itemize}%
    \let\insideitemizei\outside%
  \fi%
}

\begingroup
  \lccode`\~=`\^^M%
\lowercase{%
  \endgroup
  \def\makeenteractive{%
    \catcode`\^^M=\active
    \let~\altenditemize
}%
}

\def\newitemi{%
  \ifx\insideitemizei\inside%
    \let\altlastitem1%
    \expandafter\item%
  \else%
    \begin{itemize}%
    \let\insideitemizei\inside%
    \let\altlastitem1%
    \makeenteractive%
    \expandafter\item%
  \fi
}

\def\newitemii{
  \ifx\insideitemizeii\inside
    \expandafter\item%
  \else
    \begin{itemize}
      \let\insideitemizeii\inside
      \expandafter\item%
  \fi
}

\def\makeitemi#1{%
  \expandafter\ifx\csname cc\string#1\endcsname\relax
    \add@special{#1}%
    \expandafter
    \xdef\csname cc\string#1\endcsname{\the\catcode`#1}%
    \begingroup
      \catcode`\~\active  \lccode`\~`#1%
      \lowercase{%
      \global\expandafter\let
         \csname ac\string#1\endcsname~%
      \expandafter\gdef\expandafter~\expandafter{\newitemi}}%
    \endgroup
    \global\catcode`#1\active
  \else
  \fi
}

\def\makeitemii#1{%
  \expandafter\ifx\csname cc\string#1\endcsname\relax
    \add@special{#1}%
    \expandafter
    \xdef\csname cc\string#1\endcsname{\the\catcode`#1}%
    \begingroup
      \catcode`\~\active  \lccode`\~`#1%
      \lowercase{%
      \global\expandafter\let
         \csname ac\string#1\endcsname~%
      \expandafter\gdef\expandafter~\expandafter{\newitemii}}%
    \endgroup
    \global\catcode`#1\active
  \else
  \fi
}

\def\add@special#1{%
  \rem@special{#1}%
  \expandafter\gdef\expandafter\dospecials\expandafter
{\dospecials \do #1}%
  \expandafter\gdef\expandafter\@sanitize\expandafter
{\@sanitize \@makeother #1}}
\def\rem@special#1{%
  \def\do##1{%
    \ifnum`#1=`##1 \else \noexpand\do\noexpand##1\fi}%
  \xdef\dospecials{\dospecials}%
  \begingroup
    \def\@makeother##1{%
      \ifnum`#1=`##1 \else \noexpand\@makeother\noexpand##1\fi}%
    \xdef\@sanitize{\@sanitize}%
  \endgroup}
\AtBeginDocument{
  \makeitemi{•}
}
%——beamer versaut hier irgendwas…
\def\•{\end{itemize}}
%—— itemize-hack
\makeatother

\usepackage[normalem]{ulem}

\newcommand\messdaten{
\toprule
\bf Pendellänge $l$ [m]& \bf Dauer $T$ [s]\\\midrule
4 & 8 \\
2 & 4 \\
1   & 2 \\
.9  & 1.8 \\
0.8 & 1.6 \\
0.7 & 1.4 \\
0.6 & 1.2 \\
0.5 & 1.0 \\
0.4 & 0.8 \\
0.3 & 0.6 \\
0.2 & 0.4 \\
0.1 & 0.2 \\
0.05 & 0.1 \\
0.02 & 0.05 \\
0.01 & 0.02 \\
0.005 & 0.01 \\
0.0025 & 0.005\\
\bottomrule
}

\let\eV\relax % to avoid collision of e.V. with electron volt …
\usepackage{
  caption,
  longtable,
  pifont,
  rotating,
  supertabular,
  tabularx,
  tabulary
}

\begin{document}
% to make tabular headers work correctly. \let/ necessary due to stupidity (?) of array
\let|\myoldbar
\catcode`\/=13
\MakeShortVerb/

\section[Gleitumgebungen]{Allgemeine Gleitumgebungen}
\begin{frame}{Was sind Gleitobjekte?}
• Objekte, die frei im Dokument „gleiten“ können
• Gleiten vermeidet große Leerräume
• \TeX\ versucht optimale Positionierung
• zu beachten:
• Objekte sollen nicht vor Referenzen auftauchen
• Objekte sollen nicht die Reihenfolge tauschen
• Seitenumbruch stark abhängig von Gleitobjekten
• \emph{optimaler Seitenumbruch ist mit \TeX\ nicht möglich!}
\•
\end{frame}

\begin{frame}[fragile]{Gleitumgebungen}
• eine Gleitumgebung besteht aus verschiedenen Teilen:
• Inhalt (Bild, Tabelle, Text, \dots)
• automatische Bezeichnung: „Tabelle 1:“ (/\caption/)
• Beschriftung: „Messergebnisse“ (Argument von /\caption{}/)
• Markierung für Verweise: /\label{fig:messergebnisse}/
\•
\end{frame}

\begin{frame}[fragile]{Gleitumgebungen}
• \LaTeX\ verfügt über verschiedene Gleitumgebungen:
• /table/ für Tabellen
• /figure/ für Abbildungen
• Paket /float/ ermöglicht Definition eigener Umgebungen
• für zweispaltigen Satz: /table*/, /figure*/ über beide Spalten
\•
\end{frame}

\begin{frame}[fragile]{Gleitumgebungen}
\begin{block}{Positionierungsparameter für Gleitumgebungen:} 
/\begin{table}[...]/
\\
\begin{mydesc}
! & ignoriert Einschränkungen und fährt fort \\
h & Objekt genau an dieser Stelle setzen\\
t & Objekt am Seitenanfang setzen\\
b & dito, Seitenende\\
p & Objekt in Gleitobjektkolumne bzw. -spalte setzen\\
H & „genau hier und sonst nirgends“ – Paket /float/
\end{mydesc}
\end{block}
\end{frame}

\begin{frame}[fragile]{Gleitumgebungen}
• Wenn die automatische Positionierung nicht funktioniert:\\%
/\suppressfloats[t,b]/\\
• Unterdrückt Positionierung am Kopf oder Fuß der Seite
• vermeidet Bilder eines neuen Abschnittes im alten
• nützliche Pakete:
• /placeins/
• /afterpage/
• /endfloat/
\•
\end{frame}

\begin{frame}[fragile]{table}
\begin{LTXexample}
\begin{table}
\begin{tabular}{ccc}
a & b & c
\end{tabular}
\caption{Eine sinnlose Tabelle}
\label{tab:sinnlos}
\end{table}
Im Text kann man auf Tabelle
\ref{tab:sinnlos} verweisen.
\end{LTXexample}
\begin{table}
\begin{tabular}{ccc}
a & b & c
\end{tabular}
\caption{Eine sinnlose Tabelle}
\label{tab:sinnlos}
\end{table}
\end{frame}

\begin{frame}[fragile]{Nichtgleitende Gleitumgebungen}
• nichtgleitende Umgebungen als Gleitumgebungen ausgeben:
•[] Paket \pkg{caption}
\•
\begin{LTXexample}[pos=b]
Eine kleine Abbildung in einem Text, die eigentlich gar keine ist:
\begin{minipage}[b]{3cm}
\fbox{ich bin kein Bild}
\captionof{figure}{test}
\end{minipage}
In der /minipage/ kann jeder beliebige Inhalt stehen \dots
\end{LTXexample}
\end{frame}

\begin{frame}[fragile]{caption}
• /caption/ bietet auch vielfältige Einstellungen für Legenden:
\•
\begin{LTXexample}[pos=b]
\captionsetup[figure]{textfont=bf, labelsep=period}

\captionsetup[table]{textfont=it,singlelinecheck=false,
labelsep=newline,format=plain,justification=justified}

\begin{figure}\fbox{Bild mit \emph{nicht} angepasster Unterschrift – dank Beamer …}
\caption{Unterschrift}\end{figure}
\end{LTXexample}
\end{frame}

\begin{frame}[fragile]{Drehen von Gleitumgebungen}
• Paket /rotating/ rotiert den Inhalt um 90° bzw. 270°
• Umgebungen /sidewaysfigure/, /sidewaystable/
• nichtgleitend: /sideways/
\•
\begin{LTXexample}[width=.4\textwidth]
\centering
\begin{sideways}
[Bild]
\end{sideways}
\captionof{figure}{Nicht gedrehte Beschriftung}
\end{LTXexample}
\end{frame}

\begin{frame}[fragile]{sideways}
\begin{lstlisting}
\begin{sidewaysfigure}
\fbox{Bild}
\caption{Unterschrift}
\end{sidewaysfigure}
\end{lstlisting}
\end{frame}

\begin{frame}{Tabellen und \LaTeX}
•[\color{red}⇒] Tabellensatz mit \LaTeX\ ist aufwändig!
•[\color{red}⇒] WYSIWYG-Editoren bieten leichtere, da sichtbare Formatierung von Tabellen.
•[\bf \color{green}⇒] Ergebnis sieht in \LaTeX\ meist besser aus.
•[\color{green}⇒] Erscheinungsbild ist frei anpassbar, mit beliebig hohem Aufwand.
\•
\end{frame}

\section[tabular]{Standardumgebungen – tabular, tabular*}
\begin{frame}[fragile]{\LaTeX{}s Standardumgebungen}
• /tabular/, /tabular*/
• /tabbing/
• \alert{nicht zu verwechseln mit /table/!}
\•
\end{frame}

\begin{frame}[fragile]{tabular vs. tabbing}
\begin{tabular}[]{rcc}
& \textbf{tabular} & \textbf{tabbing}\\
Eigener Absatz & \cross & \checkmark \\
Seitenumbruch & \cross & \checkmark  \\
automatische Spaltenbreite & \checkmark & \cross\\
Schachtelung & \checkmark & \cross
\end{tabular}
\end{frame}

\begin{frame}[fragile]{tabbing}
Grundbefehle: /\=, \>/
\begin{LTXexample}
\begin{tabbing}
erster Eintrag \= zweiter \= dritter \\
eins \> zwei \> drei\\
eins \>      \> \` drei
\end{tabbing}
\end{LTXexample}
/\=/ definiert eine neue Tabulatorposition\\
/\>/ rückt zur nächsten definierten Position vor
\end{frame}

\begin{frame}[fragile]{tabbing}
Weitere Befehle: /\kill, \`/
\begin{LTXexample}
\begin{tabbing}
\hspace{1.5cm} \= \hspace{1cm} \= \qquad \kill
erster \> zweiter \> dritter \\
erster Eintrag \> zweiter Eintrag \` dritter Eintrag
\end{tabbing}
\end{LTXexample}
\begin{description}
\item[/\textbackslash kill/] löscht Inhalt der Zeile, speichert aber die Tabulatoren
\item[/\textbackslash`/] richtet Text rechtsbündig zum /tabbing/-Rand aus
\end{description}
\end{frame}

\begin{frame}[fragile]{tabular}
/tabular, tabular*/
\begin{LTXexample}[pos=b,preset={\small}]
\begin{tabular}{l|c||r|p{2cm}@{\ding{53}}c|}
links & mitte & rechts & vier & fünf\\\hline\hline
links & mitte &  & eine lange vierte Spalte, die umbrochen wird\\\hline
& & & &
\end{tabular}
\end{LTXexample}
\end{frame}

\begin{frame}[fragile]{tabular}
\begin{mydesc}
l& linksbündige Spalte\\
c & zentrierte Spalte\\
r & rechtbündige Spalte\\
| & vertikale Linie zwischen Spalten\\
|| & doppelte Linie zwischen Spalten (wird nicht durchgestrichen)\\
p\{breite\} & Fügt eine /\parbox[t]{breite}/ ein\\
@\{Inhalt\} & setzt statt Spaltenabstand /Inhalt/\\
*\{n\}\{kürz\} & setzt /n/ mal das /kürzel/, z.\,B. /*{2}{|}/\\
\end{mydesc}
\end{frame}

\section[booktabs]{Schöne Tabellen – booktabs}

\begin{frame}[fragile]{Fragwürdiges Layout}
• Paket booktabs (Simon Fear) für hohe Qualität
• Empfehlungen aus dem Paket:
\•
\begin{block}{booktabs}
\begin{enumerate}
\item \alert{Never, ever use vertical rules.}
\item \alert{Never use double rules.}\pause 
\item Put the units in the column heading (not in the body of the table).
\item Always precede a decimal point by a digit; thus /0.1/ \emph{not} just /.1/.
\item Do not use “ditto” signs or any other such convention to repeat a previous
value. In many circumstances a blank will serve just as well. If it won’t,
then repeat the value.
\end{enumerate}
\end{block}
\end{frame}

\begin{frame}[fragile]{booktabs}
\begin{LTXexample}[width=.45\textwidth,rframe={}]
\begin{tabular}{lrr}
\toprule
Artikel & Zahl & Bezeichnung\\\midrule
Die & erste & Zeile\\\cmidrule{2-3}
Die & zweite & Zeile\\
Die & dritte & Zeile\\
Die & vierte & Zeile\\
\bottomrule
\end{tabular}
\end{LTXexample}
\end{frame}

\begin{frame}[fragile]{ohne booktabs}{Negativbeispiel}
\begin{LTXexample}[width=.45\textwidth,rframe={},preset={\def\ditto{\ --"{}--\ }}]
\begin{tabular}{l||r|r}
\hline
Artikel & Zahl & Bezeichnung\\\hline
Die & erste & Zeile\\\cline{2-3}
Die & zweite & Zeile\\
Die & dritte & \ditto \\
Die & vierte & \ditto \\
\hline
\end{tabular}
\end{LTXexample}
\end{frame}

\section[array]{Erweiterungen – array}
\begin{frame}[fragile]{array}
• Paket /array/ erweitert die Möglichkeiten von /tabular/
• Änderung von vertikalen Linien, neue Spaltentypen:
\•
\begin{mydesc}
/|/ & berücksichtigt die Linienbreite\\\hline
/m{breite}/ & vertikal zentrierte Spalte der angegebenen /breite/\\
/b{breite}/ & unten ausgerichtete Spalte der angegebenen /breite/ (vgl. p)\\
/>{Befehl}/ & fügt /Befehl/ direkt vor der nächsten Spalte ein\\
/<{Befehl}/ & fügt /Befehl/ direkt hinter der letzten Spalte ein\\
/!{Befehl}/ & wie /|/, fügt aber /Befehl/ ein. Vgl. /@/, aber Abstand korrigiert
\end{mydesc}
\end{frame}

\begin{frame}[fragile]{array}
\begin{LTXexample}[pos=b]
\begin{tabular*}{6cm}{|p{1cm}p{3cm}p{1cm}|}
links & mittlerer Text mit eingebautem Umbruch & rechts
\end{tabular*}
\end{LTXexample}
\end{frame}
\begin{frame}[fragile]{array}
\begin{LTXexample}[pos=b]
\begin{tabular*}{6cm}{|m{1cm}m{3cm}m{1cm}|}
links & mittlerer Text mit eingebautem Umbruch & rechts
\end{tabular*}
\end{LTXexample}
\end{frame}
\begin{frame}[fragile]{array}
\begin{LTXexample}[pos=b]
\begin{tabular*}{6cm}{|b{1cm}b{3cm}b{1cm}|}
links & mittlerer Text mit eingebautem Umbruch & rechts
\end{tabular*}
\end{LTXexample}
\end{frame}
\begin{frame}[fragile]{array}
\begin{LTXexample}[pos=b]
\begin{tabular}{>{\bfseries}l|>{\color{red}}r}
links & rechts\\
links & rechts
\end{tabular}
\end{LTXexample}
\end{frame}

\section[tabularx, tabulary]{Automatische Breite – tabularx, tabulary}
\begin{frame}[fragile]{tabular*}
• /tabular*/ ändert \emph{Abstand} der Spalten
• /tabularx/ verteilt \emph{Breite} der Spalten \emph{gleichmäßig}
• /tabulary/ verteilet \emph{Breite} der Spalten \emph{am Inhalt orientiert}
\•
\end{frame}

\begin{frame}[fragile]{automatische Breiten}
\begin{LTXexample}[width=.4\textwidth]
\begin{tabular*}{4cm}{|l|!{\extracolsep\fill}>{(}l<{)}|r|}
a a & b b & c c
\end{tabular*}
\\ \\
\begin{tabular}{|l|!{\extracolsep\fill}l|r|}
a a & b b & c c
\end{tabular}
\\ \\
\begin{tabularx}{4cm}{|l|>{(}X<{)}|r|}
a a & b b & c c
\end{tabularx}
\end{LTXexample}
\end{frame}

\begin{frame}[fragile]{tabularx}
Automatische Berechnung der Spaltenbreite:
\begin{LTXexample}
\begin{tabularx}{\linewidth}{lX|X|r}
linke Spalte & Eine längere Spalte& kurz & rechts
\end{tabularx}
\end{LTXexample}
\end{frame}

\begin{frame}[fragile]{tabulary}
\begin{LTXexample}
\begin{tabulary}{4cm}{|L|L|L|}
a & b b b b b b b b b & c c c c c c c c c c c c c c c c c 
\end{tabulary}
\end{LTXexample}
\begin{LTXexample}
\begin{tabular}{|l|l|l|}
a & b b b b b b b b b & c c c c c c c c c c c c c c c c c 
\end{tabular}
\end{LTXexample}
\begin{LTXexample}
\begin{tabular*}{4cm}{|l|l|l|}
a & b b b b b b b b b & c c c c c c c c c c c c c c c c c 
\end{tabular*}
\end{LTXexample}
\end{frame}

\begin{frame}[fragile]{tabulary}
Mögliche Spaltentypen:\\[1em]
\begin{mydesc}
L & linksbündig\\
R & rechtsbündig\\
C & zentriert\\
J & Blocksatz
\end{mydesc}
\\[1em]
• Alle Spalten verhalten sich wie /p/-Spalten.
• Breite der Spalten ist \emph{nicht} vorher festgelegt.
\•
\end{frame}

\section[supertabular, longtable]{Mehrseitige Tabellen – supertabular, longtable}
\begin{frame}[fragile]{lange Tabellen}
Lösung: /supertabular/ oder /longtable/\\[1em]
\begin{mydesc}
supertabular   & mehrseitige Tabelle, Breite variabel\\
supertabular*  & festgesetzte Breite\\
mpsupertabular & setzt Tabelle in /minipage/\\
mpsupertabular*& /minipage/ mit fester Breite
\end{mydesc}
\end{frame}

\begin{frame}[fragile]{supertabular}
\begin{supertabular}{cc}
\messdaten
\end{supertabular}
\end{frame}

\begin{supertabular}{cc}
\messdaten
\end{supertabular}

\begin{frame}[fragile]{supertabular}
Wichtige Einstellungsmöglichkeiten:
\begin{LTXexample}
\tablehead{links & rechts \\\hline}
\tablefirsthead{\bf links & \bf rechts \\}
\tabletail{\small \textit{Fortsetzung auf der nächsten Seite} & \\}
\tablelasttail{Ende der Messdaten}
\end{LTXexample}
\end{frame}

\begin{supertabular}{cc}
\messdaten
\end{supertabular}

\begin{frame}[fragile]{longtable}
Paket /longtable/ bietet Umgebung /longtable/:
• feste Breite der Spalten auf allen Seiten
• /head/, /firsthead/ etc. werden \emph{innerhalb} der Tabelle festgelegt
• verwendet die .aux-Datei (auf Schreibrechte achten!)
\• 
\end{frame}

\begin{frame}[fragile]{longtable}
\begin{LTXexample}[pos=b]
\begin{longtable}{cc}
\textbf{Messdaten}\\
\endfirsthead
links & rechts\\
\endhead
\small \textit{Weiter auf der nächsten Seite}
\endfoot
Ende der Tabelle.
\endlastfoot
\messdaten
\end{longtable}
\end{LTXexample}
\end{frame}

\begin{longtable}{cc}
\textbf{Messdaten}\\
\endfirsthead
links & rechts\\
\endhead
\small \textit{Weiter auf der nächsten Seite}
\endfoot
Ende der Tabelle.
\endlastfoot
\messdaten
\end{longtable}

\begin{frame}[fragile]{supertabularx, longtablex}
Für Satz mehrseitiger Tabellen mit automatischer Breitenanpassung:
\only<1>{/supertabularx/ bzw. /longtablex/}
\only<2->{{\color{red}\xout{|supertabularx|}} bzw. {\color{red}\xout{|longtabularx|}}}
\pause\pause
• Paket /ltxtable/ bietet grundlegende Unterstützung
• Kombination von /longtable/ und /tabularx/
• Tabelle (\texttt{tabularx}) selbst steht in externer Datei
• Nutzer muss diese selbst anlegen, schreiben und verwalten
• Einbinden mittels \texttt{\textbackslash LTXtable\{width\}\{file\}}
• am besten mittels /filecontents/ (Umgebung, Paket)
\•
\end{frame}

\section[weiteres]{weitere nützliche Pakete}
\begin{frame}[fragile]{weitere nützliche Pakete}
\begin{mydesc}
colortbl & farbige Linien\\
hhline   & vielfältige Linien (horizontal, vertikal \dots)\\
arydshln & gestrichelte Linien \\
tabls    & Zeilenabstände einstellen (inkompatipel zu array!)\\
multirow & vertikale Ausrichtung \\
dcolumn  & Ausrichtung am Dezimalpunkt \\
threeparttable & Fußnoten an Tabellen
\end{mydesc}
\end{frame}
\end{document}