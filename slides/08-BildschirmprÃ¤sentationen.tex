\makeatletter
\@ifundefined{draftmodeon}{
  \documentclass[german,t]{beamer}
}{
  \documentclass[german,t,draft]{beamer}
}
\makeatother
\mode<presentation>{
  \usetheme{Frankfurt}
  \useoutertheme{infolines}
  \usecolortheme[RGB={0,100,130}]{structure}
  \useinnertheme{rounded}
  \setbeamertemplate{navigation symbols}{} 
}

\newcommand\subtitlei[1]{\def\insertsubtitlei{#1}}
\newcommand\subtitleii[1]{\def\insertsubtitleii{#1}}

\setbeamertemplate{title page}
  {
    \vspace*{1cm}
    \begin{centering}
      {\usebeamerfont{title}\usebeamercolor[bg]{title}
      {\Large \inserttitle}
      \\[.3cm]
      \insertsubtitlei\\
      \large \color{black}\insertsubtitleii
      }
\vspace*{1cm}
    \vbox to 1cm {\hfill \includegraphics[width=.2\textwidth]{ctanlion}}
      \usebeamerfont{subtitle}\usebeamercolor[bg]{subtitle}
      {\color{black}\Large \insertsubtitle}\vfill
    {\hfill \insertdate \hfill}
    \end{centering}

    \vfill\vfill

    \hspace*{-.47cm}
    \usebeamertemplate*{footline}
    \vspace*{-1.15cm}
}

\setbeamercolor*{author in head/foot}{parent=palette tertiary}
\setbeamercolor*{title in head/foot}{parent=palette secondary}
\setbeamercolor*{date in head/foot}{parent=palette primary}

\setbeamercolor*{section in head/foot}{parent=palette tertiary}
\setbeamercolor*{subsection in head/foot}{parent=palette primary}

\defbeamertemplate*{headline}{infolines theme changed}
{
  \leavevmode%
  \hbox{%
  \begin{beamercolorbox}[wd=.3\paperwidth,ht=2.25ex,dp=1.5ex,center]{title in head/foot}%
    \LaTeX-Kurs 2010
  \end{beamercolorbox}%
  \begin{beamercolorbox}[wd=.4\paperwidth,ht=2.25ex,dp=1.2ex,right]{section in head/foot}%
     \hfill \coursenumber\ – \coursetitle \hfill\hfill
  \end{beamercolorbox}%
  \begin{beamercolorbox}[wd=.3\paperwidth,ht=2.25ex,dp=1.2ex,left]{subsection in head/foot}%
    \hspace*{2em}\insertsectionhead
  \end{beamercolorbox}}%
  \vskip0pt%
}

\defbeamertemplate*{footline}{infolines theme changed}
{
  \leavevmode%
  \hbox{%
  \begin{beamercolorbox}[wd=.3\paperwidth,ht=2.25ex,dp=1ex,center]{author in head/foot}%
      \insertshortauthor
  \end{beamercolorbox}%
  \begin{beamercolorbox}[wd=.4\paperwidth,ht=2.25ex,dp=1ex,center]{title in head/foot}%
    \insertshortdate{}
  \end{beamercolorbox}%
  \begin{beamercolorbox}[wd=.3\paperwidth,ht=2.25ex,dp=1ex,center]{author in head/foot}%
    \insertframenumber{} / \inserttotalframenumber
  \end{beamercolorbox}}%
  \vskip0pt%
}

\setbeamersize{text margin left=1em,text margin right=1em}

\def\extractnumber"#1-#2".{#1}
\def\extracttitle"#1-#2".{#2}
\def\coursenumber{\expandafter\extractnumber\jobname.}
\def\coursetitle{\expandafter\extracttitle\jobname.}

\usepackage{
  babel,
  dtklogos,
  moreverb,
  shortvrb,
  xltxtra,
  yfonts
}
\usepackage[final]{showexpl}

\hypersetup{
  colorlinks=true,
  breaklinks=true,
  linkcolor=blue,
  urlcolor=blue,
  citecolor=black, filecolor=black, menucolor=black,
  pdfauthor={Arno L. Trautmann},
}

\MakeShortVerb|

\author{Arno Trautmann}
\institute{Heidelberg}
\title{Einführung in das Textsatzsystem\\[2ex] \Huge \LaTeX}
\subtitlei{Vorlesungsreihe im Sommersemester 2010}
\subtitleii{\textfrak{universitatis:~studii~heydelbergensis:}}
\subtitle{\coursenumber\ – \coursetitle}

\graphicspath{{../Mediales/}}

\logo{}%\includegraphics[width=7.4em]{unilogo.svg}}

\defaultfontfeatures{Scale=MatchLowercase}
\setmonofont{DejaVu Sans Mono}

\newenvironment{mydesc}{\begin{tabular}{|>{\columncolor{lightgray}\color{blue}}rl|}\hline}{\\\hline\end{tabular}}

\def\pkg#1{\texttt{#1}}
\def\macro#1{|#1|}

\def\plainTeX{\textsf{plain\TeX}}

\catcode`\⇒=\active
\def⇒{\ensuremath{\Rightarrow}}
\catcode`\⇐=\active
\def⇐{\ensuremath{\Leftarrow}}
\catcode`\…13
\let…\dots

\newcommand\notiz[1]{}
\newcommand\einschub[1][]{\textcolor{red}{⇒}}
\renewcommand\checkmark{\color{green}\ding{51}}
\newcommand\cross{\color{red}\ding{53}}

\newcommand\pdf[2][]{\bgroup
  \setbeamercolor{background canvas}{bg=}%
  \includepdf[#1]{#2}%
    \egroup
}

\newenvironment{twoblock}[2]{
  \begin{columns}
  \begin{column}{.46\textwidth}
  \begin{block}{#1}
\def\nextblock{
  \end{block}
  \end{column}
\ %
  \begin{column}{.46\textwidth}
  \begin{block}{#2}
}
}
{
  \end{block}
  \end{column}
  \end{columns}
}

\AtBeginDocument{
  \lstset{%
    backgroundcolor=\color[rgb]{.9 .9 .9},
    basicstyle=\ttfamily\small,
    breakindent=0em,
    breaklines=true,
    commentstyle=,
    keywordstyle=,
    identifierstyle=,
    captionpos=b,
    numbers=none,
    frame=tlbr,%shadowbox,
    frameround=tttt,
    pos=r,
    rframe={single},
    explpreset={numbers=none}
  }
}
\makeatletter
\g@addto@macro\beamer@lastminutepatches{ % thanks to Ulrike for this!
  \frame[plain,t]{\titlepage}
  \frame{\centerline{\huge \color[RGB]{0,100,130}Inhalt}\tableofcontents}
}

%—— itemize-hack
\def\outside{o}
\def\inside{i}
\let\insideitemizei\outside
\let\insideitemizeii\outside
\def\altenditemize{
  \if\altlastitem 1%
    \let\altlastitem0%
  \else%
    \end{itemize}%
    \let\insideitemizei\outside%
  \fi%
}

\begingroup
  \lccode`\~=`\^^M%
\lowercase{%
  \endgroup
  \def\makeenteractive{%
    \catcode`\^^M=\active
    \let~\altenditemize
}%
}

\def\newitemi{%
  \ifx\insideitemizei\inside%
    \let\altlastitem1%
    \expandafter\item%
  \else%
    \begin{itemize}%
    \let\insideitemizei\inside%
    \let\altlastitem1%
    \makeenteractive%
    \expandafter\item%
  \fi
}

\def\newitemii{
  \ifx\insideitemizeii\inside
    \expandafter\item%
  \else
    \begin{itemize}
      \let\insideitemizeii\inside
      \expandafter\item%
  \fi
}

\def\makeitemi#1{%
  \expandafter\ifx\csname cc\string#1\endcsname\relax
    \add@special{#1}%
    \expandafter
    \xdef\csname cc\string#1\endcsname{\the\catcode`#1}%
    \begingroup
      \catcode`\~\active  \lccode`\~`#1%
      \lowercase{%
      \global\expandafter\let
         \csname ac\string#1\endcsname~%
      \expandafter\gdef\expandafter~\expandafter{\newitemi}}%
    \endgroup
    \global\catcode`#1\active
  \else
  \fi
}

\def\makeitemii#1{%
  \expandafter\ifx\csname cc\string#1\endcsname\relax
    \add@special{#1}%
    \expandafter
    \xdef\csname cc\string#1\endcsname{\the\catcode`#1}%
    \begingroup
      \catcode`\~\active  \lccode`\~`#1%
      \lowercase{%
      \global\expandafter\let
         \csname ac\string#1\endcsname~%
      \expandafter\gdef\expandafter~\expandafter{\newitemii}}%
    \endgroup
    \global\catcode`#1\active
  \else
  \fi
}

\def\add@special#1{%
  \rem@special{#1}%
  \expandafter\gdef\expandafter\dospecials\expandafter
{\dospecials \do #1}%
  \expandafter\gdef\expandafter\@sanitize\expandafter
{\@sanitize \@makeother #1}}
\def\rem@special#1{%
  \def\do##1{%
    \ifnum`#1=`##1 \else \noexpand\do\noexpand##1\fi}%
  \xdef\dospecials{\dospecials}%
  \begingroup
    \def\@makeother##1{%
      \ifnum`#1=`##1 \else \noexpand\@makeother\noexpand##1\fi}%
    \xdef\@sanitize{\@sanitize}%
  \endgroup}
\AtBeginDocument{
  \makeitemi{•}
}
%——beamer versaut hier irgendwas, daher muss itemize explizit beendet werden!
\def\•{\end{itemize}}
%—— itemize-hack
\makeatother

\def\TikZ{{\rmfamily Ti\textit{k}Z}\xspace}

\begin{document}
\section{beamer}
\frame{\tableofcontents}
\begin{frame}{Vorbemerkungen}
• \LaTeX\ ist \emph{nicht} für Präsentationen gedacht
• spezielle Programme häufig besser geeignet
• Wahl des Programms vom Inhalt abhängig:
• strukturierte, klare Darstellungsform: \pkg{beamer}
\•
\end{frame}

\begin{frame}{Vorbemerkungen: pgf}
• \pkg{beamer} baut auf \pkg{pgf} auf
• \pkg{pgf}: portable graphics format (oder „pretty, good, functional“)
• \pkg{pgf} besteht aus drei verschiedenen Ebenen:
•[1] Systemebene
•[2] Basisebene
•[3] Frontend (Nutzerebene)
\•
\end{frame}

\begin{frame}[fragile]{pgf}{Systemebene}
• Abstraktion von Treibern
• Unabhängigkeit von genauen Abläufen der Treiber
• Portabilität, Stabilität, leichte Erweiterung auf neue Treiber
• unterschiedliche |\special|-Befehle je nach Treiber
• so minimalistisch wie möglich (jeder Befehl muss im Treiber umgesetzt werden)
• kann z.\,B. keine Kreise, nur Bézier-Kurven
• Nutzer muss sich nicht um Treiberabhängigkeiten kümmern
\•
\end{frame}

\begin{frame}{pgf}{Basisebene}
• bietet Basisbefehle (z.\,B. Befehl für Kreis)
• besteht aus verschiedenen Modulen:
• \emph{core}, bietet die Grundfunktionalität (mehrere Module, die zusammen benötigt werden)
• weitere optionale Module (node management, plotting …)
\•
\end{frame}

\begin{frame}[fragile]{pgf}{Frontend (Nutzerebene)}
• Vereinfacht die Benutzung der Basisebene (vgl. Makropaket für \TeX)
• \TikZ ist die normale Nutzerebene von \pkg{pgf}
• \pkg{pgfpict2e} ist eine Reimplementierung von \LaTeX s |{picture}|-Umgebung
• \pkg{beamer} ist eine spezialisierte Nutzerebene
\•
\end{frame}

\begin{frame}{Beamer}
• Dokumentklasse \pkg{beamer} ermöglich Satz von Präsentationen\pause
• Erstellen von bildschirmfüllenden „Folien“
• ansprechende Farbgebung
• strukturierte Darstellung des Inhaltes
• dynamische Effekte
• multimediale Unterstützung
\•
\end{frame}

\subsection{Verwendung}
\begin{frame}[fragile]{Aufbau einer Präsentation}
• |\documentclass{beamer}|
• alle Pakete, Befehle, Umgebungen wie normal zu verwenden (fast)
• |\tableofcontents| erzeugt Inhaltsverzeichnis, |\begin{tabular}| setzt Tabelle etc.
• \alert{wichtigste Umgebung:} |\begin{frame}|
\•
\end{frame}

\begin{frame}[fragile]{Besonderheiten}{frames}
• Umgebung |frame| erzeugt eine „Folie“
• erstes Argument: Titel, zweites: Untertitel
• optionales Argument |[fragile]| nötig für |\verb| u.\,ä.
• Jede pdf-Seite ist (meist) statisches Objekt
•[⇒] Überblendeffekte benötigen mehrere Folien
\• 
\end{frame}

\begin{frame}[fragile]{Besonderheiten}{Überblendeffekte}
• für dynamische Effekte: <>
•<+-> |<+->| lässt Objekt erscheinen und bleibt
•<+> |<+>| lässt Objet erscheinen, verschwindet wieder
•<4> |<4>| Objekt erscheint auf Folie 4
\•
\end{frame}

\subsection{Anpassung}
\begin{frame}[fragile]{themes}{allgemeine}
• themes sind Stilvorlagen, die das Layout gesamt setzen
• Einbinden mittels |\usetheme|
• Benannt nach Tagungsorten 
\•
\end{frame}

\begin{frame}[fragile]{themes}{inner}
• beeinflussen das Aussehen von Elementen in der Folie
• Aufzählungen, Abbildungsbeschriftung, Boxen etc.
• |\useinnertheme|
\• 
\end{frame}

\begin{frame}[fragile]{themes}{outer}
• beeinflussen das Aussehen der äußeren Element
• Kopfzeile, Fußzeile, Navigation etc.
• |\useoutertheme|
\• 
\end{frame}

\begin{frame}[fragile]{themes}{color}
• wie der Name sagt …
• je nach Theme werden verschiedene Elemente coloriert
• Farben anpassbar:,
• |\setbeamercolor{footnote}{fg=red}|
\•
\end{frame}

\begin{frame}{Strukturelemente}
• block
• theorem
• 
\•
\end{frame}

\section{impressive}
\begin{frame}{impressive!}
• für Präsentationen: jeder pdf-viewer geeignet
• Programm \pkg{impressive!} ermöglicht nützliche Präsentationseffekte
• Start aus Kommandozeile
• Effekte nur über Kommandozeilenargumente steuerbar
\•
\end{frame}

\section{Rollentausch}
\begin{frame}[fragile]{Mitschreiben}
• in Vorlesungen oder Übungen mit\TeX{}en manchmal nützlich
• entweder extrem hohe Tippgeschwindigkeit nötig
• oder durchdachte Befehlsdefinitionen
• \emph{wichtig:} alle strukturelle Information muss vorhanden sein!\\%
(auch, wenn es nicht gut aussieht)
\• 
\end{frame}

\begin{frame}[fragile]{Mitschreiben}
• häufig nur stichpunktartiges Aufschreiben
•[⇒] |\obeylines|
• Aufzählungen abkürzen, z.\,B. mittels |\let•\item|
• …
\•
\end{frame}

\end{document}