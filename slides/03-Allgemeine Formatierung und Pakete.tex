\makeatletter
\@ifundefined{draftmodeon}{
  \documentclass[german,t]{beamer}
}{
  \documentclass[german,t,draft]{beamer}
}
\makeatother
\mode<presentation>{
  \usetheme{Frankfurt}
  \useoutertheme{infolines}
  \usecolortheme[RGB={0,100,130}]{structure}
  \useinnertheme{rounded}
  \setbeamertemplate{navigation symbols}{} 
}

\newcommand\subtitlei[1]{\def\insertsubtitlei{#1}}
\newcommand\subtitleii[1]{\def\insertsubtitleii{#1}}

\setbeamertemplate{title page}
  {
    \vspace*{1cm}
    \begin{centering}
      {\usebeamerfont{title}\usebeamercolor[bg]{title}
      {\Large \inserttitle}
      \\[.3cm]
      \insertsubtitlei\\
      \large \color{black}\insertsubtitleii
      }
\vspace*{1cm}
    \vbox to 1cm {\hfill \includegraphics[width=.2\textwidth]{ctanlion}}
      \usebeamerfont{subtitle}\usebeamercolor[bg]{subtitle}
      {\color{black}\Large \insertsubtitle}\vfill
    {\hfill \insertdate \hfill}
    \end{centering}

    \vfill\vfill

    \hspace*{-.47cm}
    \usebeamertemplate*{footline}
    \vspace*{-1.15cm}
}

\setbeamercolor*{author in head/foot}{parent=palette tertiary}
\setbeamercolor*{title in head/foot}{parent=palette secondary}
\setbeamercolor*{date in head/foot}{parent=palette primary}

\setbeamercolor*{section in head/foot}{parent=palette tertiary}
\setbeamercolor*{subsection in head/foot}{parent=palette primary}

\defbeamertemplate*{headline}{infolines theme changed}
{
  \leavevmode%
  \hbox{%
  \begin{beamercolorbox}[wd=.3\paperwidth,ht=2.25ex,dp=1.5ex,center]{title in head/foot}%
    \LaTeX-Kurs 2010
  \end{beamercolorbox}%
  \begin{beamercolorbox}[wd=.4\paperwidth,ht=2.25ex,dp=1.2ex,right]{section in head/foot}%
     \hfill \coursenumber\ – \coursetitle \hfill\hfill
  \end{beamercolorbox}%
  \begin{beamercolorbox}[wd=.3\paperwidth,ht=2.25ex,dp=1.2ex,left]{subsection in head/foot}%
    \hspace*{2em}\insertsectionhead
  \end{beamercolorbox}}%
  \vskip0pt%
}

\defbeamertemplate*{footline}{infolines theme changed}
{
  \leavevmode%
  \hbox{%
  \begin{beamercolorbox}[wd=.3\paperwidth,ht=2.25ex,dp=1ex,center]{author in head/foot}%
      \insertshortauthor
  \end{beamercolorbox}%
  \begin{beamercolorbox}[wd=.4\paperwidth,ht=2.25ex,dp=1ex,center]{title in head/foot}%
    \insertshortdate{}
  \end{beamercolorbox}%
  \begin{beamercolorbox}[wd=.3\paperwidth,ht=2.25ex,dp=1ex,center]{author in head/foot}%
    \insertframenumber{} / \inserttotalframenumber
  \end{beamercolorbox}}%
  \vskip0pt%
}

\setbeamersize{text margin left=1em,text margin right=1em}

\def\extractnumber"#1-#2".{#1}
\def\extracttitle"#1-#2".{#2}
\def\coursenumber{\expandafter\extractnumber\jobname.}
\def\coursetitle{\expandafter\extracttitle\jobname.}

\usepackage{
  babel,
  dtklogos,
  moreverb,
  shortvrb,
  xltxtra,
  yfonts
}
\usepackage[final]{showexpl}

\hypersetup{
  colorlinks=true,
  breaklinks=true,
  linkcolor=blue,
  urlcolor=blue,
  citecolor=black, filecolor=black, menucolor=black,
  pdfauthor={Arno L. Trautmann},
}

\MakeShortVerb|

\author{Arno Trautmann}
\institute{Heidelberg}
\title{Einführung in das Textsatzsystem\\[2ex] \Huge \LaTeX}
\subtitlei{Vorlesungsreihe im Sommersemester 2010}
\subtitleii{\textfrak{universitatis:~studii~heydelbergensis:}}
\subtitle{\coursenumber\ – \coursetitle}

\graphicspath{{../Mediales/}}

\logo{}%\includegraphics[width=7.4em]{unilogo.svg}}

\defaultfontfeatures{Scale=MatchLowercase}
\setmonofont{DejaVu Sans Mono}

\newenvironment{mydesc}{\begin{tabular}{|>{\columncolor{lightgray}\color{blue}}rl|}\hline}{\\\hline\end{tabular}}

\def\pkg#1{\texttt{#1}}
\def\macro#1{|#1|}

\def\plainTeX{\textsf{plain\TeX}}

\catcode`\⇒=\active
\def⇒{\ensuremath{\Rightarrow}}
\catcode`\⇐=\active
\def⇐{\ensuremath{\Leftarrow}}
\catcode`\…13
\let…\dots

\newcommand\notiz[1]{}
\newcommand\einschub[1][]{\textcolor{red}{⇒}}
\renewcommand\checkmark{\color{green}\ding{51}}
\newcommand\cross{\color{red}\ding{53}}

\newcommand\pdf[2][]{\bgroup
  \setbeamercolor{background canvas}{bg=}%
  \includepdf[#1]{#2}%
    \egroup
}

\newenvironment{twoblock}[2]{
  \begin{columns}
  \begin{column}{.46\textwidth}
  \begin{block}{#1}
\def\nextblock{
  \end{block}
  \end{column}
\ %
  \begin{column}{.46\textwidth}
  \begin{block}{#2}
}
}
{
  \end{block}
  \end{column}
  \end{columns}
}

\AtBeginDocument{
  \lstset{%
    backgroundcolor=\color[rgb]{.9 .9 .9},
    basicstyle=\ttfamily\small,
    breakindent=0em,
    breaklines=true,
    commentstyle=,
    keywordstyle=,
    identifierstyle=,
    captionpos=b,
    numbers=none,
    frame=tlbr,%shadowbox,
    frameround=tttt,
    pos=r,
    rframe={single},
    explpreset={numbers=none}
  }
}
\makeatletter
\g@addto@macro\beamer@lastminutepatches{ % thanks to Ulrike for this!
  \frame[plain,t]{\titlepage}
  \frame{\centerline{\huge \color[RGB]{0,100,130}Inhalt}\tableofcontents}
}

%—— itemize-hack
\def\outside{o}
\def\inside{i}
\let\insideitemizei\outside
\let\insideitemizeii\outside
\def\altenditemize{
  \if\altlastitem 1%
    \let\altlastitem0%
  \else%
    \end{itemize}%
    \let\insideitemizei\outside%
  \fi%
}

\begingroup
  \lccode`\~=`\^^M%
\lowercase{%
  \endgroup
  \def\makeenteractive{%
    \catcode`\^^M=\active
    \let~\altenditemize
}%
}

\def\newitemi{%
  \ifx\insideitemizei\inside%
    \let\altlastitem1%
    \expandafter\item%
  \else%
    \begin{itemize}%
    \let\insideitemizei\inside%
    \let\altlastitem1%
    \makeenteractive%
    \expandafter\item%
  \fi
}

\def\newitemii{
  \ifx\insideitemizeii\inside
    \expandafter\item%
  \else
    \begin{itemize}
      \let\insideitemizeii\inside
      \expandafter\item%
  \fi
}

\def\makeitemi#1{%
  \expandafter\ifx\csname cc\string#1\endcsname\relax
    \add@special{#1}%
    \expandafter
    \xdef\csname cc\string#1\endcsname{\the\catcode`#1}%
    \begingroup
      \catcode`\~\active  \lccode`\~`#1%
      \lowercase{%
      \global\expandafter\let
         \csname ac\string#1\endcsname~%
      \expandafter\gdef\expandafter~\expandafter{\newitemi}}%
    \endgroup
    \global\catcode`#1\active
  \else
  \fi
}

\def\makeitemii#1{%
  \expandafter\ifx\csname cc\string#1\endcsname\relax
    \add@special{#1}%
    \expandafter
    \xdef\csname cc\string#1\endcsname{\the\catcode`#1}%
    \begingroup
      \catcode`\~\active  \lccode`\~`#1%
      \lowercase{%
      \global\expandafter\let
         \csname ac\string#1\endcsname~%
      \expandafter\gdef\expandafter~\expandafter{\newitemii}}%
    \endgroup
    \global\catcode`#1\active
  \else
  \fi
}

\def\add@special#1{%
  \rem@special{#1}%
  \expandafter\gdef\expandafter\dospecials\expandafter
{\dospecials \do #1}%
  \expandafter\gdef\expandafter\@sanitize\expandafter
{\@sanitize \@makeother #1}}
\def\rem@special#1{%
  \def\do##1{%
    \ifnum`#1=`##1 \else \noexpand\do\noexpand##1\fi}%
  \xdef\dospecials{\dospecials}%
  \begingroup
    \def\@makeother##1{%
      \ifnum`#1=`##1 \else \noexpand\@makeother\noexpand##1\fi}%
    \xdef\@sanitize{\@sanitize}%
  \endgroup}
\AtBeginDocument{
  \makeitemi{•}
}
%——beamer versaut hier irgendwas, daher muss itemize explizit beendet werden!
\def\•{\end{itemize}}
%—— itemize-hack
\makeatother

\begin{document}
\section{Der Satzspiegel}
\begin{frame}{Formatierung: Makrotypographie}
• „globale“ Typographie:
• Satzspiegel
• Gestaltung von Kopf- und Fußzeilen 
• Wahl der Schriften
• Formatierung von Abständen u.\,ä.
• Inhaltsverzeichnis, Indizes, Verzeichnisse …
\• 
\end{frame}


\begin{frame}{Der Satzspiegel}
• Verhältnis von Rändern zu Textbreiten
• Ein- oder zweiseitiger Satz?
• Schriftgröße, Laufweite
• Kopf- und Fußzeilen
\•
\end{frame}

\begin{frame}{Satzspiegel mit KOMA}
• KOMA-Skript (Hauptautor Markus Kohm)
• bietet optimale Satzspiegelkonstruktion
• eigenes Paket |typearea|
• automatische Berechnung; anpassbar (z.\,B. mittelalterliche Konstruktion)
• freie Anpassung mittels |geometry|
\• 
\end{frame}

\begin{frame}[fragile]{geometry}
• Manuelle Einstellung des Satzspiegels
• Verwendung mittels |\usepackage[top=2cm,bottom=5cm]{geometry}|
• oder |\geometry{top=2cm,bottom=5cm}|
\•
\end{frame}

\begin{frame}[fragile]{geometry}
\begin{block}{Mögliche Optionen:}
\begin{verbatim}
left, right, inner, outer, hmargin
top, bottom, vmargin
margin
bindingoffset, textwidth, textheight
twocolumn, columnsep, marginparsep, footnotesep
headsep, footsep, nofoot, nohead
hoffset, voffset, offset
\end{verbatim}
\vdots
\end{block}
\end{frame}

\section{Kopf- und Fußzeilen}
\begin{frame}[fragile]{Kopf- und Fußzeilen}
• Enthalten wichtige Informationen über das Dokument:
• Lebende Kolumnentitel
• Seitenzahlen
\•

• Anpassung mittels verschiedener Pakete
• Auswahl über |\pagestyle{Seitenstil}| oder |\thispagestyle{Seitenstil}|
• Voreinstellungen: |empty|, |plain|, |headings|
\•
\end{frame}

\begin{frame}[fragile]{|fancyhdr|}
Laden des Pakets und Einstellungen im Dokumentenkopf:
\begin{lstlisting}
\usepackage{fancyhdr}
\pagestyle{fancy}
\end{lstlisting}
\begin{columns}
\begin{column}{.4\textwidth}
Einseitiger Satz:
\begin{lstlisting}
\lhead{}    \lfoot{}
\chead{}    \cfoot{}
\rhead{}    \rfoot{}
\end{lstlisting}
\end{column}
\pause
\begin{column}{.4\textwidth}
Zweiseitiger Satz (Odd, Even)
\begin{lstlisting}
\fancyhead[LO]{}
\fancyhead[RO,LE]{}
\fancyhead[CE]{}
\fancyfoot[LO]{}
\fancyfoot[RO,LE]{}
\fancyfoot[CE]{}
\end{lstlisting}
\end{column}
\end{columns}
\end{frame}

\begin{frame}[fragile]{Aus dem KOMA-Bundle: |scrpage2|}
\begin{lstlisting}
\usepackage[automark]{scrpage2}
\pagestyle{scrheadings}
\pagestyle{scrplain}
\lehead[scrplain-links-gerade]{scrheadings-links-gerade}
\cohead[scrplain-zentriert-ungerade]{scrheadings-zentr-ungr}
\rehead[]{}
\cefoot[]{}
\ohead[]{\pagemark}
\ihead{\headmark}
\cfoot{}
\end{lstlisting}
\end{frame}

\section{DIY: Befehle und Umgebungen}

\begin{frame}[fragile]{Makros}
• Abkürzungen, die das Leben erleichtern
• \LaTeX\ definiert große Zahl von Makros vor
• Klassen und Pakete erweitern die Möglichkeiten immens
\•
\end{frame}

\begin{frame}[fragile]{Neue Befehle}
• |\newcommand{\wasser}{H$_2$O}| ⇒ H$_2$O
• Ermöglicht Abkürzungen im Text, die häufig vorkommen\pause
• Änderung: |\renewcommand{\wasser}{H\kern-.1em$_2$\kern-.1em O}|:\\ H\kern-.1em$_2$\kern-.1em O
\• 
\end{frame}

\begin{frame}[fragile]{Leerzeichen in \TeX}
\parbox{.12\textwidth}{\alert{Achtung:}\\ \includegraphics[width=1.4cm]{lion_colour}}
\TeX\ „frisst“ gerne Leerzeichen – vor allem nach Befehlen:\\
|\wasser ist nass| ⇒ H$_2$Oist nass.

\pause
•<+-> \TeX\ liest Befehle vom |\| bis zum ersten nicht-Buchstaben%
\\ (Zahl, Klammer, Leerzeichen, Punkt, \dots)
\\ \verb*|\LaTeX   ist  manchmal   umständlich|%
•<+-> Befehle im Text immer mit |\| oder |{}| beenden:
•<+-> \verb*|\LaTeX\ ist manchmal umständlich.|
\•
\begin{block}{}\centerline{
\only<2-3>{\LaTeX ist manchmal umständlich}
\only<4>{\LaTeX\ ist manchmal umständlich}}
\end{block}
\end{frame}

\begin{frame}[fragile]{Befehle mit Argumenten}
\begin{lstlisting}
\newcommand\molekuel[3][H]{Das Molekül #1$_#2$#3}
\end{lstlisting}
• Argumente werden mit |[3]| definiert
• Optionales Argument in eckigen Klammern
• Zugriff in der Definition möglich mit |#1|
• In der Verwendung meist mit geschweiften Klammern |{Co}|
\•
\newcommand\molekuel[3][H]{Das Molekül #1$_#2$#3}
|\molekuel{2}{O}| ⇒ \molekuel{2}{O}\\
|\molekuel[Co]{7}{O}| ⇒ \molekuel[Co]{7}{O}\\
\end{frame}

\begin{frame}[fragile]{Low Level: def}
• \TeX\ kennt nur den primitiven Befehl |\def|
• low level ⇒ \alert{Vorsicht!}
• \LaTeX\ definiert |\newcommand| aus guten Gründen!\\ (schützt vor Überschreiben)
• für spezielle Ansprüche kann |\def| aber nützlich sein:
\•

\begin{lstlisting}
\def\dosomething#1#2xyz#3{some code}
\end{lstlisting}
\end{frame}

\begin{frame}[fragile]{Umgebungen}
\begin{lstlisting}
\begin{umgebung}[opt. Argumente]{evtl. Argumente}
  .
  .
\end{umgebung}
\end{lstlisting}
\pause 
• Jede Umgebung ist eine Gruppierung (wie |{}| )\\
 ⇒ alle Einstellungen innerhalb einer Umgebung sind lokal%
• am Anfang und am Ende werden Befehle ausgeführt:
•[⇒] Abstände (horizontal, vertikal), Schriftumstellungen, Setzen von Zählern
• Argumente beeinflussen das Aussehen der Umgebung (Anzahl/Ausrichtung von Tabellenspalten, Anordnung von Gleitobjekten, etc.)
\•
\end{frame}

\begin{frame}[fragile]{Wichtige Umgebungen}
\begin{tabular}{ll}
Aufzählungen & |itemize| \\
Nummerierungen & |enumerate| \\
wörtliche Wiedergabe & |verbatim| \\
zweispaltiger Satz & |twocolumn|\\
Zitate & |quotation| \\
zentiert & |centering|\\
abgeschlossene Einheit & |minipage|\\
Tabellen & |tabular|, |tabularx|, |tabulary|,\\
         & |supertabular| etc. \\
Bilder & |figure| \\
Gleitumgebung & |table| \\
Beamerfolie & |frame| \\
Gleichungen & |align| (Mathe)\\
Matrizen & |matrix| (Mathe)\\
\end{tabular}
\end{frame}

\begin{frame}[fragile]{Einfache Listen}
\begin{columns}
\begin{column}{.4\textwidth}
\begin{lstlisting}
\begin{itemize}
\item Erster Punkt
\item Zweiter Punkt
\item[3] dritter Punkt
\end{itemize}
\end{lstlisting}
\begin{itemize}
\item Erster Punkt
\item Zweiter Punkt
\item[3] dritter Punkt
\end{itemize}
\end{column}
\begin{column}{.4\textwidth}
  \begin{lstlisting}
\begin{enumerate}
\item Erster Punkt
\item Zweiter Punkt
\item[3] dritter Punkt
\end{enumerate}
\end{lstlisting}
\begin{enumerate}
\item Erster Punkt
\item Zweiter Punkt
\item[3] dritter Punkt
\end{enumerate}
\end{column}
\end{columns}
\bigskip
\alert{\parbox[t]{.9\textwidth}{Aussehen von |enumerate| und |itemize| wird von der Dokumentenklasse bestimmt!}}
\end{frame}

\section{Schriftgröße}
\begin{frame}[fragile]{Schriftgröße}
\begin{block}{ Als Dokumentoption}
Fast alle Klassen bieten Option für Schriftgröße:
\begin{lstlisting}
\documentclass[10pt]{scrartcl}
\end{lstlisting}
Möglich sind 10pt, 11pt, 12pt
\end{block}\vspace{.4ex}\pause 
Wer wirklich \emph{genau weiß}, was er tut:
\begin{lstlisting}
\fontsize{Größe}{Durchschuss}\selectfont
\fontsize{10}{12}\selectfont
\end{lstlisting}
\end{frame}

\section{Leerzeichen, Striche}
\begin{frame}{Leerzeichen und Striche}
\begin{block}{gute Typographie:}
Es gibt verschieden breite Leerzeichen, je nach Zweck und Position,
auch horizontale Striche unterscheiden sich je nach Bedeutung in Höhe, Breite und Länge.
\end{block}
\end{frame}

\begin{frame}[fragile]{Leerzeichen}
• normales Leerzeichen: \verb*| |
• schmales Leerzeichen (Spatium): |\,|%
• negativer Abstand: |\!|
\\ z. B., z.\,B., z.\!B.%
• großer Abstand (Geviert) |\emspace|
• beliebiger Platz: |\hspace(*){2em}|
• Leerräume im Mathemodus:%
\\ (werden normal automatisch korrekt gesetzt)%
\\ |$a b = a\, b = a \! b$| ⇒ $a b = a\, b = a \! b$%
\pause
• Explizites Ändern des Abstandes (kerning):%
\\ |a\kern-0.1em b| ⇒ a\kern-0.1em b \notiz{Mögliche Maße an der Tafel}%
\\ |a\kern-0.5em b| ⇒ a\kern-0.5em b 
\• 
\end{frame}

\begin{frame}[fragile]{Strichlängen}
• Länge des Striches entscheidend für Grammatik und Lesbarkeit!
• Viertelgeviertstrich, Bindestrich: |-|
• Halbgeviertstrich, Gedankenstrich: |--| –
• Geviertstrich, englischer Gedankenstrich: |---| —
• Minuszeichen: |$-$| $-$
• Im Vergleich:%
\\ |- -- --- $- +$|%
\\ - – — $- +$%
\•\pause
\end{frame}

\begin{frame}[fragile]{Strichlängen}
\begin{block}{Verwendung}
Vorder- und Rückseite \dots\\
Vorderseite – oder Rückseite – \dots\\
frontmatter—or backmatter— \dots\\
Längen/Zeitangaben: Düsseldorf\,–\,Koblenz; 18\,–\,20\,Uhr; mit Spatium!
\end{block}
\end{frame}

\begin{frame}[fragile]{Auslassungen}
Auslassungspunkte sollten gesperrt gesetzt werden:
\\ Wenn ganze Satzteile ...
\\ ausgelassen werden \dots
\\ Wenn Wortteile ausgelassen werden, kommt kein Leer\dots\ davor.
\\[2em]
Befehle |\dots| oder |\ldots| sorgen für richtige Abstände.\\
Paket |ellipsis| korrigiert den Abstand vor und nach Ellipsen.
\end{frame}

\begin{frame}[fragile]{Probleme mit |ellipsis|}
|\usepackage{ellipsis}| führt zu inkonsistenten Abständen nach |\dots|:
\begin{lstlisting}
\LaTeX ist \dots manchmal kompliziert \dots
\end{lstlisting}
\LaTeX ist \dots{}manchmal kompliziert \dots
\\[1em]\pause
|\usepackage[xspace]{ellipsis}| korrigiert dies:
\begin{lstlisting}
\LaTeX\ ist \dots manchmal kompliziert \dots
\end{lstlisting}
\LaTeX\ ist \dots\ manchmal kompliziert \dots
\end{frame}

\section{Dokumentation}
\begin{frame}[fragile]{Welche Befehle möglich?}
• Woher weiß man, welche Befehle |ellipsis| oder |geometry| bieten?
• Welche Einstellungen haben die KOMA-Klassen?
• Wie bekommt man allgemein Informationen über Pakete\,/\,Klassen?
\•
\end{frame}

\begin{frame}[fragile]{texdoc}
• |texdoc| durchsucht die \LaTeX-Ordner
• liefert direkt die Dokumentation des gesuchten Paketes:
• |texdoc amsmath| öffnet |amsmath.pdf|
• |texdoc -l amsmath| listet alle Ergebnisse auf
• |texdoc -s amsmath| liefert Ergebnisse aus erweiterter Suche
• |texdoc --help| bietet weitere Informationen%
\•
⇒ Vorteile der Kommandozeile\\[2em]
|texdoctk| bietet (experimentelle) graphische Oberfläche
\end{frame}

\section{Minimalbeispiel}
\begin{frame}{Minimalbeispiel}
• dient zum Finden von Fehlern
• optimal für Fragen:
• erspart den Kampf durch unnötigen Code
• minimales Dokument, das den Fehler noch produziert
•[⇒] für die meisten Übungsabgaben gefordert
\•
\end{frame}

\begin{frame}[fragile]{Erstellen eines Minimalbeispiels}
• allen Code wegnehmen, der nichts mit dem Problem zu tun hat
• allen unnützen Text wegnehmen. Wenn langer Text nötig ist, Paket |blindtext| verwenden (Übungsaufgabe \dots)
• alle Pakete entfernen, die nichts mit dem Problem zu tun haben.
• falls die Klasse für das Problem unerheblich ist: |\documentclass{minimal}|
• allen Code wegnehmen, der nichts mit dem Problem zu tun hat!
\•
\end{frame}

\section{Fehlermeldungen}
\begin{frame}{Was tun, wenn \LaTeX\ anhält?}
\let•\item
\begin{enumerate}
• Ruhe bewahren (Dateien können nicht beschädigt werden!)
• überlegen, was man gerade geändert hat
• Schreibfehler korrigieren
• log-Datei lesen
• nicht überfliegen, sodern lesen!
\end{enumerate}
\end{frame}

\begin{frame}{Anhalten bei Fehlern}
• manche Editoren kompilieren im nonstop-mode
• Fehler werden ausgegeben, aber ignoriert.
• „Ich habe 100 Fehler in meinem Dokument“ ist keine sinnvolle Aussage:
• sobald ein Fehler auftritt, ist das Dokument meist wertlos.
\•
\end{frame}


\begin{frame}[fragile]{Fehlerausgabe}
Typische Fehlermeldung:
\begin{lstlisting}
! Undefined control sequence.
l.3 Ein \Latex-Dokument
                           .
? 
! Emergency stop.
l.3 Ein \Latex-Dokument.
                           .
No pages of output.
Transcript written on xelatextest.log.
\end{lstlisting}
⇒ Befehl in Zeile 3 falsch geschrieben.
\end{frame}

\begin{frame}[fragile]{Komplizierte Fehlermeldungen}
\begin{lstlisting}
Runaway argument?
{ \coursesection {Einleitung, Übersicht} \input {lk09-einf
! File ended while scanning use of \beamer@howtotreatframe.
<inserted text> 
                \par
\end{lstlisting}
⇒ Irgendwo nach |\coursesection| ein |}| oder |\end{umgebung}| vergessen.\\
Die Zeile wird \emph{nicht} ausgegeben, es sei denn Kommandozeilenargument wird angegeben:\\
|xelatex --file-line-error meinedatei.tex|
\end{frame}

\begin{frame}[fragile]{Zeig’s mir!}
\begin{block}{Definition zeigen mittels show:}
|\show| im Dokument, Ausgabe in der log-Datei:\\
\begin{lstlisting}
\show\wasser

> \wasser=macro:
->H$_2$O.
l.5 \show\wasser
\end{lstlisting}
\end{block}
\end{frame}
\end{document}