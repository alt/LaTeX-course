\makeatletter
\@ifundefined{draftmodeon}{
  \documentclass[german,t]{beamer}
}{
  \documentclass[german,t,draft]{beamer}
}
\makeatother
\mode<presentation>{
  \usetheme{Frankfurt}
  \useoutertheme{infolines}
  \usecolortheme[RGB={0,100,130}]{structure}
  \useinnertheme{rounded}
  \setbeamertemplate{navigation symbols}{} 
}

\newcommand\subtitlei[1]{\def\insertsubtitlei{#1}}
\newcommand\subtitleii[1]{\def\insertsubtitleii{#1}}

\setbeamertemplate{title page}
  {
    \vspace*{1cm}
    \begin{centering}
      {\usebeamerfont{title}\usebeamercolor[bg]{title}
      {\Large \inserttitle}
      \\[.3cm]
      \insertsubtitlei\\
      \large \color{black}\insertsubtitleii
      }
\vspace*{1cm}
    \vbox to 1cm {\hfill \includegraphics[width=.2\textwidth]{ctanlion}}
      \usebeamerfont{subtitle}\usebeamercolor[bg]{subtitle}
      {\color{black}\Large \insertsubtitle}\vfill
    {\hfill \insertdate \hfill}
    \end{centering}

    \vfill\vfill

    \hspace*{-.47cm}
    \usebeamertemplate*{footline}
    \vspace*{-1.15cm}
}

\setbeamercolor*{author in head/foot}{parent=palette tertiary}
\setbeamercolor*{title in head/foot}{parent=palette secondary}
\setbeamercolor*{date in head/foot}{parent=palette primary}

\setbeamercolor*{section in head/foot}{parent=palette tertiary}
\setbeamercolor*{subsection in head/foot}{parent=palette primary}

\defbeamertemplate*{headline}{infolines theme changed}
{
  \leavevmode%
  \hbox{%
  \begin{beamercolorbox}[wd=.3\paperwidth,ht=2.25ex,dp=1.5ex,center]{title in head/foot}%
    \LaTeX-Kurs 2010
  \end{beamercolorbox}%
  \begin{beamercolorbox}[wd=.4\paperwidth,ht=2.25ex,dp=1.2ex,right]{section in head/foot}%
     \hfill \coursenumber\ – \coursetitle \hfill\hfill
  \end{beamercolorbox}%
  \begin{beamercolorbox}[wd=.3\paperwidth,ht=2.25ex,dp=1.2ex,left]{subsection in head/foot}%
    \hspace*{2em}\insertsectionhead
  \end{beamercolorbox}}%
  \vskip0pt%
}

\defbeamertemplate*{footline}{infolines theme changed}
{
  \leavevmode%
  \hbox{%
  \begin{beamercolorbox}[wd=.3\paperwidth,ht=2.25ex,dp=1ex,center]{author in head/foot}%
      \insertshortauthor
  \end{beamercolorbox}%
  \begin{beamercolorbox}[wd=.4\paperwidth,ht=2.25ex,dp=1ex,center]{title in head/foot}%
    \insertshortdate{}
  \end{beamercolorbox}%
  \begin{beamercolorbox}[wd=.3\paperwidth,ht=2.25ex,dp=1ex,center]{author in head/foot}%
    \insertframenumber{} / \inserttotalframenumber
  \end{beamercolorbox}}%
  \vskip0pt%
}

\setbeamersize{text margin left=1em,text margin right=1em}

\def\extractnumber"#1-#2".{#1}
\def\extracttitle"#1-#2".{#2}
\def\coursenumber{\expandafter\extractnumber\jobname.}
\def\coursetitle{\expandafter\extracttitle\jobname.}

\usepackage{
  babel,
  dtklogos,
  moreverb,
  shortvrb,
  xltxtra,
  yfonts
}
\usepackage[final]{showexpl}

\hypersetup{
  colorlinks=true,
  breaklinks=true,
  linkcolor=blue,
  urlcolor=blue,
  citecolor=black, filecolor=black, menucolor=black,
  pdfauthor={Arno L. Trautmann},
}

\MakeShortVerb|

\author{Arno Trautmann}
\institute{Heidelberg}
\title{Einführung in das Textsatzsystem\\[2ex] \Huge \LaTeX}
\subtitlei{Vorlesungsreihe im Sommersemester 2010}
\subtitleii{\textfrak{universitatis:~studii~heydelbergensis:}}
\subtitle{\coursenumber\ – \coursetitle}

\graphicspath{{../Mediales/}}

\logo{}%\includegraphics[width=7.4em]{unilogo.svg}}

\defaultfontfeatures{Scale=MatchLowercase}
\setmonofont{DejaVu Sans Mono}

\newenvironment{mydesc}{\begin{tabular}{|>{\columncolor{lightgray}\color{blue}}rl|}\hline}{\\\hline\end{tabular}}

\def\pkg#1{\texttt{#1}}
\def\macro#1{|#1|}

\def\plainTeX{\textsf{plain\TeX}}

\catcode`\⇒=\active
\def⇒{\ensuremath{\Rightarrow}}
\catcode`\⇐=\active
\def⇐{\ensuremath{\Leftarrow}}
\catcode`\…13
\let…\dots

\newcommand\notiz[1]{}
\newcommand\einschub[1][]{\textcolor{red}{⇒}}
\renewcommand\checkmark{\color{green}\ding{51}}
\newcommand\cross{\color{red}\ding{53}}

\newcommand\pdf[2][]{\bgroup
  \setbeamercolor{background canvas}{bg=}%
  \includepdf[#1]{#2}%
    \egroup
}

\newenvironment{twoblock}[2]{
  \begin{columns}
  \begin{column}{.46\textwidth}
  \begin{block}{#1}
\def\nextblock{
  \end{block}
  \end{column}
\ %
  \begin{column}{.46\textwidth}
  \begin{block}{#2}
}
}
{
  \end{block}
  \end{column}
  \end{columns}
}

\AtBeginDocument{
  \lstset{%
    backgroundcolor=\color[rgb]{.9 .9 .9},
    basicstyle=\ttfamily\small,
    breakindent=0em,
    breaklines=true,
    commentstyle=,
    keywordstyle=,
    identifierstyle=,
    captionpos=b,
    numbers=none,
    frame=tlbr,%shadowbox,
    frameround=tttt,
    pos=r,
    rframe={single},
    explpreset={numbers=none}
  }
}
\makeatletter
\g@addto@macro\beamer@lastminutepatches{ % thanks to Ulrike for this!
  \frame[plain,t]{\titlepage}
  \frame{\centerline{\huge \color[RGB]{0,100,130}Inhalt}\tableofcontents}
}

%—— itemize-hack
\def\outside{o}
\def\inside{i}
\let\insideitemizei\outside
\let\insideitemizeii\outside
\def\altenditemize{
  \if\altlastitem 1%
    \let\altlastitem0%
  \else%
    \end{itemize}%
    \let\insideitemizei\outside%
  \fi%
}

\begingroup
  \lccode`\~=`\^^M%
\lowercase{%
  \endgroup
  \def\makeenteractive{%
    \catcode`\^^M=\active
    \let~\altenditemize
}%
}

\def\newitemi{%
  \ifx\insideitemizei\inside%
    \let\altlastitem1%
    \expandafter\item%
  \else%
    \begin{itemize}%
    \let\insideitemizei\inside%
    \let\altlastitem1%
    \makeenteractive%
    \expandafter\item%
  \fi
}

\def\newitemii{
  \ifx\insideitemizeii\inside
    \expandafter\item%
  \else
    \begin{itemize}
      \let\insideitemizeii\inside
      \expandafter\item%
  \fi
}

\def\makeitemi#1{%
  \expandafter\ifx\csname cc\string#1\endcsname\relax
    \add@special{#1}%
    \expandafter
    \xdef\csname cc\string#1\endcsname{\the\catcode`#1}%
    \begingroup
      \catcode`\~\active  \lccode`\~`#1%
      \lowercase{%
      \global\expandafter\let
         \csname ac\string#1\endcsname~%
      \expandafter\gdef\expandafter~\expandafter{\newitemi}}%
    \endgroup
    \global\catcode`#1\active
  \else
  \fi
}

\def\makeitemii#1{%
  \expandafter\ifx\csname cc\string#1\endcsname\relax
    \add@special{#1}%
    \expandafter
    \xdef\csname cc\string#1\endcsname{\the\catcode`#1}%
    \begingroup
      \catcode`\~\active  \lccode`\~`#1%
      \lowercase{%
      \global\expandafter\let
         \csname ac\string#1\endcsname~%
      \expandafter\gdef\expandafter~\expandafter{\newitemii}}%
    \endgroup
    \global\catcode`#1\active
  \else
  \fi
}

\def\add@special#1{%
  \rem@special{#1}%
  \expandafter\gdef\expandafter\dospecials\expandafter
{\dospecials \do #1}%
  \expandafter\gdef\expandafter\@sanitize\expandafter
{\@sanitize \@makeother #1}}
\def\rem@special#1{%
  \def\do##1{%
    \ifnum`#1=`##1 \else \noexpand\do\noexpand##1\fi}%
  \xdef\dospecials{\dospecials}%
  \begingroup
    \def\@makeother##1{%
      \ifnum`#1=`##1 \else \noexpand\@makeother\noexpand##1\fi}%
    \xdef\@sanitize{\@sanitize}%
  \endgroup}
\AtBeginDocument{
  \makeitemi{•}
}
%——beamer versaut hier irgendwas, daher muss itemize explizit beendet werden!
\def\•{\end{itemize}}
%—— itemize-hack
\makeatother
\usepackage[normalem]{ulem}

\begin{document}
\section{Inline vs. Display}
\begin{frame}{Inlinemode}
• Formeln, die direkt im Fließtext vorkommen
• kurze Formeln
• Brüche und Wurzeln vermeiden
• Grenzen werden \emph{neben} Integrale, Summen, Produkte gesetzt
\•
\end{frame}

\begin{frame}[fragile]{Inlinemode}
\rmfamily
$E=mc^2$ kennt jedes Kind, aber kaum jemand kann wirklich mehr damit anfangen als mit $\int^\infty_{-\infty}\sum_{n = 1}^5 dx$, wobei diese Formel nun mal gar keinen Sinn ergibt, aber zeigt, wie Grenzen im Mathesatz aussehen.\\
$E=mc^2$ kennt jedes Kind, aber kaum jemand kann wirklich mehr damit anfangen als mit $\displaystyle \int^\infty_{-\infty}\sum_{n = 1}^5 dx$, wobei diese Formel nun mal gar keinen Sinn ergibt, aber zeigt, wie Grenzen im Mathesatz aussehen.\\
$E=mc^2$ kennt jedes Kind, aber kaum jemand kann wirklich mehr damit anfangen als mit \[\int^\infty_{-\infty}\sum_{n = 1}^5 dx,\] wobei diese zweite Formel nun mal gar keinen Sinn ergibt, aber zeigt, wie Grenzen im \TeX-Mathesatz aussehen.\\
\end{frame}

\begin{frame}[fragile]{Inlinemode}
Der Inlinemode ist über drei Wege zu erreichen:
• |\(Formel\)| \only<3->{\alert{nicht robust!}}
• |\begin{math} formel \end{math}| \only<3->{\alert{nicht robust!}}\\ \only<2->{\alert{funktioniert nicht in |alltt|}}
• |$Formel$| \only<2->{\\ \alert{funktioniert nicht in |alltt|}}
\•
\end{frame}

\begin{frame}[fragile]{robust vs. fragile}
• fragile Befehle können Probleme bereiten bei:
• Schreiben in Datei, Ausgabe in log\,/\,Terminal, in Verzeichnissen
• robuste Befehle sind speziell geschützt
•[⇒] |$ $| ist meist beste Variante
\•
\end{frame}

\begin{frame}[fragile]{Umbruch}
• Formeln können von \TeX\ umbrochen werden an:
• Relationen |= < >| etc.
• binären Operatoren |+ -| etc.
• zum Vermeiden: Gruppieren
• |beamer|-Klasse kann Probleme bereiten
\•\pause
\begin{LTXexample}
Etwas Text bis zum Zeilenende
$a + b + c$\\
Etwas Text bis zum Zeilenende
${a + b + c}$\\
Eine viel zu lange Formel:
${a+b+c+d+e+f+g+h+i+j+k+l+m}$
\end{LTXexample}
\end{frame}

\begin{frame}{Display-Formeln}
• Auszeichnung wichtiger Formeln
• Darstellung langer Rechnungen
• komplexe Formeln
• mehrfach indizierte Größen
• geschachtelte Brüche
• …
\•
\end{frame}

\begin{frame}[fragile]{ Display-Formeln}
Display-Formeln sind über drei Wege zu erreichen:
• |\begin{displaymath} Formel \end{displaymath}|\\%
– abgesetzte Formel ohne Nummerierung
• |\[ Formel \]|\\%
– Abkürzung für |displaymath|
• |\begin{equation} Formel \end{equation}|%
\\ abgesetzte Formel mit Nummerierung
• |$$ Formel $$|
\•
\end{frame}

\begin{frame}[fragile]{ Display-Formeln}
Display-Formeln sind über drei Wege zu erreichen:
• |\begin{displaymath} Formel \end{displaymath}|\\%
– abgesetzte Formel ohne Nummerierung
• |\[ Formel \]|\\%
– Abkürzung für |displaymath|
• |\begin{equation} Formel \end{equation}|%
\\ abgesetzte Formel mit Nummerierung
• \catcode`\$=12 \xout{$$ Formel $$} \alert{\TeX-Syntax!}%
\\ \alert{in \LaTeX\ führt Verwenden von |$$ $$| zu unerwarteten und unerwünschten Ergebnissen; unbedingt vermeiden!}
\•
\end{frame}

\begin{frame}[fragile]{Option fleqn}
• oft sehen Formeln zentriert nicht gut aus
• „zerfledderter“ Eindruck
• linksbündige Ausrichtung oft besser
•[⇒] |fleqn| als Dokumentoption (|\documentclass[fleqn]{scrartcl}|)
•[\alert{!}] funktioniert \emph{nicht} mit |$$ $$|
\•
\end{frame}

\begin{frame}[fragile]{Display in Inline und umgekehrt}
Große Formeln im Inline-Modus:
• |\displaystyle|
\• 
\begin{LTXexample}[pos=b]
Eine Zeile Text vor dem großen Bruch, damit deutlich sichtbar wird, warum man so große Brüche: $\frac{a}{b} < \displaystyle{\frac{a}{b}}$ nicht im Fließtext setzt. Und danach und noch ein bisschen Text für die zweite Zeile, die deutlich mehr von der ersten getrennt ist als die dritte von der zweiten.
\end{LTXexample}
\end{frame}

\begin{frame}[fragile]{Display in Inline und umgekehrt}
Eher nützlich: Kleine Formeln im abgesetzten Modus:
• |\textstyle|
\•
\begin{LTXexample}[pos=b,rframe={}]
\[\frac 12 a > \frac 12 b > \frac 12 c \ vs. \
{ \textstyle \frac 12 a  > \frac 12 b > \frac 12 c}\]
\end{LTXexample}
\pause
Nützliche Definition z.\,B.
\begin{lstlisting}
\newcommand\half{\textstyle{\frac 1 2}}
\end{lstlisting}
• \AMS{}math bietet |\tfrac| und |\dfrac|
\•
\end{frame}

\begin{frame}{Mehrzeilige Formeln}
• Reihe von untereinander angeordneten, zueinander ausgerichteten Gleichungen, verwendet für:
• Herleitungen
• Übersicht
• Vergleich von Formeln
\• 
\end{frame}

\begin{frame}[fragile]{eqnarray}
• |eqnarray|: Standardumgebung für mehrzeilige Formeln
• Artikel in der DTK: „Vermeidet |eqnarray|“
•[⇒] besser: |align| aus dem |amsmath|-Paket:
\•
\begin{LTXexample}
\begin{align}
a &= b,\\
c &= d,\\
abc &= d\\
&= r
\end{align}
\end{LTXexample}
|{align*}|: keine Nummerierung
\end{frame}

\begin{frame}[fragile]{\AMS{}math}
• Paket von der American Mathematical Society (\AMS)
• besteht aus mehreren Paketen, u.\,a.:\\%
|amsmath|, |amssymb|, |amsfonts|%
• bietet umfangreiche Erweiterungen des Mathesatzes:
• vielfältige Umgebungen und Anpassungen
• neue oder verbesserte Definitionen von Befehlen
• Korrekturen von Abständen
• …
\•
\end{frame}

\begin{frame}[fragile]{Abstände}
• \TeX\ bzw. \LaTeX\ bzw. geladene Pakete kontrollieren Abstände
• Unterschiede zwischen Variablen, Operatoren, Relationen etc.
• Festgelegt durch die |\mathcode|s der Zeichen
• Änderbar mit |\kern|, |\|, |\,| etc.
• \alert{niemals} Konstrukte wie |\ \ \ \ | verwenden!
• Besser: |\quad|, |\qquad|, |\hspace{1em}|
\•
\end{frame}

\section{Größe von Formeln}
\begin{frame}[fragile]{Größenänderungen}
• Standardbefehle wie |\small, \tiny, \Huge| haben in Formeln keine Wirkung
• Aber Formeln passen sich der Umgebung an:
\• 
\begin{LTXexample}[pos=b]
\small\[E = \Huge mc^2\]
\Huge\[E = mc^2\]
\end{LTXexample}
\end{frame}

\section{Grundbefehle}
\begin{frame}[fragile]{Variablen und Zahlen}
• Variablen werden kursiv gesetzt: |$a$|: $a$
• Schriftart abhängig von der Dokumentenklasse!\\%
(Groteske, Serifen etc.)
• Ziffern werden automatisch korrekt gesetzt: $12.2$ statt 12.2
\•
\end{frame}

\begin{frame}[fragile]{Punkt vs. Komma}
im amerikanischen Satz:
\begin{LTXexample}[preset=\huge]
$1,234.567$
\end{LTXexample}\pause
im deutschen Satz:
\begin{LTXexample}[preset=\huge]
$1.234,567$
\end{LTXexample}
\alert{⇒ falsche Spationierung!}
\end{frame}
\begin{frame}[fragile]{Punkt vs. Komma}
\begin{block}{Einmalige Anpassung:}
\Large|$1{,}2\mathpunct{.}3$|\\
$1,2.3$ \normalsize (nicht angepasst)\\
\Large $1{,}2\mathord{.}3$ \normalsize(angepasst)
\end{block}
\pause
\begin{block}{Korrektur des Dezimaltrennzeichens}
|\DeclareMathSymbol{,}{\mathpunct}{letters}{"3B}|\\
|\DeclareMathSymbol{.}{\mathord}{letters}{"3A}|
\end{block}
\pause
\begin{block}{Automatische Anpassung}
Paket |icomma| passt Dezimaltrennzeichen automatisch dokumentenweit an.
\end{block}
\end{frame}

\begin{frame}[fragile]{Hoch- und Tiefstellung}
• Zeichen mit besonderer Bedeutung: |^| und |_|
• Hochstellung: |a^b|, Tiefstellung: |a_b|: $a_b$
• Gruppierungen sind möglich: |a^{bc}|, |a_{bc}|: $a_{bc}$
• Kombination ist möglich: |a^b_c|: $a^b_c$
• Ohne vorhergehendes Zeichen: |^{235}U|: $^{235}\mathrm U$
• Schachtelung nur mit Gruppierung:\\%
|a_{b_{c_{d_{e_{f^g}}}}}^{h^{i^{j_k}}}| \Large $a_{b_{c_{d_{e_{f^g}}}}}^{h^{i^{j_k}}}$\normalsize
•[] |a_b_c| produziert Fehler!
\•
\end{frame}

\begin{frame}[fragile]{Operatornamen}
• Operatornamen werden aufrecht gesetzt und sind vordefiniert:
• $\sin(x)$ vs. $sin(x)$
• |\sin \cos \tan \lim \atan \arctan|\\%
$\sin \cos \tan \lim \arctan$
• Paket |amsopn| bietet weitere Vordefinitionen:
\•
\begin{lstlisting}
\arccos \arcsin \arg \cos \cot \coth \deg \det
\exp \gcd \inf \injlim \lg \lim \limsup \ln
\max \min \projlim \sec \sinh \sup \tanh
\end{lstlisting}
\end{frame}

\begin{frame}[fragile]{Definieren von Operationen}
Sollten die vorgegebenen Definitionen nicht genügen:
\begin{lstlisting}
\usepackage{amsopn}
\DeclareMathOperator{\Res}{Res}
\end{lstlisting}
in der Präambel.
\end{frame}

\begin{frame}[fragile]{links und rechts}
• Klammerung von großen Ausdrücken kann Probleme bereiten:
\•
\begin{LTXexample}
\[(\frac{\int^a x dx}{\sum_{n=1} x})\]
\end{LTXexample}
• Besser:
\•
\begin{LTXexample}
\[ \left(
\frac{\int^a x dx}{\sum_{n=1} x}
\right) \]
\end{LTXexample}
\end{frame}

\begin{frame}[fragile]{links und rechts}
• |\left| und |\right| vor allem, was dehnbar ist
• |\left( \right]| funktioniert auch
• |\left. \right)| liefert angepasste rechte Klammer
• Hoch- und Tiefstellung werden angepasst:
\•
\begin{LTXexample}[preset=\DeleteShortVerb|]
\[\left[\int^a\right\}\]
\[\left.\int^a\right.\]
\[\left.\int^a dx\right|^5_1\]
\end{LTXexample}
\MakeShortVerb|
\end{frame}

\begin{frame}[fragile]{Operatoren}
• Operatoren sind intuitiv per Namen zugänglich
• Grenzen per |\limits| angeben
• Mehrzeilige Grenzen mit |\atop|
\•
\begin{LTXexample}
\[\int^X \int\limits^X
\sum_{n=1}^\infty
\prod_{n = 1 \atop m = 2}\]
\end{LTXexample}
\end{frame}

\begin{frame}[fragile]{Sonderzeichen}
• Viele Zeichen sind über ihren Namen ereichbar,
• Griechische Groß- und Kleinbuchstaben ebenso
\•
\begin{LTXexample}
\[\nabla \pm \mp
\alpha \beta \gamma
\rho \varrho \kappa \varkappa
\epsilon \varepsilon \theta \vartheta
A B \Gamma\]
\end{LTXexample}
\end{frame}

\begin{frame}[fragile]{Wurzeln}
• Wurzel:
• zu tiefe Unterlängen sind unschön ⇒ |\smash|
\• 
\begin{LTXexample}[preset=\Large]
\[
\sqrt[3]{a_{n_{m_p}}}
\quad\sqrt{a}\quad
\sqrt{\smash[b]{a_{n_{m_p}}}}
\]
\end{LTXexample}
\end{frame}

\section{Vektoren, Matrizen, Tensoren}
\begin{frame}[fragile]{Vektoren}
Vektoren sind vielfältig darstellbar:
• Mit Pfeil drüber als |\vec|
\•

\begin{LTXexample}
$\vec a\ \pmb a\ \mathbf a$
$\underbar a$
\end{LTXexample}
\end{frame}

\begin{frame}[fragile]{Matrizen}
\begin{LTXexample}
\[\begin{matrix}
a_{11} & a_{12}\\
a_{21} & a_{22}
\end{matrix}\]
\end{LTXexample}
\end{frame}

\begin{frame}[fragile]{Matrizen}
\begin{LTXexample}
\[\left(\begin{matrix}
a_{11} & a_{12}\\
a_{21} & a_{22}
\end{matrix}\right)\]
\end{LTXexample}
\end{frame}

\begin{frame}[fragile]{Matrizen}
\AMS{}math definiert weitere Matrixumgebungen:\\[2em]
\begin{minipage}{3cm}
\[\begin{pmatrix}
a & b \\ c & d
\end{pmatrix}\]
\centering pmatrix
\end{minipage}
\begin{minipage}{3cm}
\[\begin{Vmatrix}
a & b \\ c & d
\end{Vmatrix}\]
\centering Vmatrix
\end{minipage}
\begin{minipage}{3cm}
\[\begin{vmatrix}
a & b \\ c & d
\end{vmatrix}\]
\centering vmatrix
\end{minipage}
\\[2em]
\begin{minipage}{3cm}
\[\begin{Bmatrix}
a & b \\ c & d
\end{Bmatrix}\]
\centering Bmatrix
\end{minipage}
\begin{minipage}{3cm}
\[\begin{bmatrix}
a & b \\ c & d
\end{bmatrix}\]
\centering bmatrix
\end{minipage}
\begin{minipage}{3cm}
\[\begin{smallmatrix}
a & b \\ c & d
\end{smallmatrix}\]
\centering smallmatrix
\end{minipage}
\end{frame}

\begin{frame}{texdoc}
• Dokumentation:
• |texdoc amsmath, amsop etc.|
• |texdoc mathmode|: Öffnet |Math Mode| von Herbert Voss:
•[] umfassende Dokumentation verschiendenster Thematiken zum Mathesatz
• auch als Buch verfügbar (reduzierter Preis für DANTE-Mitglieder!)
\• 
\end{frame}

\end{document}