\let\oldbar|
\let\exclam! % what fucks up the bang?
\makeatletter
\@ifundefined{draftmodeon}{
  \documentclass[german,t]{beamer}
}{
  \documentclass[german,t,draft]{beamer}
}
\makeatother
\mode<presentation>{
  \usetheme{Frankfurt}
  \useoutertheme{infolines}
  \usecolortheme[RGB={0,100,130}]{structure}
  \useinnertheme{rounded}
  \setbeamertemplate{navigation symbols}{} 
}

\newcommand\subtitlei[1]{\def\insertsubtitlei{#1}}
\newcommand\subtitleii[1]{\def\insertsubtitleii{#1}}

\setbeamertemplate{title page}
  {
    \vspace*{1cm}
    \begin{centering}
      {\usebeamerfont{title}\usebeamercolor[bg]{title}
      {\Large \inserttitle}
      \\[.3cm]
      \insertsubtitlei\\
      \large \color{black}\insertsubtitleii
      }
\vspace*{1cm}
    \vbox to 1cm {\hfill \includegraphics[width=.2\textwidth]{ctanlion}}
      \usebeamerfont{subtitle}\usebeamercolor[bg]{subtitle}
      {\color{black}\Large \insertsubtitle}\vfill

    {\hfill \insertdate \hfill}
    \end{centering}

    \vfill\vfill

    \hspace*{-.47cm}
    \usebeamertemplate*{footline}
    \vspace*{-1.15cm}
}

\setbeamercolor*{author in head/foot}{parent=palette tertiary}
\setbeamercolor*{title in head/foot}{parent=palette secondary}
\setbeamercolor*{date in head/foot}{parent=palette primary}

\setbeamercolor*{section in head/foot}{parent=palette tertiary}
\setbeamercolor*{subsection in head/foot}{parent=palette primary}

\defbeamertemplate*{footline}{infolines theme changed}
{
  \leavevmode%
  \hbox{%
  \begin{beamercolorbox}[wd=.3\paperwidth,ht=2.25ex,dp=1ex,center]{author in head/foot}%
      \insertshortauthor
  \end{beamercolorbox}%
  \begin{beamercolorbox}[wd=.4\paperwidth,ht=2.25ex,dp=1ex,center]{title in head/foot}%
    \insertshortdate{}
  \end{beamercolorbox}%
  \begin{beamercolorbox}[wd=.3\paperwidth,ht=2.25ex,dp=1ex,center]{author in head/foot}%
    \insertframenumber{} / \inserttotalframenumber
  \end{beamercolorbox}}%
  \vskip0pt%
}

\defbeamertemplate*{headline}{infolines theme changed}
{
  \leavevmode%
  \hbox{%
  \begin{beamercolorbox}[wd=.3\paperwidth,ht=2.25ex,dp=1.5ex,center]{title in head/foot}%
    \LaTeX-Kurs 2010
  \end{beamercolorbox}%
  \begin{beamercolorbox}[wd=.4\paperwidth,ht=2.25ex,dp=1.2ex,right]{section in head/foot}%
      \coursenumber\ – \coursetitle \hspace*{2em}
  \end{beamercolorbox}%
  \begin{beamercolorbox}[wd=.3\paperwidth,ht=2.25ex,dp=1.2ex,left]{subsection in head/foot}%
    \hspace*{2em}\insertsectionhead
  \end{beamercolorbox}}%
  \vskip0pt%
}

\setbeamersize{text margin left=1em,text margin right=1em}

\def\extractnumber"#1-#2".{#1}
\def\extracttitle"#1-#2".{#2}
\def\coursenumber{\expandafter\extractnumber\jobname.}
\def\coursetitle{\expandafter\extracttitle\jobname.}

\usepackage{
  babel,
  dtklogos,
  moreverb,
  shortvrb,
  xltxtra,
  yfonts
}
\usepackage[final]{showexpl}

\hypersetup{
  colorlinks=true,
  breaklinks=true,
  linkcolor=black, citecolor=black, filecolor=black, menucolor=black,
  urlcolor=blue,
  pdfauthor={Arno L. Trautmann},
}

\MakeShortVerb|

\author{Arno Trautmann}
\institute{Heidelberg}
\title{Einführung in das Textsatzsystem\\[2ex] \Huge \LaTeX}
\subtitlei{Vorlesungsreihe im Sommersemester 2010}
\subtitleii{\textfrak{universitatis:~studii~heydelbergensis:}}
\subtitle{\coursenumber\ – \coursetitle}

\logo{}%\includegraphics[width=7.4em]{unilogo.svg}}

\defaultfontfeatures{Scale=MatchLowercase}
\setmonofont{DejaVu Sans Mono}

\def\lcode{\vspace{.5ex}\par\boxedverbatim}
\def\endlcode{\endboxedverbatim\bigskip\\}
\newenvironment{mydesc}{\begin{tabular}{|>{\columncolor{lightgray}\color{blue}}rl|}\hline}{\\\hline\end{tabular}}

\def\pkg#1{\texttt{#1}}
\def\macro#1{|#1|}

\def\plainTeX{\textsf{plain\TeX}}

\catcode`\⇒=\active
\def⇒{\ensuremath{\Rightarrow}}
\catcode`\⇐=\active
\def⇐{\ensuremath{\Leftarrow}}
\catcode`\…13
\let…\dots


\newcommand\notiz[1]{}
\renewcommand\checkmark{\color{green}\ding{51}}
\newcommand\cross{\color{red}\ding{53}}

\newcommand\pdf[2][]{\bgroup
  \setbeamercolor{background canvas}{bg=}%
  \includepdf[#1]{#2}%
    \egroup
}

\AtBeginDocument{
  \lstset{%
    backgroundcolor=\color[rgb]{.9 .9 .9},
    basicstyle=\ttfamily\small,
    breakindent=0em,
    breaklines=true,
    commentstyle=,
    keywordstyle=,
    identifierstyle=,
    captionpos=b,
    numbers=none,
    frame=tlbr,%shadowbox,
    frameround=tttt,
    pos=r,
    rframe={single},
    explpreset={numbers=none}
  }
}
\makeatletter
\g@addto@macro\beamer@lastminutepatches{ % thanks to Ulrike for this!
  \frame[plain,t]{\titlepage}
  \frame{\centerline{\huge \color[RGB]{0,100,130}Inhalt}\tableofcontents}
}

%—— itemize-hack
\def\outside{o}
\def\inside{i}
\let\insideitemizei\outside
\let\insideitemizeii\outside
\def\altenditemize{
  \if\altlastitem 1%
    \let\altlastitem0%
  \else%
    \end{itemize}%
    \let\insideitemizei\outside%
  \fi%
}

\begingroup
  \lccode`\~=`\^^M%
\lowercase{%
  \endgroup
  \def\makeenteractive{%
    \catcode`\^^M=\active
    \let~\altenditemize
}%
}

\def\newitemi{%
  \ifx\insideitemizei\inside%
    \let\altlastitem1%
    \expandafter\item%
  \else%
    \begin{itemize}%
    \let\insideitemizei\inside%
    \let\altlastitem1%
    \makeenteractive%
    \expandafter\item%
  \fi
}

\def\newitemii{
  \ifx\insideitemizeii\inside
    \expandafter\item%
  \else
    \begin{itemize}
      \let\insideitemizeii\inside
      \expandafter\item%
  \fi
}

\def\makeitemi#1{%
  \expandafter\ifx\csname cc\string#1\endcsname\relax
    \add@special{#1}%
    \expandafter
    \xdef\csname cc\string#1\endcsname{\the\catcode`#1}%
    \begingroup
      \catcode`\~\active  \lccode`\~`#1%
      \lowercase{%
      \global\expandafter\let
         \csname ac\string#1\endcsname~%
      \expandafter\gdef\expandafter~\expandafter{\newitemi}}%
    \endgroup
    \global\catcode`#1\active
  \else
  \fi
}

\def\makeitemii#1{%
  \expandafter\ifx\csname cc\string#1\endcsname\relax
    \add@special{#1}%
    \expandafter
    \xdef\csname cc\string#1\endcsname{\the\catcode`#1}%
    \begingroup
      \catcode`\~\active  \lccode`\~`#1%
      \lowercase{%
      \global\expandafter\let
         \csname ac\string#1\endcsname~%
      \expandafter\gdef\expandafter~\expandafter{\newitemii}}%
    \endgroup
    \global\catcode`#1\active
  \else
  \fi
}

\def\add@special#1{%
  \rem@special{#1}%
  \expandafter\gdef\expandafter\dospecials\expandafter
{\dospecials \do #1}%
  \expandafter\gdef\expandafter\@sanitize\expandafter
{\@sanitize \@makeother #1}}
\def\rem@special#1{%
  \def\do##1{%
    \ifnum`#1=`##1 \else \noexpand\do\noexpand##1\fi}%
  \xdef\dospecials{\dospecials}%
  \begingroup
    \def\@makeother##1{%
      \ifnum`#1=`##1 \else \noexpand\@makeother\noexpand##1\fi}%
    \xdef\@sanitize{\@sanitize}%
  \endgroup}
\AtBeginDocument{
  \makeitemi{•}
}
%——beamer versaut hier irgendwas…
\def\•{\end{itemize}}
%—— itemize-hack
\makeatother

\newcommand\messdaten{
\toprule
\bf Pendellänge $l$ [\si{m}]& \bf Dauer $T$ [\si{s}]\\\midrule
4 & 8 \\
2 & 4 \\
1   & 2 \\
.9  & 1.8 \\
0.8 & 1.6 \\
0.7 & 1.4 \\
0.6 & 1.2 \\
0.5 & 1.0 \\
0.4 & 0.8 \\
0.3 & 0.6 \\
0.2 & 0.4 \\
0.1 & 0.2 \\
0.05 & 0.1 \\
0.02 & 0.05 \\
0.01 & 0.02 \\
0.005 & 0.01 \\
0.0025 & 0.005\\
\bottomrule
}
\let\eV\relax
\usepackage{
  12many,
  braket,
  cases,
  esvect,
  feyn,
  mathtools,
  gnuplottex,
  pdfpages,
  relsize,
  siunitx,
  soul
}
\usepackage[normalem]{ulem}


\begin{document}
\section{Integral-/Differentialrechnung}
\begin{frame}[fragile]{Integrale}
\AMS{}math bietet weitere Integrale:
\begin{LTXexample} 
\[\iint \iiint \iiiint \oint \idotsint\]
\[\int \int\]
\end{LTXexample}
\end{frame}

\begin{frame}[fragile]{Integrale}
\begin{columns}
\begin{column}{.4\textwidth}
Zusätzliche Integraldarstellungen bieten:
• |wasysym|
• |txfonts|
• |esint|
• |MnSymbol|
• |mathdesign|
\•
\end{column}
\begin{column}{.5\textwidth}
\alert{Auf Kompatibilität achten}\\
Verschiedene Matheschriften zusammen können Probleme bereiten.
\end{column}
\end{columns}
\end{frame}

\section{Symbole}
\begin{frame}[fragile]{Relationen}
\begin{LTXexample}
$= \equiv \approx \asymp \bowtie \cong \dashv \doteq \sim \simeq \propto \smile$
\end{LTXexample}
\pause Negierung mit |\not|
\begin{LTXexample} 
$\not = \neq \not\equiv
\not \approx \not A
\not\kern-.2em\int \not\kern-.2em\partial \not \smile$
\end{LTXexample}
\pause Stapeln von Symbolen
\begin{LTXexample}
$\stackrel{oben}{unten}$
$\stackrel{\text e}{\text a} = $ ä
$\stackrel . = \neq \doteq$
\end{LTXexample}
\hfill$\stackrel \exclam=$
\end{frame}

\DeleteShortVerb|
\begin{frame}[fragile]{bra ket}
• abstrakte Darstellung von Zuständen in der Quantenmechanik
• Unabhängigkeit von Koordinaten
• Ket: \verb|\ket a|, Bra: \verb|\bra a|
• Skalarprodukt: Bra(c)ket: \verb|\braket{a l b}|
• Matrixelement: \verb|\braket{a l A l b}|
\•
\end{frame}

\begin{frame}[fragile]{Satz von bra und ket}
erster Ansatz:
\begin{LTXexample}
$|a> <a|A|a>$
\end{LTXexample}
\pause zweiter\,/\,dritter Ansatz:
\begin{LTXexample} 
$\left|a\right>
\left<a|\frac A B \middle|a\right>$
\end{LTXexample}
\pause Guter Ansatz: Paket \verb|braket|
\begin{verbatim}
\bra a \ket b
\braket{a|\frac A B|a}
\Braket{a|\frac A B|a}
\end{verbatim}
\end{frame}

\begin{frame}[fragile]{Akzente}
Für Operatoren benötigt man das „Dach“:
\begin{LTXexample}[width=.4\textwidth]
$\hat{\mathrm{A}} \bar h \check a \dot a\\
\ddot a \dddot a \ddddot a\\
\underbrace{E = mc^2}_\text{nach Einstein}\overbrace{\int_\infty}^{\text{Hinweis}}$
\end{LTXexample}
\end{frame}

\begin{frame}[fragile]{Pfeile}
Für Spinzustände oft verwendete Notation mittels Pfeilen:
\begin{LTXexample}[width=.4\textwidth]
$\uparrow \downarrow \Uparrow \Downarrow
\Rightarrow \leftrightarrow\\
\longrightarrow \mapsto \to \rightarrow
\leftharpoondown \rightharpoonup \rightleftharpoons
\Rsh$
\end{LTXexample}
\end{frame}

\MakeShortVerb|
\begin{frame}[fragile]{mehr Pfeile}
Über- und Unterschreibungen von Pfeilen\\ (Beschriftung von Reaktionsgleichungen etc.)
\begin{LTXexample}
$\xleftarrow[unten]{oben}
 \xrightarrow[unten]{}$
\end{LTXexample}
\begin{LTXexample}
$\overleftarrow a
\overleftrightarrow b
\stackrel\leftrightarrow T$
\end{LTXexample}
\end{frame}

\section{Schriften}
\begin{frame}[fragile]{Matheschriften}
• Matheschrift muss am Anfang des Dokumentes festgesetzt werden
• Kann nicht im Dokument geändert werden
• Pakete freier Schriften
• |mathpazo|
• |cmbright|
• |mathpazo|
• |eulervm|
\•
Eine Reihe nichtfreier Schriften ist in speziellen Paketen verfügbar.
\end{frame}
\pdf[fitpaper=false]{euler}

\begin{frame}[fragile]{Matheschriften}
Hervorhebungen/besondere Buchstaben:
• Kalligraphische Buchstaben |\mathcal|
• Serifenlose
• Fraktur ($\Re, \Im$)
• Aufrechte Buchstaben
• Fettdruck (für Griechisch: Paket |\bm|)
• „blackboard bold“: $\mathbb R$
\• 
\end{frame}

\begin{frame}[fragile]{Matheschriften}
• Paket |unicode-math| (Will Robertson) bietet experimentellen Zugriff auf otf-Matheschriften
• freie Matheschriften selten
• Unterstützung noch sehr rudimentär
• zukünftige Entwicklung vielversprechend
• in \LaTeX3 evtl. stabil verfügbar \dots
• geplant für lua\TeX
\•
\end{frame}

\section{SI-Einheiten}
\begin{frame}[fragile]{Setzen von Einheiten}
Paket |siunitx| (Joseph Wright)
\begin{LTXexample}[preset={\obeylines},pos=r]
\SI[seperr]{23.448(5)e23}{g.cm^3}
\si[per=frac]{\joule\per\eV}
\si{\joule\per\eV}
\num[dp=2]{4.4583 x 3.2 e21}
\num[mode=text]{4.58}
\num[expproduct=cdot]{1e10}
\ang[]{45}
\end{LTXexample}
\end{frame}

\begin{frame}[fragile]{}
Ändern der Voreinstellungen mittels |\sisetup|
\begin{LTXexample}
\sisetup{colourneg}
$\num{-3}, \num{3},
\num[negcolour=blue]{-5x5},
\num{2}\cdot\num 2$\\

\def\a{5.1}
$\SI{\a x 5.3}{\milli\metre}$\\
$\num{\a x 5.3}\si{\Square\milli
\metre}$\\
$\num{\a x 5.3}\si{\milli\metre
\squared}$
\end{LTXexample}
\end{frame}

\begin{frame}[fragile]{Gradangaben}
\begin{LTXexample}
\ang{10}
\ang{12.3}
\ang{4,5}
\\ Heidelberg:
\ang{49;25;}N, \ang{8;43;}O, \ang{49;25;}N, \ang{8;43;}O
\end{LTXexample}
\end{frame}

\begin{frame}[fragile]{Einheiten}
\begin{LTXexample}[preset=\large]
\SI{5.54}{ms^{-2}}\\
\SI{5.54}{m s^{-2}}\\
\SI{5.54}{m.s^{-2}}\\
\SI[valuesep=thick]{5.54}{m.s^{-2}}\\
\SI[valuesep=thin]{5.54}{m.s^{-2}}\\
\end{LTXexample}
\end{frame}

\begin{frame}[fragile]{Einheiten}
\begin{LTXexample}[width=.4\textwidth]
\sisetup{per=fraction}
\SI{1.23}{\joule\per\mole\per\kelvin}
\\ \sisetup{per=slash}
\SI{1.23}{\joule\per\mole\per\kelvin}
\\ \sisetup{per=fraction,fraction=nice}
\SI{1.23}{\joule\per\mole\per\kelvin}
\end{LTXexample}
\end{frame}

\section{Matrizen}
\begin{frame}[fragile]{Satz komplexer Matrizen}
\begin{LTXexample}
\[\begin{pmatrix}
a & b & \dots & z\\
b & \dots & \dots & z\\
\vdots & \ddots& \reflectbox{$\ddots$} & \vdots\\
\hdotsfor{4}\\
z & b & \dots & \begin{pmatrix}
a & b \\ c & d\end{pmatrix}
\end{pmatrix}\]
\end{LTXexample}
\end{frame}

\section{Platz}
\begin{frame}[fragile]{Änderung der Platzverteilung}
• Kerning
• v/hspace: |\hspace{1cm},\hspace*{1cm}|
• Achtung bei |\vspace|: Nur im vertikalen Modus möglich
• Phantome
\•
\end{frame}

\begin{frame}[fragile]{Phantome}
\begin{LTXexample}[width=.4\textwidth]
$a_x = b$\\
$\hphantom{a_x} = b$\\
$\underline{a_x} = \underline{b\vphantom{a_x}} c \underline{a_x} \underline b$
\end{LTXexample}
\begin{LTXexample}
\begin{align*}
a &= b\\
c &= d\\
\int a &= b
\end{align*}
\end{LTXexample}

\end{frame}

\begin{frame}[fragile]{Phantome}
\begin{LTXexample}[width=.4\textwidth]
$a_x = b$\\
$\hphantom{a_x} = b$\\
$\underline{a_x} = \underline{b\vphantom{a_x}}\underline b$
\end{LTXexample}
\begin{LTXexample}
\begin{align*}
a &= b\\
\vphantom{\int} c &= d\\
\int a &= b
\end{align*}
\end{LTXexample}
\end{frame}

\section{Feynman-Graphen}
\begin{frame}[fragile]{Feynman-Graphen}
• verschiedene Möglichkeiten für Feynman-Graphen:
• Paket |feynmf|
• Paket |feyn|
• Graphiksoftware
• Metafont
• TikZ/PS-Tricks
• …
\•
\end{frame}

\begin{frame}[fragile]{feyn}
• kleines, leicht bedienbares Paket
• bietet eine Matheschrift, mit der Feynman-Graphen gesetzt werden können
• (halb)intuitive Bedienung
• |\feyn|: Mathemodus
• |\Feyn|: Textmodus
• |\Diagram|: Komplexe Diagramme
\•
\end{frame}

\begin{frame}[fragile]{feyn}
\begin{LTXexample}
\[\feyn{f+g}\]
\[\feyn{fA} \feyn{gV}\]
\end{LTXexample}
\begin{LTXexample}
\[\Diagram{\vertexlabel^a \\
  fd \\
    & g\vertexlabel_{\mu,c} \\
  \vertexlabel^b fu\\
}
= ig\gamma_\mu (T^c)_{ab}\]
\end{LTXexample}
\pause
⇒ siehe (sehr gute) Dokumentation von |feyn|
\end{frame}

\section{relsize}
\begin{frame}[fragile]{Relative Größenangabe}
• Wenn normale Schriftgrößen nicht ausreichen:\\%
|\displaystyle, \textstyle, \scriptstyle, \scripscriptstyle|
• Paket |relsize|
• Grundbefehle |\relsize{n}|, |n| gibt Schrittweite an
• |\larger = \relsize{1}|
• |\smaller = \relsize{-1}|
• |\relscale{0.75}| – Skalierung auf den angegebenen Faktor
• |\mathsmaller|, |\mathlarger| – Änderung der Matheschriftgröße
\•
\end{frame}

\begin{frame}[fragile]{Relative Größenangabe}
\begin{LTXexample}[pos=b]
\[\Delta \varphi = 2
\int\limits_{r_{\min}}^{r_{\max}} \frac{ \dfrac{M}{r^2} dr} 
{\sqrt{2m (E-U) - \dfrac{M^2}{r^2}}}
\]
\end{LTXexample}
\end{frame}

\begin{frame}[fragile]{Relative Größenangabe}
\begin{LTXexample}[pos=b]
\newcommand\largeint{\mathlarger{\mathlarger{\mathlarger{\int}}}}
\[\Delta \varphi = 2
\largeint\limits_{r_{\min}}^{r_{\max}} \frac{ \dfrac{M}{r^2} dr} 
{\sqrt{2m (E-U) - \dfrac{M^2}{r^2}}}
\]
\end{LTXexample}
\end{frame}

\section{12many}
\begin{frame}[fragile]{12many}
• \pkg{12many} bietet Vereinfachung und Anpassung zum Mengensatz:\\%
$\{1, \dots, m\}$
• Befehle:\\
|\nto{n}{m}|, |\ito{m}|, |\oto{m}|
• Stil ändern mit |\setOTMstyle[]{style}|
\•
\begin{LTXexample}
\nto{n}{m}
\end{LTXexample}
\end{frame}
\stepcounter{section}
\section{amsmath}
\begin{frame}{amsmath}
sollte bekannt sein \fontspec{DejaVu Sans}☺
\end{frame}

\section{breqn}
\begin{frame}{Umbruch von Formeln}
• nicht nur Text, sondern auch lange Formeln müssen umbrochen werden
• sinnerhaltender Umbruch schwer
• Umbruch nur im Inline-Mode
• Umbruch nur bei binären Operatoren
\•
\end{frame}

\begin{frame}[fragile]{Umbruch von Formeln}
• \pkg{breqn} ermöglicht Umbruch in Display-Formeln
• eigene Umgebungen: |dmath(*)| (wie |\[ \]|)
• |dseries| 
• |dgroup| (wie |align|)
• |darray| (wie |eqnarray|)
• |dsuspend| (unterbricht)
• Befehl |\condition| für Bedingungen
\•
\end{frame}

\begin{frame}[fragile]{Probleme}
• \pkg{breqn} lädt \pkg{flexisym}
• \pkg{flexisym} definiert eigene Mathezeichen
•[⇒] Inkompatibilität mit Schriftpaketen
• speziell \alert{inkompatibel} zu \pkg{fontspec} (nicht mehr?)
\•
\end{frame}

\section{cases}
\begin{frame}[fragile]{multi-case equations}
• \pkg{cases} bietet Nummerierung von case-Konstrukten:
\• 
\begin{LTXexample}[pos=b]
\begin{numcases}{E = mc^2}
m \neq 0 & Masselose Teilchen\\
m < 0 & Antiteilchen (?)\\
m > 0 & normale Teilchen
\end{numcases}
\end{LTXexample}
\end{frame}

\section{esvect}
\begin{frame}[fragile]{Schöne Vektoren}
• manchmal hat man spezielle Anforderungen an die Vektorpfeile
• Paket \pkg{esvect} bietet Anpassungen der Pfeilform
• korrekter Satz bei Subskripten wird beachtet
\•
\begin{LTXexample}[preset={\obeylines}]
$\vv a$
$\vec a$
$\vv a$
\end{LTXexample}
• Pfeiltyp über Paketoption |[a]| bis |[h]| einstellbar
• mögliche Pfeile: siehe Dokumentation
\•
\end{frame}

\begin{frame}[fragile]{Schöne Vektoren}{Subskripte}
• Sternversion |\vv*{}{}| sorgt für passende Subskripte:
\•
\begin{LTXexample}[preset={\obeylines}]
$\vec{ab}_{\Delta}$\\[-2ex]
$\vv {ab}_{\Delta}$\\[-2ex]
$\vv*{ab}{\Delta}$
\end{LTXexample}

\end{frame}

\section{gnuplottex}
\begin{frame}[fragile]{Plotten in \LaTeX}
• \pkg{gnuplottex} bietet Interface zum Plotten aus dem \LaTeX-Dokument
• Umgebung |gnuplot| ruft Programm |gnuplot| auf dem Rechner auf
• Inhalt der Umgebung wird an Programm übergeben
• Plot wird als Graphik eingefügt
\•
\end{frame}

\begin{frame}{gnuplot}{was ist das?}
• kommandozeilenorientiertes Plotprogramm
• klein, schnell
• unintuitive Bedienung
• optimal für Ausführung aus Skripten
•[⇒] passt zur Arbeitsweise mit \TeX
• nützlich für schnelle Testplots
• auch professionelle Qualität möglich
\•
\end{frame}

\begin{frame}{gnuplot}{Plotten in \LaTeX}
• {\color{green}Vorteile}: Plotbefehle direkt im Dokument\\%
  Schriften von \LaTeX\ verwaltet ⇒  passend!
• \alert{Nachteile}: Portabilität leidet\\%
  Plot wird bei jedem Durchlauf neu erstellt\\%
  umständlich unter Windows
\•
\end{frame}

\begin{frame}[fragile]{gnuplot}{Verwendung}
• Start aus Kommandozeile (unter Windows GUI verfügbar)
• Grundbefehl: |plot|
• Abkürzungen aller Befehle möglich: |plot| = |pl| = |p|
• |p sin(x)|, |p "Datensatz" using 1:3|
• |set style data lines|, |rep|
\•
\end{frame}

\begin{frame}[fragile]{gnuplot}{Ausgabe}
• gnuplot bietet riesige Vielzahl an Ausgabeformaten
• u.\,a. ps, jpeg, mf, mp, hp500c, gif
• direkte Anzeige: wxt (windows), X11 (Unix)
• viele \TeX-Formate (pstex, pslatex, texdraw, eepic, emtex, \dots)
• \emph{kein} pdf
• aus \LaTeX: unabhängig vom Treiber
\•
\end{frame}

\begin{frame}[fragile]{gnuplot}{gnuplottex}
\begin{LTXexample}
\begin{gnuplot}[scale=0.3]
p sin(x)
\end{gnuplot}
\begin{gnuplot}[scale=0.4]
set style data linespoints
p "beispiele/plotdaten.gpt"
\end{gnuplot}
\end{LTXexample}
\end{frame}

\section{mathtools}
\begin{frame}{mathtools}
• Paket \pkg{mathtools} bietet:
• Erweiterungen/Ergänzungen/Bugfixes zu \pkg{amsmath}
• fine-tuning des Mathesatzes
• Sammlung von Tricks von Michael J. Downes
\•
\end{frame}

\begin{frame}[fragile]{mathtools}{fine-tuning: smashing}
\begin{LTXexample}[pos=t]
\[X = \sum_{1\le i\le j\le n} X_{ij}
X = \sum_{\mathclap{1\le i\le j\le n}} X_{ij}
X = \sum_{\mathclap{1\le i\le j\le n}}^{a+b+c+d} X_{ij}
X = \smashoperator[r]{\sum_{1\le i\le j\le n}^{a+b+c+d}} X_{ij}
\]
\end{LTXexample}
\end{frame}

\begin{frame}[fragile]{mathtools}{tags}
• Standardform der tags ist nicht immer schön: (4)
• Änderung mittels \pkg{amsmath}\\%
„[is] not very user friendly (it involves a macro with three @’s in its name)“
• \pkg{mathtools}’ Weg:
\• 
\begin{LTXexample}[width=.3\textwidth]
\newtagform{brackets}{[}{]}
\usetagform{brackets}
\begin{equation}E \neq mc^3\end{equation}
\newtagform{bfbrackets}[\textbf]{[}{]}
\usetagform{bfbrackets}
\begin{equation}E \neq mc^4\end{equation}
\end{LTXexample}
\end{frame}

\section{ulem, soul}
\begin{frame}{Streichen}{durch, unter, quer, …}
• Pakete \pkg{ulem} und \pkg{soul} bieten verschiedene Hervorhebungsmakros\pause
• \pkg{ulem}: \emph{underline-emphasize}
• \pkg{soul}: \emph{space out, underline}
\•
\end{frame}

\begin{frame}[fragile]{ulem}
• Hauptzweck: Ändern von |\emph| zu |\underline|
• falls nicht gewünscht: |\normalem| oder Option |normalem|
• andere Befehle:
\•
\begin{LTXexample}
\uline{test}
\uuline{test}
\uwave{test}
\sout{test}
\xout{test}
\useunder{\uwave}{\bfseries}{\textbf}
\textbf{test}
\end{LTXexample}
\end{frame}

\begin{frame}[fragile]{soul}
\begin{LTXexample} 
\so{letterspacing}
\caps{CAPITALS, Small Caps}
\ul{underline}
\st{strikeout}
\hl{highlight}
\sethlcolor{blue}
\setulcolor{red}
\setulcolor{green}
\hl{highlight}
\end{LTXexample}
\end{frame}

\section{\TeX{} – live}
\begin{frame}{Probleme mit breqn und fontspec}
\begin{block}{Antwort von Nicolas Vaughan}
Hi Arno,\\
When I have symbol clashes, I use a tip suggested by Scott Pakin in his
Comprehensive LaTeX Symbol List.
\end{block}
\end{frame}
\footnotesize
\begin{verbatim}
%%%%%%%%%%%%%%%%%%%%%%%%%%%%%%%%%%%%%%%%%%%%%%%%%%%%%%%%%%%%%%%%%%%%%%%%%%
% There are a number of symbols (e.g., \Square) that are defined by      %
% multiple packages.  In order to typeset all the variants in this       %
% document, we have to give glyph a unique name.  To do that, we define  %
% \savesymbol{XXX}, which renames a symbol from \XXX to \origXXX, and    %
% \restoresymbols{yyy}{XXX}, which renames \origXXX back to \XXX and     %
% defines a new command, \yyyXXX, which corresponds to the most recently %
% loaded version of \XXX.                                                %

% Save a symbol that we know is going to get redefined.
\def\savesymbol#1{%
  \expandafter\let\expandafter\origsym\expandafter=\csname#1\endcsname
  \expandafter\let\csname orig#1\endcsname=\origsym
  \expandafter\let\csname#1\endcsname=\relax
}
% Restore a previously saved symbol, and rename the current one.
\def\restoresymbol#1#2{%
  \expandafter\let\expandafter\newsym\expandafter=\csname#2\endcsname
  \expandafter\global\expandafter\let\csname#1#2\endcsname=\newsym
  \expandafter\let\expandafter\origsym\expandafter=\csname orig#2\endcsname
  \expandafter\global\expandafter\let\csname#2\endcsname=\origsym
}
\end{verbatim}

\begin{frame}[fragile]{Anpassung von fontspec}
\begin{block}{Antwort von Will Robertson}
Hi Arno :)
\\
Thanks for reporting the problem.
Until the next version of fontspec is released, you can write
\begin{verbatim}
\usepackage{breqn}
\usepackage[no-math]{fontspec}
\end{verbatim}
fontspec attempts to use |[no-math]| if a known maths font package is already loaded, but I forgot about including breqn in the list to check.
\end{block}
\end{frame}

\end{document}