\makeatletter
\@ifundefined{draftmodeon}{
  \documentclass[german,t]{beamer}
}{
  \documentclass[german,t,draft]{beamer}
}
\makeatother
\mode<presentation>{
  \usetheme{Frankfurt}
  \useoutertheme{infolines}
  \usecolortheme[RGB={0,100,130}]{structure}
  \useinnertheme{rounded}
  \setbeamertemplate{navigation symbols}{} 
}

\newcommand\subtitlei[1]{\def\insertsubtitlei{#1}}
\newcommand\subtitleii[1]{\def\insertsubtitleii{#1}}

\setbeamertemplate{title page}
  {
    \vspace*{1cm}
    \begin{centering}
      {\usebeamerfont{title}\usebeamercolor[bg]{title}
      {\Large \inserttitle}
      \\[.3cm]
      \insertsubtitlei\\
      \large \color{black}\insertsubtitleii
      }
\vspace*{1cm}
    \vbox to 1cm {\hfill \includegraphics[width=.2\textwidth]{ctanlion}}
      \usebeamerfont{subtitle}\usebeamercolor[bg]{subtitle}
      {\color{black}\Large \insertsubtitle}\vfill
    {\hfill \insertdate \hfill}
    \end{centering}

    \vfill\vfill

    \hspace*{-.47cm}
    \usebeamertemplate*{footline}
    \vspace*{-1.15cm}
}

\setbeamercolor*{author in head/foot}{parent=palette tertiary}
\setbeamercolor*{title in head/foot}{parent=palette secondary}
\setbeamercolor*{date in head/foot}{parent=palette primary}

\setbeamercolor*{section in head/foot}{parent=palette tertiary}
\setbeamercolor*{subsection in head/foot}{parent=palette primary}

\defbeamertemplate*{headline}{infolines theme changed}
{
  \leavevmode%
  \hbox{%
  \begin{beamercolorbox}[wd=.3\paperwidth,ht=2.25ex,dp=1.5ex,center]{title in head/foot}%
    \LaTeX-Kurs 2010
  \end{beamercolorbox}%
  \begin{beamercolorbox}[wd=.4\paperwidth,ht=2.25ex,dp=1.2ex,right]{section in head/foot}%
     \hfill \coursenumber\ – \coursetitle \hfill\hfill
  \end{beamercolorbox}%
  \begin{beamercolorbox}[wd=.3\paperwidth,ht=2.25ex,dp=1.2ex,left]{subsection in head/foot}%
    \hspace*{2em}\insertsectionhead
  \end{beamercolorbox}}%
  \vskip0pt%
}

\defbeamertemplate*{footline}{infolines theme changed}
{
  \leavevmode%
  \hbox{%
  \begin{beamercolorbox}[wd=.3\paperwidth,ht=2.25ex,dp=1ex,center]{author in head/foot}%
      \insertshortauthor
  \end{beamercolorbox}%
  \begin{beamercolorbox}[wd=.4\paperwidth,ht=2.25ex,dp=1ex,center]{title in head/foot}%
    \insertshortdate{}
  \end{beamercolorbox}%
  \begin{beamercolorbox}[wd=.3\paperwidth,ht=2.25ex,dp=1ex,center]{author in head/foot}%
    \insertframenumber{} / \inserttotalframenumber
  \end{beamercolorbox}}%
  \vskip0pt%
}

\setbeamersize{text margin left=1em,text margin right=1em}

\def\extractnumber"#1-#2".{#1}
\def\extracttitle"#1-#2".{#2}
\def\coursenumber{\expandafter\extractnumber\jobname.}
\def\coursetitle{\expandafter\extracttitle\jobname.}

\usepackage{
  babel,
  dtklogos,
  moreverb,
  shortvrb,
  xltxtra,
  yfonts
}
\usepackage[final]{showexpl}

\hypersetup{
  colorlinks=true,
  breaklinks=true,
  linkcolor=blue,
  urlcolor=blue,
  citecolor=black, filecolor=black, menucolor=black,
  pdfauthor={Arno L. Trautmann},
}

\MakeShortVerb|

\author{Arno Trautmann}
\institute{Heidelberg}
\title{Einführung in das Textsatzsystem\\[2ex] \Huge \LaTeX}
\subtitlei{Vorlesungsreihe im Sommersemester 2010}
\subtitleii{\textfrak{universitatis:~studii~heydelbergensis:}}
\subtitle{\coursenumber\ – \coursetitle}

\graphicspath{{../Mediales/}}

\logo{}%\includegraphics[width=7.4em]{unilogo.svg}}

\defaultfontfeatures{Scale=MatchLowercase}
\setmonofont{DejaVu Sans Mono}

\newenvironment{mydesc}{\begin{tabular}{|>{\columncolor{lightgray}\color{blue}}rl|}\hline}{\\\hline\end{tabular}}

\def\pkg#1{\texttt{#1}}
\def\macro#1{|#1|}

\def\plainTeX{\textsf{plain\TeX}}

\catcode`\⇒=\active
\def⇒{\ensuremath{\Rightarrow}}
\catcode`\⇐=\active
\def⇐{\ensuremath{\Leftarrow}}
\catcode`\…13
\let…\dots

\newcommand\notiz[1]{}
\newcommand\einschub[1][]{\textcolor{red}{⇒}}
\renewcommand\checkmark{\color{green}\ding{51}}
\newcommand\cross{\color{red}\ding{53}}

\newcommand\pdf[2][]{\bgroup
  \setbeamercolor{background canvas}{bg=}%
  \includepdf[#1]{#2}%
    \egroup
}

\newenvironment{twoblock}[2]{
  \begin{columns}
  \begin{column}{.46\textwidth}
  \begin{block}{#1}
\def\nextblock{
  \end{block}
  \end{column}
\ %
  \begin{column}{.46\textwidth}
  \begin{block}{#2}
}
}
{
  \end{block}
  \end{column}
  \end{columns}
}

\AtBeginDocument{
  \lstset{%
    backgroundcolor=\color[rgb]{.9 .9 .9},
    basicstyle=\ttfamily\small,
    breakindent=0em,
    breaklines=true,
    commentstyle=,
    keywordstyle=,
    identifierstyle=,
    captionpos=b,
    numbers=none,
    frame=tlbr,%shadowbox,
    frameround=tttt,
    pos=r,
    rframe={single},
    explpreset={numbers=none}
  }
}
\makeatletter
\g@addto@macro\beamer@lastminutepatches{ % thanks to Ulrike for this!
  \frame[plain,t]{\titlepage}
  \frame{\centerline{\huge \color[RGB]{0,100,130}Inhalt}\tableofcontents}
}

%—— itemize-hack
\def\outside{o}
\def\inside{i}
\let\insideitemizei\outside
\let\insideitemizeii\outside
\def\altenditemize{
  \if\altlastitem 1%
    \let\altlastitem0%
  \else%
    \end{itemize}%
    \let\insideitemizei\outside%
  \fi%
}

\begingroup
  \lccode`\~=`\^^M%
\lowercase{%
  \endgroup
  \def\makeenteractive{%
    \catcode`\^^M=\active
    \let~\altenditemize
}%
}

\def\newitemi{%
  \ifx\insideitemizei\inside%
    \let\altlastitem1%
    \expandafter\item%
  \else%
    \begin{itemize}%
    \let\insideitemizei\inside%
    \let\altlastitem1%
    \makeenteractive%
    \expandafter\item%
  \fi
}

\def\newitemii{
  \ifx\insideitemizeii\inside
    \expandafter\item%
  \else
    \begin{itemize}
      \let\insideitemizeii\inside
      \expandafter\item%
  \fi
}

\def\makeitemi#1{%
  \expandafter\ifx\csname cc\string#1\endcsname\relax
    \add@special{#1}%
    \expandafter
    \xdef\csname cc\string#1\endcsname{\the\catcode`#1}%
    \begingroup
      \catcode`\~\active  \lccode`\~`#1%
      \lowercase{%
      \global\expandafter\let
         \csname ac\string#1\endcsname~%
      \expandafter\gdef\expandafter~\expandafter{\newitemi}}%
    \endgroup
    \global\catcode`#1\active
  \else
  \fi
}

\def\makeitemii#1{%
  \expandafter\ifx\csname cc\string#1\endcsname\relax
    \add@special{#1}%
    \expandafter
    \xdef\csname cc\string#1\endcsname{\the\catcode`#1}%
    \begingroup
      \catcode`\~\active  \lccode`\~`#1%
      \lowercase{%
      \global\expandafter\let
         \csname ac\string#1\endcsname~%
      \expandafter\gdef\expandafter~\expandafter{\newitemii}}%
    \endgroup
    \global\catcode`#1\active
  \else
  \fi
}

\def\add@special#1{%
  \rem@special{#1}%
  \expandafter\gdef\expandafter\dospecials\expandafter
{\dospecials \do #1}%
  \expandafter\gdef\expandafter\@sanitize\expandafter
{\@sanitize \@makeother #1}}
\def\rem@special#1{%
  \def\do##1{%
    \ifnum`#1=`##1 \else \noexpand\do\noexpand##1\fi}%
  \xdef\dospecials{\dospecials}%
  \begingroup
    \def\@makeother##1{%
      \ifnum`#1=`##1 \else \noexpand\@makeother\noexpand##1\fi}%
    \xdef\@sanitize{\@sanitize}%
  \endgroup}
\AtBeginDocument{
  \makeitemi{•}
}
%——beamer versaut hier irgendwas, daher muss itemize explizit beendet werden!
\def\•{\end{itemize}}
%—— itemize-hack
\makeatother

\begin{document}
\section{Organisatorisches}
\begin{frame}{Organisatiorisches}{Anmeldung}
\begin{block}{Moodle}
Anmeldung obligatorisch!
• http://elearning.uni-heidelberg.de
•[⇒] Fakultät für Mathematik und Informatik
•[⇒] Suchen nach „LaTeX“
•[⇒] „LaTeX-Kurs (SS 10)“
• \url{http://elearning.uni-heidelberg.de/course/view.php?id=2885}
\•
\end{block}
\end{frame}

\begin{frame}{Organisatiorisches}{Material}
\begin{block}{Material im Moodle}
• Folien als pdf-Dateien
• Übungsblätter als pdf-Dateien
• Ergänzungen, Anmerkungen, Links, etc.
\•
\end{block}
\begin{block}{Material auf github}
• unter \url{http://github.com/alt/LaTeX-course}
• Quellcode zu allen relevanten Dateien (Folien, Übungen, Lösungen, …)
• schneller, einfacher Zugriff per git\,/\,im Browser
• Entwicklung „live“ einsehbar
\•
\end{block}
\end{frame}

\begin{frame}{Organisatorisches}{Scheine}
\begin{block}{Scheinbedingungen}
• 2 SWS Vorlesung/Übung
• Übungsblätter müssen bearbeitet werden
• 50\% der Übungspunkte für Scheinvergabe
• Übungspunkte ergeben Scheinnote
• 2 ECTS-Punkt (BSc, Informatik)
\•
\end{block}
\end{frame}

\begin{frame}{Organisatorisches}{Übungsbetrieb}
\begin{block}{Übungen}
• Blätter im Moodle/github, Ausgabe Freitag nach Vorlesung
• Abgabe Freitag \emph{vor} Vorlesung
• Abgaben in Dreiergruppen möglich (ausgenommen erstes Blatt!)
• Besprechung am Beginn der nächsten Vorlesung
• Bereitschaft zur Vorstellung der eigenen Lösung
\•
\end{block}
\end{frame}

\begin{frame}[fragile]{Organisatorisches}{Übungsbetrieb}
\begin{block}{Übungsabgabe}
• Je nach Aufgabenstellung: Ausdruck, per Mail, handschriftlich etc.%
\\ E-Mail: Herpich@stud.uni-heidelberg.de\\%
Betreff: |LaTeX-Übung: Musterfrau, Mustermann|
• Dateinamen: |uebung1.1_musterfrau_mustermann.tex|
• Formatierung bitte einhalten, da sonst großer Mehraufwand beim Sortieren!
• Ausdrucke freitags vor Vorlesungsbeginn abgeben
\•
\end{block}
\end{frame}

\begin{frame}{Inhalt (vorläufig)}
\begin{enumerate}\let•\item
• Einführung, Übersicht
• Schriften, Zeichensätze, Kodierungen
• allgemeine Formatierung; Pakete
• Gleitobjekte: Bilder, Tabellen, Verzeichnisse
• Mathesatz
• Umfangreiche, mehrsprachige Dokumente
• Präsentationen
• Typographische Feinheiten
• Professionelle Briefe, Lebenslauf
• \LaTeX-Pakete und -Klassen verstehen\,/\,selbst schreiben
•[…] weitere Vorschläge?
•[\textbullet] Einschub: Con\TeX t
\end{enumerate}
\end{frame}

\begin{frame}{Bestandteile der Vorlesungen}
\begin{itemize}
\item  Nutzung von \LaTeX
\\ \begin{quote}\alert{Wie} erreiche ich, was ich haben will?\end{quote}
\item Verstehen von \LaTeXTeX
\\ \begin{quote} \alert{Was} passiert da eigentlich, wenn ich auf den Knopf drücke?\end{quote}
\item Typographische Tips
\\ \begin{quote}\alert{Warum} sollte ich das genau so setzen und nicht anders?\end{quote}
\end{itemize}
\end{frame}

\begin{frame}{„Laientypographie“}
\begin{quote}
Typographie ist doch Geschmackssache. Ich mach das so, wie es schön aussieht!
\end{quote}
\pause
\begin{block}{Hans Peter Willberg/Friedrich Forssmann}
\begin{quotation}Das Selbermachen ist längst üblich, die Ergebnisse oft fragwürdig, weil Laien-Typografen nicht sehen, was nicht stimmt und nicht wissen können, worauf es ankommt. So gewöhnt man sich an falsche und schlechte Typografie. [\dots] Jetzt könnte der Einwand kommen, Typografie sei doch Geschmackssache. Wenn es um Dekoration ginge, könnte man das Argument vielleicht gelten lassen, da es aber bei Typografie in erster Linie um Information geht, können Fehler nicht nur stören, sondern sogar Schaden anrichten.
\end{quotation}
\end{block}
\end{frame}

\section{The name of the game}
\begin{frame}[<+->][t]{The name of the game}
• Programm \alert{\TeX} (Seit 1977)
\only<1>{\\ Geschrieben von Donald E. Knuth für sein Buch „The Art of Computer Programming“.
\\ „\TeX“ von griechisch \fontspec{Arno Pro}τέχνη}
• Makropaket \alert{plain}\TeX
\only<2>{\\ Macht \TeX\ für normale Nutzer bedienbar.}
• großes Makropaket \alert{La}\TeX\ (Anfänge 1980er)
\only<3>{\\ Von Leslie Lamport: „Lamport’s \TeX“.\\ Viele Vereinfachungen für den normalen Anwender.}
• aktuelle, stabile Version: \alert{La}\TeX\,\alert{2$_\varepsilon$} (1994)
\only<4>{\\ „in einer $\varepsilon$-Umgebung von 2“…}
• zukünftige Entwicklung: \LaTeX3 \\ noch nicht eigenständig verfügbar, aber als Paket \texttt{expl3} in \LaTeXe\\ (bietet \LaTeX3-Syntax für Paketautoren)
\•
\end{frame}

\section{Was ist \TeX?}
\begin{frame}{ Was ist \TeX\ – was nicht?}
\only<1>{\begin{block}{Was kann \TeX?}
• Programm, um „The Art of Computer Programming“ zu schreiben
• Für alle Schriftstücke mit logischem Aufbau geeignet:
• Naturwissenschaftliche Arbeiten (hervorragender Mathesatz)
• Geisteswissenschaftliche Arbeiten (hervorragende Mehrsprachigkeit, Bibliographieerstellung, Erstellung von Apparaten etc.)
• Artikel, Diplomarbeiten, Dissertationen, …
• Buchreihen, Briefe
• Präsentationen
• Unmenge an „Missbrauch“ durch kreative Paketautoren
\•
\end{block}}
\only<2>{\begin{block}{Was kann \TeX\ nicht?}
• Präsentationen (bunt, drehend, blinkend, „durcheinander“)
• Werbezettel
• Plakate 
• Alles ohne logische Struktur
• Dokumente mit vielen uneinheitlichen Bildern, die frei bewegt werden
\• 
\end{block}}
\end{frame}

\section[Erste Schritte]{Erste Schritte – \TeX\ for the impatient}
\begin{frame}[fragile]{Wie funktioniert \TeX?}
\begin{block}{Das Prinzip von \TeX}
• reine Textdateien
• keine Einstellungen, die der Nutzer nicht sieht
• kein „Rumklicken“
• Auszeichnungen besonderer Textstellen durch Befehle:\\%
  „Ich will einen Artikel schreiben“\\%
  „schreibe das Folgende fett“\\%
  „schreibe eine Überschrift“\\%
  „setze eine Tabelle mit 3 Spalten, die links, rechts, und …“
\•
\end{block}
\end{frame}

\begin{frame}{Wie funktioniert \TeX?}
\begin{columns}\begin{column}{.45\textwidth} 
\begin{block}{ Vorteile}
• Extreme Stabilität
• Bearbeiten von Textdatein ist nicht aufwendig
• Portabilität
• Dateien mit beliebigem Programm lesbar
• Winzige Dateigrößen auch bei riesigen Projekten
• Trennung von z.\,B. Bildern und Text
\• 
\end{block}
\end{column}
\begin{column}{.45\textwidth}
\begin{block}{ Nachteile}
• Dokument ist nicht direkt sichtbar
• Unintuititve Bedienung
• Bei Änderungen muss alles neu kompiliert werden
• Trennung von z.\,B. Bildern und Text
\•
\end{block}
\end{column}
\end{columns}
\end{frame}

\begin{frame}[fragile]{Ein einfaches \TeX-Dokument}
•<+->„Quellcode“, ähnlich Programmiersprachen.
•<+->Problem: Wie Text von Befehlen unterscheiden?
•<+->In Programmiersprachen: Befehle werden direkt eingegeben, Text in Anführungszeichen:
\•
\pause
\begin{lstlisting}
print ("Hallo Welt!")
\end{lstlisting}
⇒ Für ein Textsatzprogramm ungeeignet.
\end{frame}

\begin{frame}[fragile]{Ein einfaches \TeX-Dokument}
• In \TeX: Einzelne Zeichen haben besondere Bedeutungen:
• \emph{escape character}: Backslash \verb|\|\\
markiert den Anfang eines Befehls: \verb|\title, \author|%
\• 
\begin{lstlisting}
Hallo Welt! \bye
\end{lstlisting}

\pause\vspace{1cm}
⇒ Erzeugt eine |dvi|-Datei (DeVice Independent)\\
(Darstellbar z.\,B. mit |yap|, |evince|, …)\\
In andere Formate (|ps|, |pdf|) umwandelbar
\end{frame}

\begin{frame}[fragile]{Weitere Befehlszeichen}
• \emph{grouping character}: Geschweifte Klammern \verb|{}|\\%
gruppieren Zeichen, die zusammen gehören, z.\,B. Argumente: \verb|\textbf{fetter Text}|
• \emph{math character} |$| startet\,/\,endet Mathemodus
• \emph{tabbing character} |&| trennt Spalten in Tabellen
• weitere Sonderzeichen: |# ^ _ ~ %| haben alle besondere Bedeutungen
\•
\end{frame}

\begin{frame}[fragile]{Ein einfaches \LaTeX-Dokument} 
\begin{LTXexample}
\documentclass{minimal}
\begin{document}
Hallo Welt!
\end{document}
\end{LTXexample}
\end{frame}

\begin{frame}{Dokumentenklassen}
• Legen das Layout des Dokumentes fest:
• Standardschriften
• Seitenaufteilung
• Gliederungsbefehle
• Aussehen von Verzeichnissen, Tabellen etc.
• Durch Änderung von Optionen oder Laden von Paketen änderbar
\•
\end{frame}

\begin{frame}{Dokumentenklassen}
\begin{tabular}{rl}
&\kern-1.5cm \bf Standardklassen\\
article & (Kurze) Artikel\\
report & Reporte, Tagungsberichte\\
book & Bücher\\
letter & Briefe\\
minimal & Für Minimalbeispiele\\ \\
&\kern-1.5cm \bf KOMA-Script\\
scrartcl & Erweiterung von article\\
scrreprt & Erweiterung von report\\
scrbook & Erweiterung von book\\
scrlttr2 & Sehr mächtige Briefklasse\\ \\
&\kern-1.5cm \bf Spezialklassen\\
beamer & Für Beamer-Präsentationen\\
powerdot & einfachere Präsentationen
\end{tabular}
\end{frame}

\begin{frame}[fragile]{Gliederungsbefehle}
• Gliederungen strukturieren Dokumente,
• ermöglichen automatische Nummerierung, Eintragung in Verzeichnisse, Kolumnentitel etc.
• Werden von der Dokumentenklasse definiert
• Grundstruktur im Kernel festgelegt
•[⇒] bestimmte Elemente immer verfügbar
\•
\vfill
\begin{lstlisting}
\part{Mechanik und Wärme}
\chapter{Gase}
\section{Transportprozesse in Gasen}
\section{Diffussion}
\subsubsection{Unterunterabschnitt}
\paragraph{Paragraph}
\subparagraph{Unterparagraph}
\end{lstlisting}
\end{frame}

%\pdf[pages={1,3}]{beispiele/gliederung_scrbook.pdf}
%\pdf[pages=1]{beispiele/gliederung_scrartcl.pdf}

\begin{frame}[fragile]{Pakete}
• Pakete liefern zusätzlichen Code
• Vereinfachen die Arbeit
• Korrigieren Fehler
• Bieten viele zusätzliche Features
• Einbinden mittels |\usepackage[options]{paketname}| in Präambel:
\• 

\begin{lstlisting}
\documentclass{scrartcl}
\usepackage{
  amsmath,
  keyboard
}
\usepackage[left=2cm]{geometry}
\end{lstlisting}

\end{frame}

\begin{frame}{Nützliche Pakete (sehr kleine Auswahl)}
\begin{block}{}
\begin{tabular}{rl}
graphics/x & bietet (erweiterte) Graphikunterstützung\\
amsmath & Verbesserungen, Erweiterungen für den Mathesatz\\
babel/polyglossia & Vielsprachigkeit\\
lmodern & Stellt auf die lmodern-Schriften um\\
xltxtra & wichtige Pakete für \XeLaTeX\\
inputenc & ermöglicht verschiedene Eingabekodierungen\\
fontenc & ermöglicht verschiedene Fontkodierungen\\
tikz & bietet sehr mächtige Zeichenumgebung\\
\vdots & \vdots
\end{tabular}
\end{block}
\end{frame}


\begin{frame}[c,fragile]{Grundbefehle}{allgemein}
\begin{tabular}{ll}
|\textrm{Serifen}| & \textrm{Serifen agy}\\
|\textit{kursiv}| & \textrm{\textit{kursiv agy}} (\emph{nicht} |{\it kursiv}| verwenden)\\
|\textsl{geneigt}| & \textrm{\textsl{geneigt agy}}\\
|\textsf{serifenlos}| & \textsf{serifenlos}\\
|\textbf{fett}| & \textbf{fett}\\
|\texttt{Schreibmaschine}| & \texttt{Schreibmaschine}\\
|\textsc{Kapitälchen}| & \textsc{Kapitälchen}\\
|\emph{Hervorhebung}| & \emph{Hervorhebung}\\
|\\| & Zeilenende\\
|\par| oder Leerzeile & Absatzende\\
|$E = \frac{p^2}{2m}$| & Inline-Mathemodus: $E = \frac{p^2}{2m}$\\
|\[E = \frac{p^2}{2m}\]| & Display-Mathemodus: $\displaystyle E = \frac{p^2}{2m}$\\ % well that’s cheated …
|\tableofcontents| & Produziert Inhaltsverzeichnis
\end{tabular}
\end{frame}

\begin{frame}[c,fragile]{Grundbefehle}{Schriftgrößen}
\begin{block}{}
\begin{tabular}{ll}
|\tiny| & \tiny winzig \\
|\small| & \small klein \\
|\normalsize| & \normalsize normal\\
|\large| & \large große\\
|\Large| & \Large größer\\
|\LARGE| & \LARGE noch größer\\
|\huge| & \huge riesig\\
|\Huge| & \Huge noch riesiger\\
\end{tabular}
\end{block}
Manuelle Anpassung: |\fontsize{10}{12}\selectfont|
\end{frame}

\section[Warum LaTeX?]{Was ist toll an \LaTeX?}
\begin{frame}{\LaTeX\ vs. Word}{Portabilität}
\begin{block}{\strut\LaTeX}
• Dokument als Textdatei
•[⇒] \emph{immer} lesbar!
• Ausgabe als dvi oder pdf
•[⇒] Aussehen überall gleich
\•
\end{block}
\end{frame}

\begin{frame}{\LaTeX\ vs. Word}{Portabilität}
\begin{block}{Word}
• Nicht offenes Format
•[⇒] Nur mit größerem Aufwand lesbar
• Neuere Versionen nicht kompatibel
• Immer neuste Version nötig (nicht kostenlos!)
• Ausgabe als pdf:
• Schriften werden oft \emph{nicht} eingebunden
•[⇒] Dokument kann überall anders aussehen!
\• 
\end{block}
\end{frame}

\section{(Weiter-)Entwicklungen von \TeX}
\begin{frame}[c]{Entwicklung von \TeX\ in über 3 Jahrzehnten …}
In über 30 Jahren \TeX\ gab es viele Entwicklungen des Programms selbst sowie der Makropakete, die die Bedienung vereinfachen.\\
Eine kurze Übersicht:
\\ \url{http://github.com/alt/tex-overview}
\end{frame}

\section{Hilfsdateien – Datenmüll?}
\begin{frame}{So viele Dateien?}
• \LaTeXTeX\ verwendet einige Hilfsdateien:
• speichern von temporären Informationen
• Verzeichniseinträge
• Spracheinstellungen
• Ausgabe von Fehlermeldungen
• etc. …
\•
\end{frame}

\begin{frame}[c]{Hilfsdateien}
\begin{tabular}{ll}
.tex & \TeX-Datei mit Dokumententext\\ \\
& \bf Ausgabeformate\\
.dvi & normale \TeX-Ausgabe\\
.xdv & \XeTeX-Ausgabe\\
.pdf & pdf\TeX-Ausgabe oder Umwandlung von (x)dvi\\ \\\pause
& \bf Hilfsdateien (nur schreiben)\\
.log & Log-Datei mit Informationen, Warnungen, Fehlern\\ \\\pause
& \bf Hilfsdateien (schreiben und lesen)\\
.aux & Hilfsdatei mit temporären Informationen\\
.toc & table of contents\\
.lof & list of figures\\
.synctex.gz & nötig für die Sync\TeX-Funktion\\
\vdots & \vdots
\end{tabular}
\end{frame}

\section{Installation von \TeX live}



\begin{frame}[c]{}
\Huge Happy \TeX{}ing!
\end{frame}
\end{document}