\makeatletter
\@ifundefined{draftmodeon}{
  \documentclass[german,t]{beamer}
}{
  \documentclass[german,t,draft]{beamer}
}
\makeatother
\mode<presentation>{
  \usetheme{Frankfurt}
  \useoutertheme{infolines}
  \usecolortheme[RGB={0,100,130}]{structure}
  \useinnertheme{rounded}
  \setbeamertemplate{navigation symbols}{} 
}

\newcommand\subtitlei[1]{\def\insertsubtitlei{#1}}
\newcommand\subtitleii[1]{\def\insertsubtitleii{#1}}

\setbeamertemplate{title page}
  {
    \vspace*{1cm}
    \begin{centering}
      {\usebeamerfont{title}\usebeamercolor[bg]{title}
      {\Large \inserttitle}
      \\[.3cm]
      \insertsubtitlei\\
      \large \color{black}\insertsubtitleii
      }
\vspace*{1cm}
    \vbox to 1cm {\hfill \includegraphics[width=.2\textwidth]{ctanlion}}
      \usebeamerfont{subtitle}\usebeamercolor[bg]{subtitle}
      {\color{black}\Large \insertsubtitle}\vfill
    {\hfill \insertdate \hfill}
    \end{centering}

    \vfill\vfill

    \hspace*{-.47cm}
    \usebeamertemplate*{footline}
    \vspace*{-1.15cm}
}

\setbeamercolor*{author in head/foot}{parent=palette tertiary}
\setbeamercolor*{title in head/foot}{parent=palette secondary}
\setbeamercolor*{date in head/foot}{parent=palette primary}

\setbeamercolor*{section in head/foot}{parent=palette tertiary}
\setbeamercolor*{subsection in head/foot}{parent=palette primary}

\defbeamertemplate*{headline}{infolines theme changed}
{
  \leavevmode%
  \hbox{%
  \begin{beamercolorbox}[wd=.3\paperwidth,ht=2.25ex,dp=1.5ex,center]{title in head/foot}%
    \LaTeX-Kurs 2010
  \end{beamercolorbox}%
  \begin{beamercolorbox}[wd=.4\paperwidth,ht=2.25ex,dp=1.2ex,right]{section in head/foot}%
     \hfill \coursenumber\ – \coursetitle \hfill\hfill
  \end{beamercolorbox}%
  \begin{beamercolorbox}[wd=.3\paperwidth,ht=2.25ex,dp=1.2ex,left]{subsection in head/foot}%
    \hspace*{2em}\insertsectionhead
  \end{beamercolorbox}}%
  \vskip0pt%
}

\defbeamertemplate*{footline}{infolines theme changed}
{
  \leavevmode%
  \hbox{%
  \begin{beamercolorbox}[wd=.3\paperwidth,ht=2.25ex,dp=1ex,center]{author in head/foot}%
      \insertshortauthor
  \end{beamercolorbox}%
  \begin{beamercolorbox}[wd=.4\paperwidth,ht=2.25ex,dp=1ex,center]{title in head/foot}%
    \insertshortdate{}
  \end{beamercolorbox}%
  \begin{beamercolorbox}[wd=.3\paperwidth,ht=2.25ex,dp=1ex,center]{author in head/foot}%
    \insertframenumber{} / \inserttotalframenumber
  \end{beamercolorbox}}%
  \vskip0pt%
}

\setbeamersize{text margin left=1em,text margin right=1em}

\def\extractnumber"#1-#2".{#1}
\def\extracttitle"#1-#2".{#2}
\def\coursenumber{\expandafter\extractnumber\jobname.}
\def\coursetitle{\expandafter\extracttitle\jobname.}

\usepackage{
  babel,
  dtklogos,
  moreverb,
  shortvrb,
  xltxtra,
  yfonts
}
\usepackage[final]{showexpl}

\hypersetup{
  colorlinks=true,
  breaklinks=true,
  linkcolor=blue,
  urlcolor=blue,
  citecolor=black, filecolor=black, menucolor=black,
  pdfauthor={Arno L. Trautmann},
}

\MakeShortVerb|

\author{Arno Trautmann}
\institute{Heidelberg}
\title{Einführung in das Textsatzsystem\\[2ex] \Huge \LaTeX}
\subtitlei{Vorlesungsreihe im Sommersemester 2010}
\subtitleii{\textfrak{universitatis:~studii~heydelbergensis:}}
\subtitle{\coursenumber\ – \coursetitle}

\graphicspath{{../Mediales/}}

\logo{}%\includegraphics[width=7.4em]{unilogo.svg}}

\defaultfontfeatures{Scale=MatchLowercase}
\setmonofont{DejaVu Sans Mono}

\newenvironment{mydesc}{\begin{tabular}{|>{\columncolor{lightgray}\color{blue}}rl|}\hline}{\\\hline\end{tabular}}

\def\pkg#1{\texttt{#1}}
\def\macro#1{|#1|}

\def\plainTeX{\textsf{plain\TeX}}

\catcode`\⇒=\active
\def⇒{\ensuremath{\Rightarrow}}
\catcode`\⇐=\active
\def⇐{\ensuremath{\Leftarrow}}
\catcode`\…13
\let…\dots

\newcommand\notiz[1]{}
\newcommand\einschub[1][]{\textcolor{red}{⇒}}
\renewcommand\checkmark{\color{green}\ding{51}}
\newcommand\cross{\color{red}\ding{53}}

\newcommand\pdf[2][]{\bgroup
  \setbeamercolor{background canvas}{bg=}%
  \includepdf[#1]{#2}%
    \egroup
}

\newenvironment{twoblock}[2]{
  \begin{columns}
  \begin{column}{.46\textwidth}
  \begin{block}{#1}
\def\nextblock{
  \end{block}
  \end{column}
\ %
  \begin{column}{.46\textwidth}
  \begin{block}{#2}
}
}
{
  \end{block}
  \end{column}
  \end{columns}
}

\AtBeginDocument{
  \lstset{%
    backgroundcolor=\color[rgb]{.9 .9 .9},
    basicstyle=\ttfamily\small,
    breakindent=0em,
    breaklines=true,
    commentstyle=,
    keywordstyle=,
    identifierstyle=,
    captionpos=b,
    numbers=none,
    frame=tlbr,%shadowbox,
    frameround=tttt,
    pos=r,
    rframe={single},
    explpreset={numbers=none}
  }
}
\makeatletter
\g@addto@macro\beamer@lastminutepatches{ % thanks to Ulrike for this!
  \frame[plain,t]{\titlepage}
  \frame{\centerline{\huge \color[RGB]{0,100,130}Inhalt}\tableofcontents}
}

%—— itemize-hack
\def\outside{o}
\def\inside{i}
\let\insideitemizei\outside
\let\insideitemizeii\outside
\def\altenditemize{
  \if\altlastitem 1%
    \let\altlastitem0%
  \else%
    \end{itemize}%
    \let\insideitemizei\outside%
  \fi%
}

\begingroup
  \lccode`\~=`\^^M%
\lowercase{%
  \endgroup
  \def\makeenteractive{%
    \catcode`\^^M=\active
    \let~\altenditemize
}%
}

\def\newitemi{%
  \ifx\insideitemizei\inside%
    \let\altlastitem1%
    \expandafter\item%
  \else%
    \begin{itemize}%
    \let\insideitemizei\inside%
    \let\altlastitem1%
    \makeenteractive%
    \expandafter\item%
  \fi
}

\def\newitemii{
  \ifx\insideitemizeii\inside
    \expandafter\item%
  \else
    \begin{itemize}
      \let\insideitemizeii\inside
      \expandafter\item%
  \fi
}

\def\makeitemi#1{%
  \expandafter\ifx\csname cc\string#1\endcsname\relax
    \add@special{#1}%
    \expandafter
    \xdef\csname cc\string#1\endcsname{\the\catcode`#1}%
    \begingroup
      \catcode`\~\active  \lccode`\~`#1%
      \lowercase{%
      \global\expandafter\let
         \csname ac\string#1\endcsname~%
      \expandafter\gdef\expandafter~\expandafter{\newitemi}}%
    \endgroup
    \global\catcode`#1\active
  \else
  \fi
}

\def\makeitemii#1{%
  \expandafter\ifx\csname cc\string#1\endcsname\relax
    \add@special{#1}%
    \expandafter
    \xdef\csname cc\string#1\endcsname{\the\catcode`#1}%
    \begingroup
      \catcode`\~\active  \lccode`\~`#1%
      \lowercase{%
      \global\expandafter\let
         \csname ac\string#1\endcsname~%
      \expandafter\gdef\expandafter~\expandafter{\newitemii}}%
    \endgroup
    \global\catcode`#1\active
  \else
  \fi
}

\def\add@special#1{%
  \rem@special{#1}%
  \expandafter\gdef\expandafter\dospecials\expandafter
{\dospecials \do #1}%
  \expandafter\gdef\expandafter\@sanitize\expandafter
{\@sanitize \@makeother #1}}
\def\rem@special#1{%
  \def\do##1{%
    \ifnum`#1=`##1 \else \noexpand\do\noexpand##1\fi}%
  \xdef\dospecials{\dospecials}%
  \begingroup
    \def\@makeother##1{%
      \ifnum`#1=`##1 \else \noexpand\@makeother\noexpand##1\fi}%
    \xdef\@sanitize{\@sanitize}%
  \endgroup}
\AtBeginDocument{
  \makeitemi{•}
}
%——beamer versaut hier irgendwas, daher muss itemize explizit beendet werden!
\def\•{\end{itemize}}
%—— itemize-hack
\makeatother

\usepackage{
  blindtext,
  boxedminipage,
  ecltree,
  epic,
  eepic,
  fancybox,
  shadow,
  subfloat,
  wrapfig
}
\usepackage{bar}
\let\barbar\bar
\let\bar\relax

\begin{document}
\let\bar\barbar
\section{Graphikarten}
\begin{frame}{Bilder}
\begin{block}{Pixelgraphik}
• eine Menge an Punkten
• jedem Punkt wird eine Farbe zugeordnet
• Ergebnis von Photos, Scans, etc. 
• keine Skalierbarkeit
\•
\end{block}

\begin{block}{Vektorgraphik}
• Beschreibung durch mathematische Objekte
• Kurven (Bézier-Kurven, Polynome, …) o.\,ä.
• beliebige Skalierbarkeit
• meist kleine Dateigröße
• moderne Schriften sind Vektorgraphiken!
\•
\end{block}
\end{frame}

\begin{frame}[fragile]{\TeX\ und Bilder}
• \TeX\ stammt aus einer Zeit, in der Texte den Informationsaustausch dominierten
• \TeX\ kennt \emph{keine} Möglichkeit, externe Bilder einzubinden
• \TeX\ kennt nur Boxen
• Für alles, was darüber hinaus geht: |\special|
•[⇒] abhängig vom „Ausgabegerät“!
\•
\end{frame}

\section[Internes]{„Interne“ Graphiken}
\begin{frame}[fragile]{Portable Graphiken}
• \LaTeX-Pakete bieten Möglichkeiten, mittels \TeX-Code Graphiken zu erstellen
• Beschränkung auf Linien, Boxen und Zusammengesetztes
• komplexe Graphiken sind mit speziellen Paketen gut möglich
•[] aber: \alert{\LaTeX\ ist kein Zeichenprogramm}
\•
\end{frame}

\begin{frame}{Portable Graphiken}
• kreative Möglichkeiten:
• „Missbrauch“ von Schriften zum Erstellen von Graphiken\\%
z.B. \texttt{feyn}
• Graphik aus \TeX-Boxen zusammensezten \dots
\pause
• sinnvoll: richtige Boxen als graphische Textauszeichnung
\•
\end{frame}

\begin{frame}[fragile]{boxedminipage, shadow}
\begin{LTXexample}
\begin{boxedminipage}[t]{10em}
Ein kleines Beispiel für eine eingerahmte Minipage, in der beliebiger Inhalt stehen kann.
\end{boxedminipage}
\end{LTXexample}
\begin{LTXexample}
\shabox{Ein kleines Beispiel für eine Box mit Schatten.}
\end{LTXexample}
\end{frame}

\begin{frame}[fragile]{fancybox}
• Paket \pkg{fancybox} bietet vielseitige Möglichkeiten, Boxen um Text zu legen
• viele Parameter beliebig einstellbar
\•
\begin{LTXexample}
\doublebox{Doppelte Umrandung}
\ovalbox{ovale Umrandung}
\shadowbox{schattierte Box}
\end{LTXexample}
\end{frame}

\begin{frame}[fragile]{epic!}
• Paket |epic| bietet Erweiterungen zur \LaTeX-Umgebung |picture|
• Paket |eepic| bietet Erweiterungen zu den Erweiterungen
• z.\,B. Paket |ecltree| bietet Möglichkeit für Baumdiagramme:
\•
\begin{figure}
\begin{bundle}{Arno}
\chunk{
  \begin{bundle}{Ina\strut}
  \chunk{Laura}\end{bundle}
}
\chunk{\begin{bundle}{Stefan\strut}
	\chunk{Mainzer Studenten}\drawwith{\dottedline{3}}
\end{bundle}}
\chunk{\begin{bundle}{Jakob\strut}
  \chunk{Heidelberger Studenten}\drawwith{\dottedline{3}}
\end{bundle}}
\end{bundle}
\caption{Verbreitung von \TeX}
\end{figure}
\end{frame}

\begin{frame}[fragile]{more epic}
• Paket |bar| bietet Möglichkeit für Balkendiagramme:
\•
\begin{LTXexample}[preset=\vspace*{1cm},pos=b]
\begin{barenv}
\bar{69}{1}[Windows]
\bar{23}{4}[Linux]
\bar{14}{3}[Mac OS]
\bar{0}{1}[andere]
\end{barenv}
\end{LTXexample}


\end{frame}

\section{externe Graphiken}
\begin{frame}[fragile]{externe Graphiken}
• \LaTeX\ bietet Möglichkeit, extern erzeugte Bilder einzubinden
• je nach Treiber sind verschiedene Formate möglich
• u.\,a. jpeg, ps, pdf, gif, tiff, \dots
• unter Umständen Umformatierung nötig!
• \XeLaTeX\ bietet Unterstützung für meisten gebräuchlichen Formate
\•
\end{frame}

\begin{frame}{externe Graphiken}
\begin{block}{Vorteile externer Graphiken}
• freie Gestaltungsmöglichkeit
• Erzeugung in WYSIWYG-Graphikprogrammen
• Unabhängigkeit vom Dokument
• spezialisierte Programme für jeden Zweck
• Programme bieten guten Export nach \TeX\
\•
\end{block}
\pause
\begin{block}{Nachteile externer Graphiken}
• getrennt vom Dokument ⇒ Portabilität leidet
• Layout passt nicht zum Schriftbild
• Bildbeschriftungen müssen zur Brotschrift oder Matheschrift passen
• Treiberabhängigkeit
\•
\end{block}
\end{frame}

\begin{frame}[fragile]{Inkompatible Formate}
• falls eine Graphik benötigt wird, mit welcher der Treiber nicht umgehen kann:
\•
\begin{verbatim}
\Declaregraphiksrule{<Endung>}{<Typ>}{<Größe>}{<Befehl>}
\DeclareGraphicsRule{.jpg}{eps}{}{'jpg2eps #1'}
\end{verbatim}
• Paket \pkg{epstopdf} erleichtert Umgang mit PostScript-Dateien
• externe Umwandlung empfohlen:\\%
z.\,B. IrvanView, gimp \dots
\•
\end{frame}

\section[graphics/x]{Pakete graphics und graphicx}
\begin{frame}[fragile]{graphics}
• Grundbefehl: |\includegraphics[optionen]{datei}|
• Dateiendung muss nicht angegeben werden
• bei Arbeit mit pdf- \emph{oder} dvi-Ausgabe:
•[] Dateiendung besser weglassen
• keine absoluten Pfadangaben verwenden (Portabilität)
• nützlich, aber nicht ganz zuverlässig: |\graphicspath|
\•
\end{frame}

\begin{frame}[fragile]{graphicx}
• \pkg{graphicx} erweitert \pkg{graphics}
• |key=value|-Interface:
•[] |[scale = 0.5,angle=50]|
\•
|graphics: \scalebox{0.5}{\includegraphics{a}}|\\
|graphicx: \includegraphics[scale=.5]{a}|
\end{frame}

\begin{frame}[fragile]{Einbinden von Graphiken}
\begin{LTXexample}[pos=b,width=.9\textwidth]
\includegraphics[width=2cm]{unilogo.svg}
\includegraphics[width=.3\textwidth,angle=25]{unilogo.svg}
\end{LTXexample}
\end{frame}

\frame{\includegraphics{unilogo.svg}}

\begin{frame}[fragile]{Optionen für \texttt{includegraphics}}
|\includegraphics| kennt viele Optionen, z.\,B.
\\[2ex]
\begin{mydesc}
Schlüssel & Werte\\\midrule
|scale| & |0.8| \\
|width| & |.2\textwidth| \\
|height| & |2em| \\
|keepaspectratio| & |true| oder |false|\\
|angle|  & |50| \\
|bb| & |0 0 10 20| \\
|clip| & |true| oder |false|
\end{mydesc}
\\[2ex]
⇒ siehe Dokumentation zu |graphicx|
\end{frame}

\begin{frame}[fragile]{Bilder im Text}
• aus Textverarbeitungssystemen bekannt:\\%
Text, der Bild umfließt (nicht rechteckig, sondern der Form angepasst)
• typographisch fragwürdig – Abhebung des Bildes vom Text
• Umfließen stört Lesefluss erheblich
• \TeX\ kann prinzipiell keine Graphiken umfließen
• mit immensem Aufwand evtl. möglich
• Platzierung am Rand einfach möglich
•[⇒] Pakete \pkg{wrapfig}, \pkg{picins}, \pkg{picinpar}, \pkg{floatflt}
\•
\end{frame}

\begin{frame}[fragile]{wrapfig}
\begin{LTXexample}[preset=\fontsize{4}{5}\selectfont,pos=b]
\blindtext
\begin{wrapfigure}{l}[0.2\width]{0pt}
\includegraphics[width=1cm]{unilogo.svg}
\end{wrapfigure}
\blindtext[3]
\end{LTXexample}
\end{frame}

\begin{frame}[fragile]{wrapfig}
\pkg{wrapfig} bietet folgende Optionen:
|\begin{wrapfigure}[zeilen]{position}[überhang]{breite}|
\\[2ex]
\begin{mydesc}
zeilen & Anzahl schmaler Zeilen\\\midrule
position & Seite, auf die gesetzt wird.\\
|L,R| & links bzw. rechts, gleitend\\
|l,r| & links bzw. rechts, nicht gleitend\\
|i,o,I,O| & innen bzw. außen, analog\\\midrule
überhang & Einrückung in den Rand\\\midrule
breite & Breite der Abbildung
\end{mydesc}
\end{frame}

\begin{frame}[fragile]{wrapfig}
• |wrapfig| bietet auch |wraptable|
• |wrapfigure| und |wraptable| unterstützen |\caption| und |\label| :%{}
• Verwendung wie normale Gleitumgebungen
• |\width| für natürliche Breite
\•
\end{frame}

\begin{frame}[fragile]{picinpar}{Löcher im Absatz}
• \pkg{picinpar} ermöglicht Satz von beliebigem Inhalt innerhalb eines Absatzes
• Umgebung \pkg{window}: keine weitere Formatierung
• Umgebungen \pkg{figwindow} und \pkg{tabwindow} ermöglichen konsistenten Satz von Unterschriften
• \alert{Vorsicht: } evtl. Probleme mit Gleitumgebungen (|figure|, |table|)
\•
\end{frame}

\begin{frame}[fragile]{floatflt}
• \pkg{floatflt} kann Bilder am Anfang eines Absatzes setzen
• auf CTAN zu finden, nicht bei \TeX live dabei
\•
\begin{lstlisting}
\begin{floatingfigure}{60mm}
    \includegraphics[width=60mm]{B-7-Physik-Kloster.jpg} %
    \caption[Grundriß des zweiten Obergeschosses im Dominikanerkloster \newline Heidelberger Jahrbücher der Literatur 13,4 (1820)]
    {\label{Physik-Kloster} Grundriß des zweiten Obergeschosses im Dominikanerkloster}
\end{floatingfigure}
\end{lstlisting}
\end{frame}


\begin{frame}[fragile]{Teilbilder – subfig}
• Paket \pkg{subfig} definiert |\subfloat[]{}|
• optionales Argument ist Beschriftung
• obligatorisches Argument ist Inhalt
• inkompatibel mit \pkg{beamer} \dots
\•
\begin{lstlisting}
\begin{table}
\subfloat[Erste Tabelle]{
\begin{tabular}{ccc} a & b & c \end{tabular}
}
\subfloat[Zweite Tabelle]{
\begin{tabular}{ccc} a & b & c \end{tabular}
}
\caption{Zwei Tabellen}
\end{table}
\end{lstlisting}
\end{frame}

\begin{frame}[fragile]{Teilbilder – subfloat}
• definiert Umgebungen |subfigures| und |subtables|
• Objekte werden unabhängig platziert
• Legenden einzeln, aber zusammenhängend
\•
\end{frame}

\begin{frame}[fragile]{Teilbilder – subfloat}
\begin{LTXexample}[pos=b]
\begin{subfigures}
\begin{figure}
\fbox{Bild 1}\caption{Legende 1}
\end{figure}
\begin{figure}
\fbox{Bild 2}\caption{Legende 2}
\end{figure}
\end{subfigures}
\end{LTXexample}
\end{frame}

\begin{frame}[fragile]{Legenden seitlich setzen}
• Paket \pkg{sidecap} ermöglicht Satz von Legenden \emph{neben} Objekten
• Umgebungen |SCfigure| und |CStable|
• viele Optionen zur Formatierung der Legende
\•
\end{frame}

\begin{frame}[fragile]{sidecap}
\begin{lstlisting}
\begin{SCtable}[0.5][t]
\fbox{Eine Tabelle}
\caption{Eine Legende neben der tollen Tabelle}
\end{SCtable}
\end{lstlisting}
\end{frame}

\section{Zeichenpakete}
\newcommand\TikZ{Ti\emph{k}Z\xspace}
\begin{frame}[fragile]{Zeichenpakete}
• PS-Tricks und \TikZ{}\\%
(PostScript-Tricks, \TikZ \emph{i}st \emph{k}ein \emph{Zeichenprogramm}
• bieten immens große Möglichkeit, Graphiken zu erstellen
• viele spezielle Erweiterungspakete zu PS-Tricks
• z.\,B. Erstellen von Knotendiagrammen, Schaltplänen etc. 
\•
\end{frame}

\begin{frame}[fragile]{pst-circ}
\begin{lstlisting}
\begin{pspicture}(3,2)
\pnode(0,1){A}\pnode(3,1){B}
\pnode(3,0){C}\pnode(0,0){D}
\resistor(A)(B){$R$}
\capacitor(B)(C){$C$}
\LED(C)(D){$\mathcal D$}
\end{lstlisting}
\end{frame}

\begin{frame}[fragile]{pst-circ}
\begin{lstlisting}
\begin{pspicture}(3,3)
\pnode(0,3){A}\pnode(0,0){B}\pnode(3,3){C}\pnode(3,0){D}
\transformer(A)(B)(C)(D){$\mathcal{T}$}
\end{pspicture}\begin{pspicture}(3,3)
\logic[logicType=nand,logicShowNode,logicWidth=1,logicHeight=3,
logicNInput=6,logicChangeLR](2,1){NAND3}\end{pspicture}
\end{lstlisting}
\end{frame}

\begin{frame}[fragile]{TikZ}
\begin{lstlisting}
\begin{tikzpicture}
\node(tex) at (3,2) {\TeX};
\node(TeX-XeT) at (3,0) {\TeX-XeT};

\draw(tex) to (TeX-XeT);
\end{tikzpicture}
\end{lstlisting}
\end{frame}

\begin{frame}{TGC}
Für alles weitere: \\
Herbert Voss: PS Tricks\\
Michel Goossens, Sebastian Rahtz, Frank Mittelbach:\\
The \LaTeX\ Graphics Companion
\end{frame}

\end{document}