\makeatletter
\@ifundefined{draftmodeon}{
  \documentclass[german,t]{beamer}
}{
  \documentclass[german,t,draft]{beamer}
}
\makeatother
\mode<presentation>{
  \usetheme{Frankfurt}
  \useoutertheme{infolines}
  \usecolortheme[RGB={0,100,130}]{structure}
  \useinnertheme{rounded}
  \setbeamertemplate{navigation symbols}{} 
}

\newcommand\subtitlei[1]{\def\insertsubtitlei{#1}}
\newcommand\subtitleii[1]{\def\insertsubtitleii{#1}}

\setbeamertemplate{title page}
  {
    \vspace*{1cm}
    \begin{centering}
      {\usebeamerfont{title}\usebeamercolor[bg]{title}
      {\Large \inserttitle}
      \\[.3cm]
      \insertsubtitlei\\
      \large \color{black}\insertsubtitleii
      }
\vspace*{1cm}
    \vbox to 1cm {\hfill \includegraphics[width=.2\textwidth]{ctanlion}}
      \usebeamerfont{subtitle}\usebeamercolor[bg]{subtitle}
      {\color{black}\Large \insertsubtitle}\vfill
    {\hfill \insertdate \hfill}
    \end{centering}

    \vfill\vfill

    \hspace*{-.47cm}
    \usebeamertemplate*{footline}
    \vspace*{-1.15cm}
}

\setbeamercolor*{author in head/foot}{parent=palette tertiary}
\setbeamercolor*{title in head/foot}{parent=palette secondary}
\setbeamercolor*{date in head/foot}{parent=palette primary}

\setbeamercolor*{section in head/foot}{parent=palette tertiary}
\setbeamercolor*{subsection in head/foot}{parent=palette primary}

\defbeamertemplate*{headline}{infolines theme changed}
{
  \leavevmode%
  \hbox{%
  \begin{beamercolorbox}[wd=.3\paperwidth,ht=2.25ex,dp=1.5ex,center]{title in head/foot}%
    \LaTeX-Kurs 2010
  \end{beamercolorbox}%
  \begin{beamercolorbox}[wd=.4\paperwidth,ht=2.25ex,dp=1.2ex,right]{section in head/foot}%
     \hfill \coursenumber\ – \coursetitle \hfill\hfill
  \end{beamercolorbox}%
  \begin{beamercolorbox}[wd=.3\paperwidth,ht=2.25ex,dp=1.2ex,left]{subsection in head/foot}%
    \hspace*{2em}\insertsectionhead
  \end{beamercolorbox}}%
  \vskip0pt%
}

\defbeamertemplate*{footline}{infolines theme changed}
{
  \leavevmode%
  \hbox{%
  \begin{beamercolorbox}[wd=.3\paperwidth,ht=2.25ex,dp=1ex,center]{author in head/foot}%
      \insertshortauthor
  \end{beamercolorbox}%
  \begin{beamercolorbox}[wd=.4\paperwidth,ht=2.25ex,dp=1ex,center]{title in head/foot}%
    \insertshortdate{}
  \end{beamercolorbox}%
  \begin{beamercolorbox}[wd=.3\paperwidth,ht=2.25ex,dp=1ex,center]{author in head/foot}%
    \insertframenumber{} / \inserttotalframenumber
  \end{beamercolorbox}}%
  \vskip0pt%
}

\setbeamersize{text margin left=1em,text margin right=1em}

\def\extractnumber"#1-#2".{#1}
\def\extracttitle"#1-#2".{#2}
\def\coursenumber{\expandafter\extractnumber\jobname.}
\def\coursetitle{\expandafter\extracttitle\jobname.}

\usepackage{
  babel,
  dtklogos,
  moreverb,
  shortvrb,
  xltxtra,
  yfonts
}
\usepackage[final]{showexpl}

\hypersetup{
  colorlinks=true,
  breaklinks=true,
  linkcolor=blue,
  urlcolor=blue,
  citecolor=black, filecolor=black, menucolor=black,
  pdfauthor={Arno L. Trautmann},
}

\MakeShortVerb|

\author{Arno Trautmann}
\institute{Heidelberg}
\title{Einführung in das Textsatzsystem\\[2ex] \Huge \LaTeX}
\subtitlei{Vorlesungsreihe im Sommersemester 2010}
\subtitleii{\textfrak{universitatis:~studii~heydelbergensis:}}
\subtitle{\coursenumber\ – \coursetitle}

\graphicspath{{../Mediales/}}

\logo{}%\includegraphics[width=7.4em]{unilogo.svg}}

\defaultfontfeatures{Scale=MatchLowercase}
\setmonofont{DejaVu Sans Mono}

\newenvironment{mydesc}{\begin{tabular}{|>{\columncolor{lightgray}\color{blue}}rl|}\hline}{\\\hline\end{tabular}}

\def\pkg#1{\texttt{#1}}
\def\macro#1{|#1|}

\def\plainTeX{\textsf{plain\TeX}}

\catcode`\⇒=\active
\def⇒{\ensuremath{\Rightarrow}}
\catcode`\⇐=\active
\def⇐{\ensuremath{\Leftarrow}}
\catcode`\…13
\let…\dots

\newcommand\notiz[1]{}
\newcommand\einschub[1][]{\textcolor{red}{⇒}}
\renewcommand\checkmark{\color{green}\ding{51}}
\newcommand\cross{\color{red}\ding{53}}

\newcommand\pdf[2][]{\bgroup
  \setbeamercolor{background canvas}{bg=}%
  \includepdf[#1]{#2}%
    \egroup
}

\newenvironment{twoblock}[2]{
  \begin{columns}
  \begin{column}{.46\textwidth}
  \begin{block}{#1}
\def\nextblock{
  \end{block}
  \end{column}
\ %
  \begin{column}{.46\textwidth}
  \begin{block}{#2}
}
}
{
  \end{block}
  \end{column}
  \end{columns}
}

\AtBeginDocument{
  \lstset{%
    backgroundcolor=\color[rgb]{.9 .9 .9},
    basicstyle=\ttfamily\small,
    breakindent=0em,
    breaklines=true,
    commentstyle=,
    keywordstyle=,
    identifierstyle=,
    captionpos=b,
    numbers=none,
    frame=tlbr,%shadowbox,
    frameround=tttt,
    pos=r,
    rframe={single},
    explpreset={numbers=none}
  }
}
\makeatletter
\g@addto@macro\beamer@lastminutepatches{ % thanks to Ulrike for this!
  \frame[plain,t]{\titlepage}
  \frame{\centerline{\huge \color[RGB]{0,100,130}Inhalt}\tableofcontents}
}

%—— itemize-hack
\def\outside{o}
\def\inside{i}
\let\insideitemizei\outside
\let\insideitemizeii\outside
\def\altenditemize{
  \if\altlastitem 1%
    \let\altlastitem0%
  \else%
    \end{itemize}%
    \let\insideitemizei\outside%
  \fi%
}

\begingroup
  \lccode`\~=`\^^M%
\lowercase{%
  \endgroup
  \def\makeenteractive{%
    \catcode`\^^M=\active
    \let~\altenditemize
}%
}

\def\newitemi{%
  \ifx\insideitemizei\inside%
    \let\altlastitem1%
    \expandafter\item%
  \else%
    \begin{itemize}%
    \let\insideitemizei\inside%
    \let\altlastitem1%
    \makeenteractive%
    \expandafter\item%
  \fi
}

\def\newitemii{
  \ifx\insideitemizeii\inside
    \expandafter\item%
  \else
    \begin{itemize}
      \let\insideitemizeii\inside
      \expandafter\item%
  \fi
}

\def\makeitemi#1{%
  \expandafter\ifx\csname cc\string#1\endcsname\relax
    \add@special{#1}%
    \expandafter
    \xdef\csname cc\string#1\endcsname{\the\catcode`#1}%
    \begingroup
      \catcode`\~\active  \lccode`\~`#1%
      \lowercase{%
      \global\expandafter\let
         \csname ac\string#1\endcsname~%
      \expandafter\gdef\expandafter~\expandafter{\newitemi}}%
    \endgroup
    \global\catcode`#1\active
  \else
  \fi
}

\def\makeitemii#1{%
  \expandafter\ifx\csname cc\string#1\endcsname\relax
    \add@special{#1}%
    \expandafter
    \xdef\csname cc\string#1\endcsname{\the\catcode`#1}%
    \begingroup
      \catcode`\~\active  \lccode`\~`#1%
      \lowercase{%
      \global\expandafter\let
         \csname ac\string#1\endcsname~%
      \expandafter\gdef\expandafter~\expandafter{\newitemii}}%
    \endgroup
    \global\catcode`#1\active
  \else
  \fi
}

\def\add@special#1{%
  \rem@special{#1}%
  \expandafter\gdef\expandafter\dospecials\expandafter
{\dospecials \do #1}%
  \expandafter\gdef\expandafter\@sanitize\expandafter
{\@sanitize \@makeother #1}}
\def\rem@special#1{%
  \def\do##1{%
    \ifnum`#1=`##1 \else \noexpand\do\noexpand##1\fi}%
  \xdef\dospecials{\dospecials}%
  \begingroup
    \def\@makeother##1{%
      \ifnum`#1=`##1 \else \noexpand\@makeother\noexpand##1\fi}%
    \xdef\@sanitize{\@sanitize}%
  \endgroup}
\AtBeginDocument{
  \makeitemi{•}
}
%——beamer versaut hier irgendwas, daher muss itemize explizit beendet werden!
\def\•{\end{itemize}}
%—— itemize-hack
\makeatother

\DeleteShortVerb|

\usepackage{
tikz
}
\usetikzlibrary{trees}
\MakeShortVerb|

\begin{document}

\begin{frame}{Dokumentelemente}
• Schmutztitel
• Titelei
• Verzeichnisse
• Gliederung
• Kopf-/Fußzeilen
• Fußnoten, Randbemerkungen
• Formeln
• Abbildungen, Tabellen etc.
• Verweise
• Programmcode
• Anhang
• Bibliographie
• Indices
\•
\end{frame}

\section{Aufteilung – mehrere Dateien}
\begin{frame}{Aufteilung}
• Nachteil von \TeX: lange Dokumente werden unübersichtlich\pause
• Vorteil von \TeX: Teile des Dokumentes können in externe Dateien ausgelagert werden
• geschickte Aufteilung und Verwaltung eines Dokumentes:
• eine Hauptdatei, leeres Gerüst
• eine header-Datei (evtl. weitere Datei für spezielle Befehlsdefinitionen)
• Inhalte in einem Unterordner nach strukturierter Anordnung
• Abbildungen, sonstige Materialien in weiteren Unterordnern
\•
\end{frame}

\DeleteShortVerb|

\begin{frame}{Aufteilung}{Beispiel für einen missglückten TikZ-Baum …}
\begin{tikzpicture}[edge from parent fork right,
	every node/.style={fill=red!30,rounded corners},
	edge from parent/.style={red,-,thick,draw}]
\node {Main}
child {node [fill=blue!30] {Thesis}}
child {node [fill=blue!30] {header}}
child {node [fill=blue!30] {defs}}
child {node {bilder}
child {node [fill=blue!30]{pic1}}
child {node [fill=blue!30]{pic2}}
}
child {node {inhalte}
child {node [fill=blue!30]{chapter1}}
child {node [fill=blue!30]{chapter2}}
}
;
\end{tikzpicture}
\end{frame}

\MakeShortVerb|
\begin{frame}[fragile]{input, include, includeonly, excludeonly}
• |\input| und |\include| fügen externe Dateien am angegebenen Ort aus
• \TeX\ „springt“ aus dem aktuellen Dokument, liest woanders, und springt wieder zurück \pause
• \TeX-Version: |\input| liest den Code einfach ein, als gehöre er ins Hauptdokument
• \LaTeX-Version: |\include| erstellt eigene |.aux|-Datei (sinnvoll, wenn |.aux| benötigt, s.\,u.)
• |\includeonly{a.tex,b.tex}| in der Präambel lässt nur die angegebenen Dateien für |\include| zu
• |\excludeonly{b.tex,c.tex}| lässt die angegebenen Dateien für |\include| \emph{nicht} zu (benötigt Paket \pkg{excludeonly})
\•
\end{frame}

\begin{frame}[fragile]{root Dokument}
• nach Aufteilung muss immer das Hauptdokument kompiliert werden
•[⇒] ständiges Wechseln zwischen Dokumenten\pause
• gute Editoren nehmen die Arbeit ab:
• Definition von Hauptdokumenten möglich
• in \TeX works: Setzen von \verb*?% !TeX root = Hauptdokument.tex?
• Dokument im Oberverzeichnis: |% !TeX root = ../Hauptdokument.tex|
• Kompiliert automatisch das zugehörige Hauptdokument
\•
\end{frame}

\begin{frame}[fragile]{Hauptdokument}
\begin{block}{Ein Beispieldokument}
\begin{verbatim}
\input{header}

\includeonly{chapter1}
\excludeonly{anhang} % erfordert Paket excludeonly!

\begin{document}
\include{chapter1}
\include{chapter2}
...
\include{anhang}
\end{document}
\end{document}
\end{verbatim}
\end{block}
⇒ Nur |chapter1| wird hier gesetzt, |anhang| explizit nie.
\end{frame}

\section{Header}
\begin{frame}[fragile]{Header-Dokument}{Einstellungen}
• Satzspiegel
• Schriften (Brotschrift, Überschriften)
• Formatierung von Formeln
• …
• alles, was vor |\begin{document}| steht
\•
\end{frame}

\section{Titelei}
\begin{frame}[fragile]{Titelei}
• enthält alles bis zur ersten Inhaltsseite
• wird innerhalb des |header| definiert
• enthält Autor, Titel, etc.
• mit KOMA: Dokumentoption |titlepage=true/false| setzt eigene Seiten oder einen Titelkopf
• Umgebung |\begin{titlepage}| setzt eine frei gestaltbare Titelseite
• Befehl |\maketitle| setzt vordefinierte Titelei
• Angaben von |\title, \author, \extratitle| etc. nötig und möglich
\•
\end{frame}

\begin{frame}[fragile]
\begin{block}{Titeleibefehle im KOMA-Bundle}
\begin{verbatim}
\documentclass{scrbook}
\usepackage{xltxtra}
\begin{document}
\titlehead{{\Large Universität Schlauenheim}}
\subject{Diplomarbeit}
\title{Digitale Raumsimulation mit dem DSP\,56004}
\subtitle{Klein aber fein?}
\author{cand.\,stup. Uli Ungenau}
\date{30. Februar 2001}
\publishers{Betreut durch Prof.\,Dr.\,rer.\,stup. Naseweis}
\dedication{I love you to peaces, to thee I dedicate this thesis.}
\maketitle
\end{document}
\end{verbatim}
\end{block}
\end{frame}

\begin{frame}{abstract}
• Umgebung |abstract| setzt eine kurze Zusammenfassung des Dokumentes
• wird im Dokument gesetzt (nicht im header)
• mehrere Abstractices möglich (z.\,B. englisch\,/\,deutsch etc.)
\•
\end{frame}

\section{Verzeichnisse (TOC, LOF, LOT)}
\begin{frame}[fragile]{TOC, LOF, LOT}
• Verzeichnisse fassen strukturierte Elemente zusammen
• prinzipiell kann alles in ein eigenes Verzeichnis aufgenommen werden
• üblich: Inhaltsverzeichnis (|\tableofcontents|), Abbildungsverzeichnis (|\listoffigures|), Tabellenverzeichnis (|\listoftables|)
• Aufnamhme der Verzeichnisse ins Inhaltsverzeichnis: Dokumentenoption |toc=totoc|
• möglich: Codeverzeichnis, Beispielverzeichnis, …
\• 
\end{frame}

\section{Fußnoten, Randbemerkungen}
\begin{frame}[fragile]{Fußnoten, Randbemerkungen}
• zusätzlicher Text, der nicht ins Hauptdokument\,/\, in den Textfluss passt
• |\footnote{}|
• |\marginnote| (Paket |marginnote|)
• |\marginpar| ist gleitende Randnotiz
\•
\end{frame}

\section{Verweise}
\begin{frame}[fragile]{Verweise}
• Elemente können mittels |\label{}| bezeichnet werden
• mögliche Elemente sind Überschriften (sections etc.), |table|, |figure|, Formeln, …
• Referenzierung mit |\ref{}|
• Pakete liefern sehr vielfältige Referenzierunge:\\%
\pkg{fancyref}, \pkg{varioref}, \pkg{cleveref}\\%
• geschicktes Benennen:
• |\label{fig:Haus}| ⇒ Pakete können erkennen, dass es eine Abbildung ist
\•
\end{frame}

\section{Bibliographie}
\begin{frame}[fragile]{Bibliographie}
\def\exclam{!}  %% needed for some strange reason …
• Aussehen Bibliographie wird von der Dokumentenklasse bzw. Paketen verwaltet
• bestimmte Syntax zum Setzen der Bibilographie
• manuelles Erstellen (Sortieren etc.) im Dokument möglich, aber umständlich
• Einträge nicht wiederverwertbar\exclam\pause
•[⇒] Programm \BibTeX\ übernimmt Sortierung und Verwaltung der Einträge
\•
\end{frame}

\begin{frame}[fragile]{Bibliographie}
• Einträge liegen als Textdatei (.bib) in fest vorgegbener Syntax vor
• Referenz im Dokument mit |\cite{mittelbach2004}|
• Art der Referenz vielfältig einstellbar
• Zugriff auf große Menge an verfügbaren Referenzen
\•
\end{frame}

\begin{frame}{\BibTeX}
• Verwendung unintuitiv
• graphische Oberflächen erleichtern das Leben
• z.\,B. jabref, citavi, etc.
• direkte online-Suche z.\,B. bei \url{http://scholar.google.de/}
\•
\end{frame}

\begin{frame}[fragile]{Pakete}
• Gestaltung der Bibliographie mittels Paketen
• \pkg{natbib}, \pkg{jurabib}\pause, \pkg{biblatex} (vielfältigste Gestaltungsmöglichkeiten)
• Programm \pkg{biber} als Nachfolger von \BibTeX\ für \pkg{biblatex}
\•
\end{frame}

\begin{frame}[fragile]{Erstellung: |natbib|}
\begin{block}{im (Haupt-)Dokument}
\begin{verbatim}
\usepackage[optionen]{natbib}
\begin{document}
\bibliographystyle{plainnat} % oder andere ...
Text ... \cite{quelle} \citet{quelle} \citep{quelle}
\bibliography{Bibquellen}
\end{document}
\end{verbatim}
\end{block}

\begin{block}{in der .bib-Datei}
\begin{verbatim}
@Book{Danzer1972,
 author    ={Danzer, Klaus},
 title     ={{Robert W. Bunsen und Gustav R. Kirchhoff}},
 publisher ={B. G. Teubner},
 address   ={Leipzig},
 year      ={1972}}
\end{verbatim}
\end{block}
\end{frame}

\begin{frame}[fragile]{Erstellung: |biblatex|}
• sämtliche Layouteinstellungen sind über reine \LaTeX-Makros definiert
• andere Syntax als die „alten“ Pakete (verwenden \BibTeX-Code)
• Einstellungen über Paketoptionen (siehe \pkg{biblatex}-Dokumentation, |3 User guide|)
• sehr lange beta-Phase, daher nicht in \TeX live2009 enthalten, aber in 2010 verfügbar
\• 
\end{frame}

\section{Code}
\begin{frame}[fragile]{Setzen von Code}
• für kurze Sequenzen: |\verb~\befehl~|
• für längeer Sequzenzen: |\begin{verbatim} \befehle \end{verbatim}|
• beide bieten |*|-Version für Anzeigen von Leerzeichen: \verb$ $
• für Setzen von Programmcode: Paket |listings|
• für Setzen von \LaTeX-Beispielcode: Paket |showexpl|
\•
\end{frame}

\section{Index}
\begin{frame}[fragile]{Indexerstellung}
• Indexerstellung ist immens aufwändiges Unterfangen:
• sämtliche (sinnvollen!) Erscheinungen von Namen\,/\,Ereignissen\,/\,Sachthemen müssen registriert werden\\%
nicht jede Nennung eines Namens soll im Index erwähnt werden!
• sinnvolle Seitenangabe: |1,2–4,17|
\• 
\end{frame}

\begin{frame}[fragile]{Indexerstellung}
• dank logischer Struktur leichte Erstellung in \TeX:
• Definieren von Befehlen erleichtert die Eingabe:\\%
\verb|\kirchhoff| statt |Kirchhoff \index{Kirchhoff}|
• mit \LaTeX\ dreistufiger Prozess:
• im \LaTeX-Lauf wird Hilfsdatei erstellt
• Verarbeitung mittels Programm |makeindex| (Sortierung, Seitenangaben etc.)
• Einbettung im nächsten \LaTeX-Lauf
\•
\end{frame}

\begin{frame}[fragile]{Indexerstellung}{makeidx}
\begin{block}{im Dokument}
\begin{verbatim}
\usepackage{makeidx}
\makeindex{stichwoerter} % VOR \begin{document}!!
\index{Stichwort} %% IM Dokument!
\printindex{stichwoerter} % druckt das Verzeichnis hier
\end{verbatim}
\end{block}
\begin{block}{in der Kommandozeile}
Aufruf von |makeindex hauptdocument| im Ordner des Hauptdokumentes
\end{block}
\end{frame}

\begin{frame}[fragile]{Indexerstellung}{multind}
\pkg{multind} ermöglicht Erstellung mehrerer Indizes – Unterscheidung mit zusätzlichem Attribut:
\begin{block}{im Dokument}
\begin{verbatim}
\usepackage{multind}
\makeindex{stichwoerter}\makeindex{Personen}
\index{Stichwoerter}{Stichwort}\index{Personen}{Kirchhoff}
\printindex{stichwoerter}{Index der Stichwörter} \printindex{personen}{Personenverzeichnis}
\end{verbatim}
\end{block}
\begin{block}{in der Kommandozeile}
|makeindex personen|
|makeindex stichwoerter|
\end{block}
\end{frame}

\begin{frame}[fragile]{xeindex}
• Paket \pkg{xeindex} verwendet \XeTeX-Interna, um automatisch Indizes zu erstellen
• \pkg{xesearch} durchsucht dabei (mittels \XeTeX-Befehlen!) selbst das Dokument
• gefundene Einträge werden Indiziert\pause
•[$+$] extrem leichtes Erstellen von Indizes beliebiger Größe
•[$-$] Sinnhaftigkeit fragwürdig – nicht jede Nennung eines Begriffes sollte indiziert werden, sonst ist der Index nutzlos.
• Der Leser sollte nur die wichtigsten Einträge finden.
\•
\end{frame}

\begin{frame}[fragile]{xeindex}
• verwendet intern \pkg{makeidx}, daher sind |\makeindex|, |\printindex| und |\index| weiter verfügbar
• zu suchende Einträge:
\•
\begin{block}{IndexList}
|\IndexList⟨*⟩{⟨name⟩}{⟨list of entries⟩}|\\
\pkg{*} ⇒ case insensitive\\
\pkg{name} ⇒ beliebiger Name für die Liste (mehrere möglich)\\
\pkg{list of entries} ⇒ |katze, hund?, maus|\\
\pkg{hund?} ⇒ findet auch |hundehütte|
\end{block}
\end{frame}
\end{document}