\makeatletter
\@ifundefined{draftmodeon}{
  \documentclass[german,t]{beamer}
}{
  \documentclass[german,t,draft]{beamer}
}
\makeatother
\mode<presentation>{
  \usetheme{Frankfurt}
  \useoutertheme{infolines}
  \usecolortheme[RGB={0,100,130}]{structure}
  \useinnertheme{rounded}
  \setbeamertemplate{navigation symbols}{} 
}

\newcommand\subtitlei[1]{\def\insertsubtitlei{#1}}
\newcommand\subtitleii[1]{\def\insertsubtitleii{#1}}

\setbeamertemplate{title page}
  {
    \vspace*{1cm}
    \begin{centering}
      {\usebeamerfont{title}\usebeamercolor[bg]{title}
      {\Large \inserttitle}
      \\[.3cm]
      \insertsubtitlei\\
      \large \color{black}\insertsubtitleii
      }
\vspace*{1cm}
    \vbox to 1cm {\hfill \includegraphics[width=.2\textwidth]{ctanlion}}
      \usebeamerfont{subtitle}\usebeamercolor[bg]{subtitle}
      {\color{black}\Large \insertsubtitle}\vfill
    {\hfill \insertdate \hfill}
    \end{centering}

    \vfill\vfill

    \hspace*{-.47cm}
    \usebeamertemplate*{footline}
    \vspace*{-1.15cm}
}

\setbeamercolor*{author in head/foot}{parent=palette tertiary}
\setbeamercolor*{title in head/foot}{parent=palette secondary}
\setbeamercolor*{date in head/foot}{parent=palette primary}

\setbeamercolor*{section in head/foot}{parent=palette tertiary}
\setbeamercolor*{subsection in head/foot}{parent=palette primary}

\defbeamertemplate*{headline}{infolines theme changed}
{
  \leavevmode%
  \hbox{%
  \begin{beamercolorbox}[wd=.3\paperwidth,ht=2.25ex,dp=1.5ex,center]{title in head/foot}%
    \LaTeX-Kurs 2010
  \end{beamercolorbox}%
  \begin{beamercolorbox}[wd=.4\paperwidth,ht=2.25ex,dp=1.2ex,right]{section in head/foot}%
     \hfill \coursenumber\ – \coursetitle \hfill\hfill
  \end{beamercolorbox}%
  \begin{beamercolorbox}[wd=.3\paperwidth,ht=2.25ex,dp=1.2ex,left]{subsection in head/foot}%
    \hspace*{2em}\insertsectionhead
  \end{beamercolorbox}}%
  \vskip0pt%
}

\defbeamertemplate*{footline}{infolines theme changed}
{
  \leavevmode%
  \hbox{%
  \begin{beamercolorbox}[wd=.3\paperwidth,ht=2.25ex,dp=1ex,center]{author in head/foot}%
      \insertshortauthor
  \end{beamercolorbox}%
  \begin{beamercolorbox}[wd=.4\paperwidth,ht=2.25ex,dp=1ex,center]{title in head/foot}%
    \insertshortdate{}
  \end{beamercolorbox}%
  \begin{beamercolorbox}[wd=.3\paperwidth,ht=2.25ex,dp=1ex,center]{author in head/foot}%
    \insertframenumber{} / \inserttotalframenumber
  \end{beamercolorbox}}%
  \vskip0pt%
}

\setbeamersize{text margin left=1em,text margin right=1em}

\def\extractnumber"#1-#2".{#1}
\def\extracttitle"#1-#2".{#2}
\def\coursenumber{\expandafter\extractnumber\jobname.}
\def\coursetitle{\expandafter\extracttitle\jobname.}

\usepackage{
  babel,
  dtklogos,
  moreverb,
  shortvrb,
  xltxtra,
  yfonts
}
\usepackage[final]{showexpl}

\hypersetup{
  colorlinks=true,
  breaklinks=true,
  linkcolor=blue,
  urlcolor=blue,
  citecolor=black, filecolor=black, menucolor=black,
  pdfauthor={Arno L. Trautmann},
}

\MakeShortVerb|

\author{Arno Trautmann}
\institute{Heidelberg}
\title{Einführung in das Textsatzsystem\\[2ex] \Huge \LaTeX}
\subtitlei{Vorlesungsreihe im Sommersemester 2010}
\subtitleii{\textfrak{universitatis:~studii~heydelbergensis:}}
\subtitle{\coursenumber\ – \coursetitle}

\graphicspath{{../Mediales/}}

\logo{}%\includegraphics[width=7.4em]{unilogo.svg}}

\defaultfontfeatures{Scale=MatchLowercase}
\setmonofont{DejaVu Sans Mono}

\newenvironment{mydesc}{\begin{tabular}{|>{\columncolor{lightgray}\color{blue}}rl|}\hline}{\\\hline\end{tabular}}

\def\pkg#1{\texttt{#1}}
\def\macro#1{|#1|}

\def\plainTeX{\textsf{plain\TeX}}

\catcode`\⇒=\active
\def⇒{\ensuremath{\Rightarrow}}
\catcode`\⇐=\active
\def⇐{\ensuremath{\Leftarrow}}
\catcode`\…13
\let…\dots

\newcommand\notiz[1]{}
\newcommand\einschub[1][]{\textcolor{red}{⇒}}
\renewcommand\checkmark{\color{green}\ding{51}}
\newcommand\cross{\color{red}\ding{53}}

\newcommand\pdf[2][]{\bgroup
  \setbeamercolor{background canvas}{bg=}%
  \includepdf[#1]{#2}%
    \egroup
}

\newenvironment{twoblock}[2]{
  \begin{columns}
  \begin{column}{.46\textwidth}
  \begin{block}{#1}
\def\nextblock{
  \end{block}
  \end{column}
\ %
  \begin{column}{.46\textwidth}
  \begin{block}{#2}
}
}
{
  \end{block}
  \end{column}
  \end{columns}
}

\AtBeginDocument{
  \lstset{%
    backgroundcolor=\color[rgb]{.9 .9 .9},
    basicstyle=\ttfamily\small,
    breakindent=0em,
    breaklines=true,
    commentstyle=,
    keywordstyle=,
    identifierstyle=,
    captionpos=b,
    numbers=none,
    frame=tlbr,%shadowbox,
    frameround=tttt,
    pos=r,
    rframe={single},
    explpreset={numbers=none}
  }
}
\makeatletter
\g@addto@macro\beamer@lastminutepatches{ % thanks to Ulrike for this!
  \frame[plain,t]{\titlepage}
  \frame{\centerline{\huge \color[RGB]{0,100,130}Inhalt}\tableofcontents}
}

%—— itemize-hack
\def\outside{o}
\def\inside{i}
\let\insideitemizei\outside
\let\insideitemizeii\outside
\def\altenditemize{
  \if\altlastitem 1%
    \let\altlastitem0%
  \else%
    \end{itemize}%
    \let\insideitemizei\outside%
  \fi%
}

\begingroup
  \lccode`\~=`\^^M%
\lowercase{%
  \endgroup
  \def\makeenteractive{%
    \catcode`\^^M=\active
    \let~\altenditemize
}%
}

\def\newitemi{%
  \ifx\insideitemizei\inside%
    \let\altlastitem1%
    \expandafter\item%
  \else%
    \begin{itemize}%
    \let\insideitemizei\inside%
    \let\altlastitem1%
    \makeenteractive%
    \expandafter\item%
  \fi
}

\def\newitemii{
  \ifx\insideitemizeii\inside
    \expandafter\item%
  \else
    \begin{itemize}
      \let\insideitemizeii\inside
      \expandafter\item%
  \fi
}

\def\makeitemi#1{%
  \expandafter\ifx\csname cc\string#1\endcsname\relax
    \add@special{#1}%
    \expandafter
    \xdef\csname cc\string#1\endcsname{\the\catcode`#1}%
    \begingroup
      \catcode`\~\active  \lccode`\~`#1%
      \lowercase{%
      \global\expandafter\let
         \csname ac\string#1\endcsname~%
      \expandafter\gdef\expandafter~\expandafter{\newitemi}}%
    \endgroup
    \global\catcode`#1\active
  \else
  \fi
}

\def\makeitemii#1{%
  \expandafter\ifx\csname cc\string#1\endcsname\relax
    \add@special{#1}%
    \expandafter
    \xdef\csname cc\string#1\endcsname{\the\catcode`#1}%
    \begingroup
      \catcode`\~\active  \lccode`\~`#1%
      \lowercase{%
      \global\expandafter\let
         \csname ac\string#1\endcsname~%
      \expandafter\gdef\expandafter~\expandafter{\newitemii}}%
    \endgroup
    \global\catcode`#1\active
  \else
  \fi
}

\def\add@special#1{%
  \rem@special{#1}%
  \expandafter\gdef\expandafter\dospecials\expandafter
{\dospecials \do #1}%
  \expandafter\gdef\expandafter\@sanitize\expandafter
{\@sanitize \@makeother #1}}
\def\rem@special#1{%
  \def\do##1{%
    \ifnum`#1=`##1 \else \noexpand\do\noexpand##1\fi}%
  \xdef\dospecials{\dospecials}%
  \begingroup
    \def\@makeother##1{%
      \ifnum`#1=`##1 \else \noexpand\@makeother\noexpand##1\fi}%
    \xdef\@sanitize{\@sanitize}%
  \endgroup}
\AtBeginDocument{
  \makeitemi{•}
}
%——beamer versaut hier irgendwas, daher muss itemize explizit beendet werden!
\def\•{\end{itemize}}
%—— itemize-hack
\makeatother

\begin{document}

\begin{frame}[c]
\centerline{\large \textcolor{blue}{Teil I}}
\centerline{Briefe}
\end{frame}

\section{Briefe mit KOMA: scrlettr2}
\subsection{Grundlegende Bedienung}

\begin{frame}{scrlettr2: Grundidee}
• wie immer in \LaTeX: Trennung von Form und Inhalt
• alle formalen Elemente werden per Makro gesetzt
• Briefinhalt selbst wird direkt eingegeben
• Positionierung von Elementen mittels Befehlen anpassbar
\•
\end{frame}

\begin{frame}[fragile]{scrlettr2: neue Syntax}
\LaTeX\ kennt folgende „Dinge“:
• Befehle (|\texttt{}|)
• Umgebungen (|\begin{abstract} \end{abstract}|)
• Zähler (|\thepage|)
• Längen (|\pageheight=3cm|)
• Optionen (einfacher Wert oder Key-Value: |ngerman|, |top=2cm|)
\•
KOMA-Skript erweitert dies um:
• Elemente (|\setkomafont{title}{\fontspec{Arno Pro}}|) \pause
• \pkg{Variablen} (nur in der Briefklasse \pkg{scrlettr2})
\•
\end{frame}

\begin{frame}[fragile]{Variablen in scrlettr2}
• Setzen von Variablen mittels |\setkomavar{variable}{wert}|
• \alert{nicht zu verwechseln} mit |\KOMAoptions{}|
• mögliche Elemente: (kleine Auswahl)
\•

\begin{tabular}{>{\ttfamily}ll}\toprule
fromname & Absendername\\
fromaddress & Absenderadresse\\
fromemail & E-Mailadresse des Absenders\\\pause
myref & Feld für „Mein Zeichen“\\
specialmail & Versandart (Luftpost …)\\
backaddressseparator & Trennzeichen in der Rücksendeadresse\\\bottomrule
\end{tabular}\\[2ex]
⇒ siehe |texdoc scrguide|
\end{frame}

\begin{frame}[fragile]{Setzen von Variablen}
• Variablen verfügen über \alert{Inhalt}:\\%
|\setkomavar{fromname}{Mustermann}|
• aber auch über \alert{Bezeichnung}:\\%
|\setkomavar*{fromname}{Absender}| (statt „Von“)
• Kurzform:\\%
|\setkomavar{fromname}[Absender]{Mustermann}|
• Ausgabe:\\%
Absender: Mustermann
\•
\end{frame}

\begin{frame}[fragile]{Nutzen von Variablen}
• normalerweise werden Variablen nur gesetzt und von der Klasse genutzt
• Dokumentklasse kümmert sich dann um alles
• eigene Variablen können definiert werden
• Verwendung mittels |\usekomavar|
\•
|\newkomavar[Bezeichnung]{Name}|\\
|\usekomavar[anweisung]{fromname}| ⇒ |Mustermann|\\
|\usekomavar*[anweisung]{fromname}| ⇒ |Absender|\\
Dabei kann mit |anweisung| beliebiger Code ausgeführt werden\\
(z.\,B. |\MakeUppercase|)
\end{frame}

\begin{frame}[fragile]{Beispiel}
\begin{verbatim}
\documentclass{scrlttr2}

\setkomavar{fromname}{Arno Trautmann}
\setkomavar{fromaddress}{Rieslingweg 21\\55546 Hackenheim}

\begin{document}
\begin{letter}{Arno Trautmann\\Boxbergring 10\\ 69126 HD-Boxberg}
  \opening{Hallo,}
  dies ist mein erster Brief.
  \closing{Gruß}
\end{letter}

\end{document}
\end{verbatim}
\end{frame}

\begin{frame}[fragile]{Besonderheiten}
• \pkg{scrlettr2} unterscheidet sich in der Bedienung von anderen Klassen:
• es werden erst Briefe gesetzt, wenn |\begin{letter}| angegeben wird!
• nur sehr wenige Elemente werden dort angegeben, wo sie verwendet werden
•[⇒] sehr strikte Trennung von Form und Inhalt
\•
\end{frame}

\subsection{Letter Class Options}
\begin{frame}[fragile]{letter class option}
• Für standardisiertes Layout: immer gleiche Einstellungen
•[\alert{⇒}] copy \& paste?
•[\alert{⇒}] eigene |.cls| oder |.sty|-Datei?
•[\alert{⇒}] eigene |.tex|?
•[\alert{⇒}] Inkompatibilität, nicht gut portierbar\pause
•[⇒] eigenes Format für \pkg{scrlettr2}: |.lco|-Dateien
\•
\end{frame}

\begin{frame}[fragile]{letter class option}
• KOMA definiert bereits einige |.lco|-Dateien
• einfache Definition eigener |.lco|
• leichter Austausch\\%
⇒ normierte Geschäftsbriefe möglich
• nach Laden Anpassungen möglich\\%
⇒ dem Zweck angepasstes, schönes Format
• Verwendung: Als Klassenoption:\\%
|\documentclass[lconame]{scrlttr2}|\\%
oder im Dokument\\%
|\LoadLetterOption{lconame}|
\•
\end{frame}

\begin{frame}{letter class option}
\begin{table}
\begin{tabular}{ll}\toprule
DIN & gemäß DIN 676\\
DINmtext & Alternative für mehr Text auf der ersten Seite\\
KOMAold & Aussehen der alten \pkg{scrlettr}-Klasse\\
NipponEL & japanische Briefe\\
NipponEH & alternative japanische Briefe\\
SN & schweizer Briefe nach SN 010 130 (Anschrift rechts)\\
SNleft & dito, Anschrift links\\\bottomrule
\end{tabular}
\caption{einige Voreinstellungen für lco-Dateien}
\end{table}
Erstellen eigener |.lco|: siehe Dokumentation
\end{frame}

\subsection{Adressverwaltung}
\begin{frame}[fragile]{Adressverwaltung}
• Eingabe von Adressen nervig, zeitaufwändig und fehleranfällig
• Widerspricht dem Ansatz von \LaTeX
•[⇒] |.adr|-Dateien verwalten Adressen
• Einträge mit |\adrentry| bzw. |\addrentry|
• Verwenden mit |\input{adressen.adr}|
\•
\end{frame}

\begin{frame}[fragile]{adrentry vs. addrentry}
• |\adrentry| nimmt 8 Argumente
• |\addrentry| nimmt 9 Argumente
• letztes Argument definiert Befehl |\Kürzel|
\•
\begin{columns}
\begin{column}{.1\textwidth}
\begin{verbatim}
\adrentry{Name}
  {Vorname}
  {Adresse}
  {Telefon}
  {frei1}
  {frei2}
  {Kommentar}
  {Kürzel}
\end{verbatim}
\end{column}
\begin{column}{.1\textwidth}
\begin{verbatim}
\addrentry{Name}
  {Vorname}
  {Adresse}
  {Telefon}
  {frei1}
  {frei2}
  {frei3}
  {frei4}
  {Kürzel}
\end{verbatim}
\end{column}
\end{columns}
\end{frame}

\begin{frame}[fragile]{automatische Adressen}
• Verwendung im Brief:
\•
\begin{verbatim}
\begin{letter}{\Kürzel}
\opening{...}
\end{letter}
\end{verbatim}
⇒ Setzt automatisch die Adresse, die zum Eintrag |Kürzel| gehört\\
(z.\,B. |\ATRAUT|)
\end{frame}

\begin{frame}[fragile]{adrconv}
• damit die ganze Arbeit nicht nur im Brief steht:
• Paket \pkg{adrconv} kann Adressverzeichnisse oder Telefonlisten erstellen
• verwendet |\adrentry|, |\adrchar{E}| (wird von \pkg{scrlttr2} ignoriert)\\%
oder eigene Datenbank
•[⇒] |texdoc adrconv|
\•
\end{frame}

\subsection{Serienbriefe}
\begin{frame}[fragile]{Serienbriefe}
• „Missbrauch“ der Adressdatei:
• umdefinieren von |\ad(d)rentry| als Briefanfang
•[⇒] erstellt Brief an alle Einträge
\•
\pause
\begin{verbatim}
\renewcommand{\adrentry}[8]{%
  \begin{letter}{#2 #1\\#3}
    \opening{Sehr geehrte Geschäftsparnter,}
    die nächste Sitzung findet morgen statt!
    \closing{Hochachtungsvoll}
  \end{letter}
}
\input{geschäftspartner.adr}
\end{verbatim}
\end{frame}

\begin{frame}[c]
\centerline{\large \textcolor{blue}{Teil II}}
\centerline{Lebensläufe}
\end{frame}

\section{Lebensläufe}
\begin{frame}{Lebensläufe}
• professionelles Layout für Bewerbungen
• häufig standardisiert
• schlichtes Layout besser als überladenes
• Farben dezent einsetzen!
• Layout dem Zweck anpassen\\%
(Wohnheim, Universität, Bestattungsinstitut, \dots)
•[⇒] Beispieldokumente im Moodle\,/\,github
\•
\end{frame}

\subsection[europecv]{europecv – europäische Standards}
\begin{frame}[fragile]{europecv}
\begin{quotation}
As of 11 March 2002 the European Commission has defined a common format for curricula vitæ. This
class is an unofficial \LaTeX\ implementation of that format. Although primarily intended for users in the
European Union, the class can be used for any kind of curriculum vitæ.
\end{quotation}
• gute Dokumentation
• schlichtes, „klassisches“ Layout
• ausreichend formatierbar
\•
\end{frame}

\subsection{moderncv – farbig \fontspec{DejaVu Sans Mono}☺}
\begin{frame}[fragile]{moderncv}
• bietet ein modernes, lockeres Layout
• \alert{keine} offizielle Dokumentation
•[⇒] Beispieldokumente, README (|texdoc -s moderncv|)
•[⇒] |moderncv.cls| ansehen
\•
\end{frame}

\subsection{curve – Trennung in Rubrikdateien}
\begin{frame}[fragile]{curve}
• Grundidee: Trennung von Hauptdokument (skeleton) und Inhalt
• Inhalte (Rubriken) stehen in eigenen Datein
• unterschiedliche |\flavor| möglich: \\%
je nach Zweck angepasster Lebenslauf
• Dateinamen: |name.flavorname.rubrikname|:\\%
|sprachkenntnisse.mpi.tex|\\%
|programmierkenntnisse.mpi.tex|\\%
|grogrammierkenntnisse.dante.tex|
• Einbinden mittels |\makerubric{dateiname}|
\•
\end{frame}

\subsection{simplecv – kiss}
\begin{frame}[fragile]{simplecv}
• einfacher und schichter, schnell zu erzeugender Lebenslauf
• Setzen von Headern: |\leftheader{}\rightheader{}|
• |\title|, |\maketitle| wie gewohnt
• |\section| und |\subsection| zur Strukturierung
• Aufzählungen in der |topic|-Umgebung
• Bibliographie möglich!
• Dokumentation am einfachsten über Suchfunktion von \pkg{texdoc}\\ (u.\,U. selbst kompilieren)
\• 
\end{frame}

\begin{frame}[c]
\centerline{\large \textcolor{blue}{Teil III}}
\centerline{\TeX works}
\end{frame}

\section{\TeX works, sync\TeX }
\begin{frame}{\TeX works}{Historische Folie}
• Editor \TeX works bietet Bezug zwischen Quellcode und pdf
• inspiriert von \TeX shop (Mac)
• Verwendung von |synctex|\\%
(nur von sehr wenigen Programmen unterstützt \fontspec{DejaVu Sans} ☹)
• freie Software
• Binärdateien für Windows und Mac
• auf Linux-Systemen sehr leicht zu kompilieren
• pre-$\alpha$-Status!
\•
\end{frame}

\newcommand{\comb}[2]{\texttt{#1} – \texttt{#2}}


\begin{frame}{\TeX works – Tips zur Bedienung}{Editor}
• Suchen: \comb{strg}s
• nochmals Suchen: \comb{strg}g
• Ersetzen: \comb{strg}r
• Kompilieren: \comb{strg}t
• (Aus)Kommentieren: \comb{strg}d, \comb{strg – shift}d
• Einrücken: \texttt{Tab}
• Remove aux: \texttt{alt}3$\times$\texttt{Pfeil hoch}\texttt{Enter}
\• 
\end{frame}


\begin{frame}{\TeX works – Tips zur Bedienung}{Einstellungen}
• Einstellen\,/\,Ändern der Maschinen
\•
\end{frame}
\end{document}