\documentclass[
%	solution,
%	draft,
	blatt=10,
	ausgabe=18.\,06.\,2010,
	rückgabe=25.\,06.\,2010
]{lcourse-hd}

\header{Ein bekannter Verlag sucht für hochwertige Veröffentlichungen einen Kenner des Satzsystemes \LaTeX. Da Sie gerade auf der Suche nach einem neuen Arbeitsplatz sind, sollten Sie diese Chance ergreifen und sich für die Stelle melden.}
\begin{document}

\begin{exercise}[
  name=Motivationsschreiben,
  punkte=5,
  abgabe = Quelltext per Mail{,} das fertige Dokument als Ausdruck.
]{brief}
Kontaktieren Sie zunächst den Verlag mit einem Motivationsschreiben, also einem Brief an den Empfänger, der kurz begründet, warum Sie hervorragend geeignet sind.

Da es um die Suche nach einem \LaTeX-Spezialisten geht, sollte der Brief formal für sich sprechen. Auf die inhaltichen Sachen müssen Sie also keinerlei Mühen legen, aber das Aussehen sollte Ihren erfahrenen Umgang mit \LaTeX\ zeigen.

Erstellen Sie also einen Brief, der Absender, Empfänger, Datum, Titel, Anrede etc. enthält. Außerdem soll eine Anlage (s.\,u.) angezeigt werden. Verwenden Sie die Klasse |scrlettr2| aus dem KOMA-Bundle. Sollte Ihnen eine andere Briefklasse lieber sein, können Sie diese verwenden, wenn Sie deren Vorteile ggü. |scrlettr2| ausführen.
\end{exercise}

\begin{exercise}[
  name=Lebenslauf,
  punkte=5,
  abgabe = Quelltext per Mail{,} das fertige Dokument als Ausdruck.
]{lebenslauf}
Neben dem Motivationsschreiben gehört zu einer Bewerbung selbstverständlich ein Lebenslauf. Fertigen Sie einen solchen an unter Verwendung einer der in der Vorlesung besprochenen Klassen. Hier gilt ebenfalls: Das Dokument spricht für sich selbst. Sie müssen also dem Inhalt keine Beachtung schenken und können beliebig (un-)sinnige Sachen schreiben.
\end{exercise}

\begin{expertexercise}[
  name=Ich bin ich.,
  abgabe= Quelltext per Mail.]
{selfcontained}
Eine besondere Freude für den Programmierer sind Programme, die ihren eigenen Code ausgeben. Beeindrucken Sie also Ihren zukünftigen Arbeitgeber, indem Sie ein solches Dokument erstellen. Die \LaTeX-Datei soll also im pdf genau den Code ausgeben, mit dem sie selbst geschrieben ist. Es sollen keine zusätzlichen Befehle im Quellcode stehen, die nicht im pdf als Ausgabe erscheinen!

Falls Ihnen das nicht gelingt, suchen Sie im Internet danach. Sollten Sie ein Beispiel für einen solchen Code finden, hängen Sie ihn an Ihre Bewerbung an unter Angabe der Quelle (!). Damit zeigen Sie, dass Sie Probleme pragmatisch lösen können – wenn es schon jemand gemacht hat, muss man das Rad nicht neu erfinden. Versuchen Sie dennoch, den Code weitestgehend zu verstehen und erläutern Sie ihn gegebenenfalls.

\end{expertexercise}

\end{document}