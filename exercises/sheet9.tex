\documentclass[
%	solution,
	draft,
	blatt=9,
	ausgabe=12.\,06.\,2010,
	rückgabe=18.\,06.\,2010
]{lcourse-hd}

\header{Wie in der Vorlesung besprochen, soll ein umfangreiches Dokument korrigiert und erweitert werden. Die Abgabe des Quellcodes soll nur \emph{einmal} erfolgen, wenn alle Aufgaben zusammen bearbeitet sind. Also diesmal bitte nicht den Code für Aufgabe 1 und Aufgabe 2 getrennt schicken!}
\begin{document}
\begin{exercise}[
  name={Aufsetzen und Korrigieren},
  punkte=4,
  abgabe = \begin{itemize}
\item Handschriftliche Skizze der Ordnerstruktur.
\item {Falls Aufgabe 2 nicht bearbeitet wird: Den gesamten Quellcode als Email (mehrere Dateien). Bitte überlegen Sie sich, falls möglich, eine Archivierung (zip o.\,ä.), um den Transfer zu erleichtern. Falls Aufgabe 2 bearbeitet wird, keine Abgabe des Quellcodes für Aufgabe 1.}
\end{itemize}
]{korrektur}
Im Moodle finden sich fünf Dateien, die zu einem Dokument gehören. Laden Sie diese auf Ihren Rechner, ordnen Sie sie in einer sinnvollen Ordnerstruktur und kompilieren Sie das Dokument.

Beachten Sie, dass ein Dokument |chapter1.tex| im Unterordner |inhalte| eingebunden wird mittels |\include{inhalte/chapter1}|. Dies muss ggf. im Hauptdokument geändert werden!

Korrigieren Sie nun alles, was in diesem Dokument falsch gemacht wurde. Dabei zählt hier als „falsch“ auch mangelnde Orthographie, undurchdachte Namensgebung von Labels, ungeschickte Platzierungen von Inhalten etc.

Falls Sie den \TeX works-Editor verwenden, testen Sie die Fähigkeit von |TeX root|. (Vorlesungsfolie 7)

Geben Sie handschriftlich eine Skizze Ihrer Ordnerstruktur (vgl. Tafelbild), damit diese für die Korrektur nachvollziehbar ist.
\end{exercise}

\begin{exercise}[
  name={Erweiterung},
  punkte=6,
  abgabe = \begin{itemize}\item {Den gesamten Quellcode als Email (mehrere Dateien). Bitte überlegen Sie sich, falls möglich, eine Archivierung (zip o.\,ä.), um den Transfer zu erleichtern.}
\item Fertiges Dokument (pdf) per Mail.
\item {Ausdruck von Inhaltsverzeichnis, erster Textseite sowie Index bzw. Bibliographie. Also drei repräsentative Seiten Ihres Dokumentes.}
\item Handschriftliche Begründung für die Indexwahl.
\end{itemize}
]{erweiterung}
Wesentliche Elemente des Dokumentes fehlen noch. Ergänzen Sie also:
• Die gesamte Titelei (Verwendung der Voreinstellungen der KOMA-Klasse, keine eigene Anpassung des Layouts – |\maketitle|). Es sollten gesetzt werden: 
Schmutztitel, Titelseite, Inhaltsverzeichnis. Elemente wie Titel, Autor etc. können von Ihnen frei benannt werden.
• Erstellen Sie einen Anhang (Umgebung |abstract|), der zwei der folgenden drei Punkte enthält:
•[1] Ein Abbildungs- \emph{und} Tabellenverzeichnis.
•[2] Einen Index. Überlegen Sie sich dabei, ob Sie mit |makeidx| oder mit |xeindex| arbeiten wollen. |multind| ist ebenfalls möglich, dann müssen aber mindestens zwei Indizes erstellt werden. Begründen Sie handschriftlich ihre Wahl.
•[3] Eine Bibliographie. Wählen Sie dabei einen Bibliographiestil und ein Backend (|bibtex|, |bibtex8/u|, |biber|) Ihrer Wahl.

Um diese Elemente (bzw. die, die Sie verwendet haben,) mit Inhalt zu füllen:
• Indizieren Sie mindestens drei Wörter, die jeweils mindestens auf drei verschiedenen Seiten vorkommen.
• Fügen Sie im Dokument mindestens drei Referenzen ein, die in der Bibliograhpie genannt werden.

\subexercise[+1]{Kreativität}
Fügen Sie weitere Elemente Ihrer Wahl zum Dokument hinzu. Sehen Sie dies als persönliche Übung; für sinnvolle Einträge kann es den Zusatzpunkt geben.

\end{exercise}
\end{document}