\documentclass[
%	solution,
%	draft,
	blatt=12,
	ausgabe=02.\,07.\,2010,
	rückgabe=09.\,07.\,2010
]{lcourse-hd}

\header{\textbf{Wahlmöglichkeit:} Es müssen 2 der 3 Aufgaben dieses Blattes bearbeitet werden. Die Höchspunktzahl des gesamten Blattes beträgt 10 Punkte. Es gibt keine Zusatzpunkte für Bearbeitung aller Aufgaben. Teilweise Bearbeitungen werden nicht zusammengezählt.}

\begin{document}

\begin{exercise}[
  name=xparse,
  punkte=5,
  abgabe = Quelltext per Mail und ausgedruckt. Anmerkungen zu optionalen Argumenten hand- oder maschinenschriftlich.
]{xparse}
Wie in der Vorlesung vorgestellt, bietet das |xparse|-Paket eine sehr nutzerfreundliche Möglichkeit, komplexe Makros mit beliebigen Argumenten zu erstellen. Verwenden Sie den |\DeclareDocumentCommand|-Befehl, um ein Makro zu definieren, das folgendermaßen verwendet wird:

\begin{verbatim}
\mycommand[eins]<zwei>[drei]{vier}
\end{verbatim}
Dabei sollen erstes und drittes Argument optional sein, das zweite ebenfalls optional, aber in spitzen Klammern gegeben werden, und das vierte Argument soll mandatorisch sein. Das Makro soll mindestens alle vier Argumente wieder ausgeben. (|\textbf{#1}, \textit{#2} und \textsubscript{#3} sowie #4| oder ähnlich, Sie dürfen kreativ sein.)

Testen Sie in einem kurzen Dokument, was passiert, wenn man die Argumente eins bis drei angibt oder nicht. Ermitteln Sie mithilfe der |xparse|-Dokumentation, wie man testen kann, ob ein optionales Argument angegeben wurde oder nicht.

\subexercise[\normalfont keine Punkte]{optional: optionale Argumente prüfen} Falls noch Zeit bleibt: geben Sie mit Ihrem Makro eine Meldung aus (entweder im Text oder mithilfe der Fehlermeldungen von \LaTeX3), falls ein Parameter nicht angegeben wurde (und welcher).
\end{exercise}

\begin{exercise}[
  name=Typographie auf Anfrage – Teil 2,
  punkte=5,
  abgabe = Quelltext per Mail und ausgedruckt.\newline Fertiges Dokument (in der |classicthesis|-Version) als pdf per Mail.
]{classicorugly}
Es kann vorkommen, dass Korrektoren eine Arbeit mit 1,5-fachem Zeilenabstand fordern – entgegen aller typographischen Regeln zur guten Lesbarkeit. 
Erstellen Sie ein Dokument, das wahlweise \emph{entweder} auf der |classicthesis|-Vorlage basiert \emph{oder} die \LaTeX-Standardklasse |article| (\emph{nicht} |scrartcl|!) und 1,5-fachen Zeilenabstand verwendet.

\subexercise{Grundaufbau}
Da die Abfrage bereits vor der Dokumentenklasse gemacht werden muss, empfehlen sich die \TeX-Primitiva für |if|-Abfragen.

Verwenden Sie |\iftrue \documentclass... \else \documentclass... \fi|, um im |if|-Fall das Paket |classicthesis| (mit Klasse |scrartcl|!) zu verwenden, im |else|-Fall aber die |article|-Klasse zu laden. Durch Ändern von |\iftrue| in |\iffalse| kann umgeschaltet werden.

\subexercise{Allgemeine Einstellungen}
Bei |classicthesis| müssen keine Einstellungen vorgenommen werden. Für den 1,5-fachen Zeilenabstand hingegen sollten Sie das Paket |setspace| verwenden. Laden Sie es direkt in der |\iftrue|-Abfrage am Dokumentenanfang. Wie man |setspace| verwendet, finden Sie im Internet oder direkt in der |.sty|-Datei. Sie finden diese in Ihrem \TeX live-Ordner \footnote{TeXlive-Ordner/texmf-dist/tex/latex/setspace/setspace.sty} oder auf CTAN.
\footnote{ \url{http://www.ctan.org/tex-archive/macros/latex/contrib/setspace}} Die Makronamen (Zeile 322 bis 338) sind selbsterklärend.

\subexercise[\normalfont keine Punkte]{optional: seitenweise Einstellungen}
Falls noch Zeit bleibt, kann diese aufwändigere Übung noch bearbeitet werden: Bestimmte Seiten (Inhaltsverzeichnisse, Danksagung u.\,a.) sollen nicht 1,5-zeilig, sondern normal gesetzt werden. Damit man für diese Elemente nicht immer umschalten muss, sollte dies ebenfalls mit einer |\if|-Abfrage gesetzt werden. Die kürzeste Möglichkeit, dies umzusetzen, ist, sich einen neuen Befehl in folgender Art zu definieren:
\begin{verbatim}
\let\setforcorrector\iftrue  %% Satz für Korrektor
% \let\setforcorrector\iffalse %% Satz für Leute, die die Arbeit lesen wollen
\end{verbatim}
Erklären Sie (hand- oder maschinenschriftlich) diesen Code. Wie kann mit dieser Definition geschickt zwischen einfachem und 1,5-fachem Abstand gewechselt werden? Bedenken Sie, dass |\setforcorrector| bereits beim Laden der Dokumentenklasse angewendet werden kann. Schreiben Sie Ihr Dokument so, dass das Inhaltsverzeichnis und eine Danksagung normal, der restliche Text aber 1,5-zeilig gesetzt wird.
\end{exercise}


\begin{exercise}[
  name=Fehlersuche,
  punkte=5,
  abgabe = Korrigierter Quelltext per Mail und ausgedruckt. Falls nötig{,} Kommentare zu nicht lösbaren Fehlern.
]{error}
Das folgende Dokument (auch im Moodle/github verfügbar) weist einige Fehler auf. Finden und korrigieren Sie die Fehler. Falls das nicht möglich ist, isolieren Sie die fehlerhaften Stellen und kommentieren Sie sie aus. Es gibt auch Fehler, die im Dokument erst sichtbar werden und nicht während des Kompilierens auftreten! (Ein fehlendes Bild zählt hier natürlich \emph{nicht} als Fehler!)
\begin{verbatim} 
\documentclass[ngerman]{article}

\usepackage{blindtext,babel,graphicx,tikz,xparse}

\ExplSyntaxNamesOn
\NewDocumentEnvironment{morsepotential}{O{}}{
  \begin{tikzpicture}[domain=1:3]
    \draw plot[smooth] (\x,{(1*(1-exp(-0.7*(\x*2-1))))^2}){};}{
  \end{tikzpicture}}
\ExplSyntaxNamesOff

\begin{document}
\blindtext
\begin{figure}
\includegraphics{\textwidth}{testpic}
\label{fig:testbild}
\caption{Dies ist ein tolles Bild.}
\end{figure}
\blintext
\begin{morsepotential}
Morse hatte Potential!
\end{morsepotential}
Das Bild \ref{fig:testbild} ist sehr aussagekräftig in diesem Zusammenhang.
\blindtext
\end{document}
\end{verbatim}
\subexercise[\normalfont keine Punkte]{optional: Lösung}
Falls noch Zeit bleibt: Lesen Sie die Dokumentationen zu |expl3| und zum |tikz|-Paket, um die Probleme beim Verwenden von |:| zu verstehen. Entwickeln Sie dann eine Lösung für das Problem und geben Sie an, wie der Code lauffähig gemacht wird.

Alternativ können Sie die Nachricht von Joseph Wright zu genau diesem Problem im Archiv der |texhax|-Liste lesen. (\url{http://tug.org/pipermail/texhax/2010-June/015259.html})
\end{exercise}

\end{document}