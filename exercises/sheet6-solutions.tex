%% solutions by Jakob Herpich

\startsolution{kalorien}

\subsolution{Uhrzeit}
Es war ein Makro wie das Folgende gefordert. Die Formatierung war dabei beliebig (z.\,B. 19\textsuperscript{32} oder 19:30\,Uhr).
\begin{lcode}
\renewcommand\zeit[2]{#1\textsuperscript{#2}}
\end{lcode}
Da dieses Makro in Tabellen verwendet werden soll, habe ich auf den Zusatz \emph{Uhr} oder \emph h verzichtet. \verb+\zeit{14}{34}+ ergibt dann \zeit{14}{34}.

\subsolution{Tabelle}
In dieser Aufgabe war eine Tabell wie die folgende gefordert:
\begin{center}
	\captionof{table}{Pfingstmästung}
	\begin{tabular}{ll}
		\toprule
		{\textbf{Uhrzeit}}	&	{\textbf{Speise}}	\\
		\midrule 
		\zeit{10}{00}				&	Müsli							\\
		\zeit{12}{00}				&	Schweinshaxe			\\
		\zeit{18}{00}				&	5 Steaks					\\
		\bottomrule
	\end{tabular}
\end{center}
Der Quellcode sieht so aus:
\begin{lcode}
\begin{table}
	\centering
	\caption{Pfingstmästung}
	\begin{tabular}{ll}
		\toprule
		{\textbf{Uhrzeit}}	&	{\textbf{Speise}}	\\
		\midrule 
		\zeit{10}{00}				&	Müsli							\\
		\zeit{12}{00}				&	Schweinshaxe			\\
		\zeit{18}{00}				&	5 Steaks					\\
		\bottomrule
	\end{tabular}
\end{table}
\end{lcode}
\end{minipage}

\begin{minipage}{\textwidth}
\subsolution{Kalorien}
Nun sollte man die Tabelle um eine Kalorienangabe erweitern:
\begin{lcode}
	\begin{table}
		\center
		\caption{Pfingstmästung}
		\label{tab:pfingstmaestung}
		\begin{tabular}{clS[tabnumalign=right,tabformat=2.3]}
			\toprule
			{\textbf{Uhrzeit}}  &
			{\textbf{Speise}}   &
			\textbf{Brennwert} [\kilo\joule]	\\
			\midrule 
			\zeit{10}{00}		&	Müsli				&	1.345	\\
			\zeit{12}{00}		&	Schweinshaxe		&	23.44	\\
			\zeit{18}{00}		&	5 Steaks	&	3.445	\\
			\bottomrule
		\end{tabular}
	\end{table}
\end{lcode}

Die Ausgabe sieht dann so aus:
  \begin{center}
    \captionof{table}{Pfingstmästung}
    \label{tab:pfingstmaestung}
    \begin{tabular}{clS[tabnumalign=right,tabformat=2.3]}
      \toprule
      {\textbf{Uhrzeit}}  &
      {\textbf{Speise}}   &
      \textbf{Brennwert} [\kilo\joule] \\
      \midrule 
      \zeit{10}{00}   & Müsli       & 1.345  \\
      \zeit{12}{00}   & Schweinshaxe    & 23.44  \\
      \zeit{18}{00}   & 5 Steaks  & 3.445 \\
      \bottomrule
    \end{tabular}
  \end{center}
\end{minipage}

\begin{minipage}\textwidth
\subsolution{Zusatzinformationen}
In diesem Teil traten die meisten Probleme auf. Zuerst zu den Fehlerangaben in Tabellen.
Hier zeigt \verb+siunitx+ ein merkwürdiges Verhalten, wenn der Wert ganzzahlig ist:
\begin{lcode}
\begin{table}
	\centering
	\begin{tabular}{S[tabnumalign=right,tabformat=3.5e1,expproduct=cdot]}
		\tobrule
		\textbf{Brennwert} [\kilo\joule] \\
		\midrule
		134(4)	\\
		134.34(5)	\\
		3.34(4)e3	\\
		\bottomrule
	\end{tabular}
\end{table}
\end{lcode}
\begin{center}
	\begin{tabular}{S[tabnumalign=right,tabformat=3.5e1,expproduct=cdot]}
		\toprule
		\textbf{Brennwert} [\kilo\joule] \\
		\midrule
		134(4)	\\
		134.34(5)	\\
		3.34(4)e3	\\
		\bottomrule
	\end{tabular}
\end{center}
\end{minipage}

\begin{minipage}\textwidth
Die \verb+table+-Umgebung "fängt" Fußnoten ein. D.\,h., dass sie nur innerhalb der \verb+table+-Umgebung existieren und daher nicht richtig gesetzt werden können.
Es gibt zahlreiche Möglichkeiten, dieses Problem zu vermeiden. Eine ist, die Tabelle in einer Minipage zu setzen. Dabei werden die Fußnoten mit einer internen Nummerierung direkt unter die Tabelle gesetzt.

Alternativ kann man die Befehle \verb+\footnotemark+ und \verb+\footnotetext+ verwenden, bei denen man allerdings die Nummerierung manuell vornehmen muss.

Aufgrund der inkonsistenten Nummerierung dieser beiden Methoden, sind sie recht ungegeignet.
\begin{lcode}
\begin{minipage}\textwidth
\captionof{table}{Überschrift für Tabelle in einer Minipage}
\begin{center}
\begin{tabular}{l}
\toprule
Tabellenkopf	\\
\midrule
Inhalt mit Fußnote\footnote{Fußnote in einer Tabelle}	\\
Mehr Inhalt\footnote{Noch eine Fußnote}	\\
Zeile mit \verb+\footnotemark+\footnotemark	\\
\bottomrule
\end{tabular}
\end{center}
\footnotetext[3]{Fußnote mit \verb+\footnotetext+}
\end{minipage}
\end{lcode}
\begin{minipage}\textwidth
\captionof{table}{Überschrift für Tabelle in einer Minipage}
\begin{center}
\begin{tabular}{l}
\toprule
Tabellenkopf  \\
\midrule
Inhalt mit Fußnote\footnote{Fußnote in einer Tabelle} \\
Mehr Inhalt\footnote{Noch eine Fußnote}	\\
Zeile mit \verb+\footnotemark+\footnotemark	\\
\bottomrule
\end{tabular}
\end{center}
\footnotetext[3]{Fußnote mit \texttt{\textbackslash footnotetext}}
\end{minipage}
\end{minipage}

\begin{savenotes}
\begin{minipage}\textwidth
Eleganter ist es die \verb+savenotes+-Umbebung aus dem Paket \verb+footnote+ zu verwenden. Diese Umgebung "speichert" die Fußnoten und macht sie dann global verfügbar.
\begin{lcode}
\begin{savenotes}
\captionof{table}{Tabelle mit \verb+savenotes+-Umbebung}
\begin{center}
\begin{tabular}{l}
\toprule
Tabellenkopf  \\
\midrule
Inhalt mit Fußnote\footnote{Fußnote in einer Tabelle} \\
Mehr Inhalt\footnote{Noch eine Fußnote}	\\
\bottomrule
\end{tabular}
\end{center}
\end{savenotes}
\end{lcode}

\captionof{table}{Tabelle mit Umbebung}
\begin{center}
\begin{tabular}{l}
\toprule
Tabellenkopf  \\
\midrule
Inhalt mit Fußnote\footnote{Fußnote in einer Tabelle} \\
Mehr Inhalt\footnote{Noch eine Fußnote}	\\
\bottomrule
\end{tabular}
\end{center}
\end{minipage}
\end{savenotes}
\begin{minipage}\textwidth



\stopsolution
