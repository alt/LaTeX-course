\documentclass[
%	solution,
	draft,
	blatt=11,
	ausgabe=25.\,06.\,2010,
	rückgabe=02.\,07.\,2010
]{lcourse-hd}

\begin{document}

\begin{exercise}[
  name=Typographie auf Anfrage,
  punkte=5,
  abgabe = Quelltext per Mail{,} das fertige Dokument als Ausdruck.
]{iftypo}
Manchmal kann es nötig sein, ein inhaltlich gleiches Dokument mit verschiedenen \TeX-Maschinen zu erstellen. Angenommen, Sie haben zwei verschiedene Vorgaben für ein langes Dokument:

• Der Verlag besteht auf der hausinternen Schrift mit bestimmten, festgelegten Schriftfeatures,
• das Dekanat Ihrer Fakultät hingegen setzt Mikrotypographie voraus, damit der Korrektor die Arbeit möglichst schnell lesen kann und mehr Zeit für sonstige Verwaltungsaufgaben hat.

Machen Sie es beiden recht, indem Sie ein Dokument erstellen, das zunächst aus folgendem Grundgerüst besteht:

• Dokumentenklasse: scrreprt
• eine Titelseite mit Autor, Titel und Datum
• Inhaltsverzeichnis
• zwei Abschnitte mit ein paar Zeilen Text

Um die Anforderungen zu erfüllen, müssen Sie eine Fallunterscheidung (|if|-Abfrage) machen. Der folgende Pseudokode soll das Vorgehen verdeutlichen:
\begin{verbatim}
neueVariable falls_pdf
setze falls_pdf true

if falls_pdf then
  lade_alles_für_pdfTeX
else
  lade_alles_für_XeTeX
fi
\end{verbatim}

Bedienen Sie sich dazu der Möglichkeiten des |ifthen|-Paketes. (In dieser Übung \emph{keine} \TeX-Interna und Kernel-low-level-Befehle!) Legen Sie damit zunächst eine wahr/falsch-Variable (Boolsche Variable) an und setzen Sie sie auf einen Wert (|true| oder |false|).

Nun folgt die Abfrage: Prüfen Sie, ob die Variable |true| ist und schreiben Sie in diesem Fall alles, was für ein Dokument mit Mikrotypographie nötig ist. Für den |false|-Fall soll eine beliebige Systemschrift als Standardschrift geladen werden, bei der noch zwei Optionen gesetzt werden sollen. Das können z.\,B. spezielle Ligaturen, Farben, Skalierungen o.\,ä. sein.

Achten Sie in beiden Fällen (\XeTeX, pdf\TeX) darauf, die |utf8|-Kodierung zu verwenden und bedenken Sie, welche Pakete jeweils dafür nötig sind!
\end{exercise}
\end{document}