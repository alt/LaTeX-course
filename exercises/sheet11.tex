\documentclass[
%	solution,
%	draft,
	blatt=11,
	ausgabe=25.\,06.\,2010,
	rückgabe=02.\,07.\,2010
]{lcourse-hd}

\begin{document}

\begin{exercise}[
  name=Typographie auf Anfrage,
  punkte=7,
  abgabe = Quelltext per Mail und ausgedruckt. Fertige Dokumente als pdf per Mail.
]{iftypo}
Manchmal kann es nötig sein, ein Dokument mit verschiedenen \TeX-Maschinen zu erstellen, z.\,B. in diesem fiktiven Fall:

• Der Verlag besteht auf der eigenen Schrift mit festgelegten Schriftfeatures,
• die Fakultät hingegen setzt Mikrotypographie voraus, damit der Korrektor die Arbeit schnell lesen kann und mehr Zeit für sonstige Verwaltungsaufgaben hat.

Machen Sie es beiden recht, indem Sie ein Dokument erstellen, das zunächst aus folgendem Grundgerüst besteht:

• Dokumentenklasse: scrreprt
• eine Titelseite mit Autor, Titel und Datum
• zwei Abschnitte mit ein paar Zeilen Text

Um die Anforderungen zu erfüllen, müssen Sie eine Fallunterscheidung (|if|-Abfrage) machen, die es ermöglicht, das Dokument entweder mit pdf\LaTeX\ oder mit \XeLaTeX\ zu kompilieren, ohne dass Fehler auftreten. Der folgende Pseudokode soll das Vorgehen verdeutlichen:
\begin{verbatim}
neueVariable falls_pdf
setzeVariable(falls_pdf)(true)
if falls_pdf
  then(lade_alles_für_pdfTeX)
  else(lade_alles_für_XeTeX)
\end{verbatim}
Es soll dabei \emph{nicht} automatisch geprüft werden, welche \TeX-Maschine läuft, sondern der Schalter soll immer von Hand gesetzt werden! Bedienen Sie sich dazu der Möglichkeiten des |ifthen|-Paketes.\footnote{In dieser Übung \emph{keine} \TeX-Interna und Kernel-low-level-Befehle!} Alles, was Sie brauchen, steht auf der ersten Seite der Paketdokumentation. Legen Sie damit zunächst eine wahr/falsch-Variable (Boolsche Variable) an und setzen Sie sie auf einen Wert (|true| oder |false|).

Nun folgt die Abfrage: Prüfen Sie, ob die Variable |true| ist und schreiben Sie in diesem Fall alles, was für ein Dokument mit Mikrotypographie nötig ist. (Vor allem Paket |microtype|!)

Für den |false|-Fall soll eine beliebige Systemschrift als Standardschrift geladen werden, bei der noch zwei Optionen gesetzt werden sollen. Das können z.\,B. spezielle Ligaturen, Farben, Skalierungen o.\,ä. sein.

Achten Sie in beiden Fällen (\XeTeX, pdf\TeX) darauf, die |utf8|-Kodierung zu verwenden und bedenken Sie, welche Pakete jeweils dafür nötig sind – der ganze Code kommt in die Präambel!

\footnotetext[0]{Falls Sie Verständnisprobleme mit den Fachausdrücken aus der Informatik haben, kontaktieren Sie den Tutor oder Dozenten; die Aufgabe soll nicht an Begrifflichkeiten scheitern!}
\end{exercise}

\begin{expertexercise}[
  name=Wunderwaffe,
  abgabe = Quelltext per Mail{,} das fertige Dokument als Ausdruck.
]{luatex}
Dank der Entwicklungen des |luaotfload|-Paketes ist es möglich, die obigen Anforderungen gleichzeitig zu erfüllen. Lesen Sie dazu in der Dokumentation dieses Paketes, wie man unter lua\LaTeX\ eine Schrift laden muss, um Mikrotypographie zu unterstützen. Testen Sie dies und erweitern Sie obiges Dokument um die Möglichkeit, mit lua\LaTeX\ zu kompilieren oder schreiben Sie handschriftlich, wie man die Schrift laden muss und ob und wie es bei Ihnen funktioniert hat.

\end{expertexercise}

\begin{exercise}[
  name=Maschinenfragen,
  punkte=3,
  abgabe = Handschriftliche Zusammenfassung der Paketdokumentationen.
]{engineif}
Sehen Sie sich die Dokumentation der Pakete |ifpdf|, |ifxetex| und |ifluatex| an und fassen Sie kurz handschriftlich zusammen, wofür die Pakete gut sind und wie sie bedient werden.
\end{exercise}

\end{document}