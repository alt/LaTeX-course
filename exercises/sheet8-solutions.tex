\startsolution{beamer}
\subsolution{Grundgerüst}
Gefordert ist ein einfaches \verb+beamer+-Dokument:
\begin{lcode}
\documentclass[ngerman]{beamer}
\usepackage{babel,xltxtra}
\begin{document}
\begin{frame}{Titel der ersten Folie}{Untertitel}
Inhalt der ersten Folie
\end{frame}
\begin{frame}{Zweite Folie}
Inhalt der zweiten Folie
\end{frame}
\begin{frame}{Dritte Folie}
Inhalt der dritten Folie
\end{frame}
\end{document}
\end{lcode}

\subsolution{Inhalt}
Wenn man \texttt{verbatim}-Code mit \texttt{beamer} darstellen möchte, muss man die entsprechende Folie als \emph{fragile} kennzeichnen:
\begin{lcode}
\begin{frame}[fragile]{Titel}
\end{lcode}
Die dritte Folie könnte dann wie folgt aussehen:
\begin{lcode}
% Eine "fragile" Folie für verbatim-Code
\begin{frame}[fragile]{Verwendung}
	Nachfolgend Beispiele für Makrodefinitionen
mittels \verb+\newcommand+ und \verb+\renewcommand+:

	\begin{verbatim}
		\newcommand\wasser{\ensuremath{\text H_2\text O}}
		\newcommand\zeit[2][00]{#2\,\textsuperscript{#1}}
		\renewcommand\emph[1]{\textbf{#1}}
	\end{verbatim}
	Die zweite Definition nimmt ein mandatorisches
und ein optionales Argument entgegen.

	Der Befehl \verb+\renewcommand+ dient zum
Überschreiben bereits definierter Makros.
\end{frame}
\end{lcode}
\end{minipage}
\begin{minipage}\textwidth
\subsolution{Dynamik}
Die ersten beiden Folien erhalten nun Aufzälungen bzw. Nummerierungen mit verschiedenen Überblendeffekten:
\begin{lcode}
% Folie 1
\begin{frame}{Einleitung}{\TeX{} und \LaTeX}
	% Liste mit Überblendeffekten
	\begin{itemize}
		\item<+-> \TeX{} von D. Knuth entwickelt
		\item<+> Bedienung umständlich
		\item<3> Großes Makropaket: \LaTeX
	\end{itemize}
\end{frame}

% Folie 2
\begin{frame}{Was sind Makros}
	% Nun eine Aufzählung mit Überblendeffekten
	\begin{enumerate}
		\only<1>{\item Abkürzungen um Schreibarbeit zu sparen}
		\uncover<2->{\item Bereitgestellt durch zahlreiche Pakete}
		\uncover<3>{\item Können manuell definiert werden}
	\end{enumerate}
\end{frame}
\end{lcode}

\subsolution{Struktur}
Nun fügen wir vor jeder Folie eine Gliederungsüberschrift ein und zusätzlich am Beginn der Präsentation eine Folie mit dem Inhaltsverzeichnis:
\begin{lcode}
\begin{frame}{Inhalt}
\tableofcontents
\end{frame}

% Erster Abschnitt
\section{Einleitung}
\begin{frame}{...}
...
\end{lcode}
Für die Titelfolie muss die Präambel erweitert werden. Sie wird vor allen anderen Folien eingefügt:
\begin{lcode}
\author{Jakob Herpich}
\title{\LaTeX-Makros}
\begin{document}
\begin{frame}
\maketitle
\end{frame}
...
\end{lcode}
\end{minipage}
\begin{minipage}{\textwidth}
\subsolution{Aussehen}
Zum Schluss kann noch das Aussehen der Präsentation angepasst werden. Dazu stehen viele vordefinierte Themes zur Auswahl (s. Paketdokumentation). Dies sollte in der Präambel geschehen. Zum Beispiel:
\begin{lcode}
\usetheme{CambridgeUS}
\end{lcode}
Mit beliebig hohem Aufwand kann man fast alles anpassen:
\begin{lcode}
\useinnertheme{rounded} % Für runde Listenpunkte etc.
\useoutertheme{smoothtree} % Zeigt aktuelle section an
\setbeamertemplate{navigation symbols}{} % Entfernt Symbole für Navigation
\end{lcode}
\stopsolution
\startsolution{showexpl}
Hierzu muss man nur das Paket \texttt{showexpl} laden und statt der \texttt{verbatim}-Umgebung die \texttt{LTXexample}-Umgebung verwenden:
\begin{lcode}
\begin{frame}[fragile]{Beispiele}
	\begin{LTXexample}
		\wasser \\
		\zeit[59]{23} \\
		\zeit{23} \\
		\emph{hervorgehoben}
	\end{LTXexample}
\end{frame}
\end{lcode}
\stopsolution
