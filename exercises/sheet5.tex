\documentclass[
%	solution,
	draft,
	blatt=5,
	ausgabe=14.\,05.\,2010,
	rückgabe=21.\,05.\,2010
]{lcourse-hd}

\usepackage{
  caption,
  colortbl
}


\begin{document}
\begin{exercise}[
  name={Tabellen – in Farbe!},
  punkte=5,
  abgabe = Quellcode per Mail{,} Quellcode und fertiges Dokument (schwarz-weiß) ausgedruckt.]{colortable}
Bisher waren alle Übungsaufgaben in klassischem schwarz/weiß gehalten. An dieser Stelle soll aber gezeigt werden, dass \LaTeX\ sehr wohl auch mit Farben umgehen kann\footnote{Sogar Videos können in \LaTeX\ eingebunden werden!} – und zwar am Beispiel einer Tabelle.

Farben können in \LaTeX\ mit dem Paket |color| verwendet werden, das den Befehle |\color{}| zur Verfügung stellt. |\color{}| nimmt als Argument eine bekannte Farbe und stellt auf diese um, z.\,B. |\color{blue}|. Angaben in rgb-Code o.\,ä. sind auch möglich, z.\,B. |\color[gray]{0.8}|. Das Paket |xcolor| erweitert die Möglichkeiten des |\color|-Befehls noch enorm. (Unter anderem ist die Angabe der entsprechenden Wellenlänge (!) möglich.)

Studieren Sie die Dokumentation des Paketes |colortbl| und produzieren Sie mithilfe dieses Paketes ein Minimalbeispiel, das eine Tabelle enthalten soll, die aussieht wie die folgende:
\par 
\centering
\begin{tabular}{l!{\hspace*{-4.3mm}}l}
\colorbox{gray}{\textcolor{black}{Wochentag}} & \colorbox{gray}{\textcolor{black}{Buchstaben\vphantom{Wg}}}\\
\colorbox{blue}{\textcolor{white}{Montag\hspace*{.8em}}} & 6 \\
\colorbox{blue}{\textcolor{white}{Dienstag\hspace*{.34em}}} & 8\\
\colorbox{blue}{\textcolor{white}{Mittwoch}} &  8\\
\vdots & \vdots \\
\end{tabular}
\captionof{table}{Eine wichtige Tabelle hat immer\\ eine vielsagende Beschriftung!}
\subexercise{\raisebox{1ex}{\scalebox{.5}{\manual\char127}} \TeX\ als Rechner}
Nur für die \TeX perten: Überlegen Sie sich eine Methode, die Buchstabenanzahl automatisch berechnen und in der gewünschten Formatierung ausgeben zu lassen!
\end{exercise}

\begin{expertexercise}[
  name={{\LARGE $\pi$} Day – two months after},
  abgabe = Hand- oder maschinenschriftliche Zusammenfassung der Ergebnisse.]{piday}
Zwei Monate nach dem internationalen $\pi$-Tag dürfen Sie in diesem Dokument Ihre Fähigkeiten als \TeX-Compiler testen.
\subexercise{{\Large$\pi$} Day?}
Erklären Sie, warum gerade der 14.\,03. internationaler $\pi$-Tag ist.
\subexercise{Do The Lion!}
Versuchen Sie (ohne zu komplieren!) die Ausgabe des Dokumentes vorherzusagen~– d.\,h. kompilieren Sie im Kopf! Versuchen Sie dabei schrittweise vorzugehen und einzelne Kommandos zu identifizieren, suchen Sie nach bekannten Namen primitiver Anweisungen etc. und schreiben Sie alle identifizierten Schritte auf.
\begin{lcode}
              \let~\catcode
          ~`z0~`'1~`,2~`q13~`z#
       14~`46zdefq41'~`4113zgdef,q
     QQ41'~`4113zlet,qBB415425'41-P#
    7427,QPPzexpandafterqAA414243'H#H
   42434341542415,qCC41742743'41i-D42#
  434343,qww',zedefw'PPPCPBAyap!,qEE',q
 6641'~`4113zcountdef,QNNzifnum6RR1R20E6
 YY2QAAzadvanceY-RAY1QMMzmultiply6XX3EqS
 S'Ezhskip0.5em,EQJJzjobnameEQmmwE6TT5qj
 j4142x'41zgdefm'42,,EQGGzglobalE6CC7Eqr
 r41'T41MT41ACT,qHH'C0rXrYC-CrR,qOO'zifx
 mEPjwxEzelsePjmxzfi,qcc'NX<RHNC<0Szelse
  OEzfiAX1PczelseEGAY2zfi,EzedefzJ'J,qv
   v'@,zifxzJvQOO*zfiEqll'NY<RX-Rzhbox
    'c,Plzfi,zttlzbye3.14159265358979
     3238462643383279502884197169399
       375105820974944592307816406
          286208998628034825342
              1170679821...
\end{lcode}
\subexercise{\TeX\ it!}
Testen Sie nun Ihre Vorhersage. Sollte etwas anders laufen als in Ihrem Kopf, notieren Sie dies.
\subexercise{Ostern?}
In dem Code ist ein sog. \emph{easteregg} eingebaut – ein spezielles Feature in Programmen, das bei einer bestimmten Eingabe oder an einem bestimmten Datum ausgeführt wird. In diesem Fall ist es an den |\jobname| gekoppelt. Schreiben Sie noch vor |\let| die Anweisung |\def\jobname{@}|. Damit wird \TeX\ gesagt, dass es die Datei so behandeln soll, als heiße sie |@.tex|. Die Ausgabedatei heißt dann |@.pdf|. Die Ausgabe sollte Sie natürlich nicht überraschen nach der zweiten Teilaufgabe.
\end{expertexercise}

\end{document}