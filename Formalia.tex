\documentclass[ngerman]{scrartcl}

\usepackage{
  babel,
  hyperref,
  xltxtra
}

\setmainfont{Arno Pro}

\title{Übersicht der formalen Anforderungen zur Scheinerlangung}
\subtitle{\LaTeX-Kurs im Sommersemester 2010}
\author{Arno Trautmann\thanks{Arno.Trautmann@gmx.de}}

\hypersetup{
  colorlinks
}

\begin{document}
\maketitle

\minisec{Für eine erfolgreiche Teilnahme an der Vorlesung mit Scheinerwerb sind folgende Formalien einzuhalten:\vspace*{1em}}
\catcode`\•13
\let•\item
\begin{itemize}
• Anmeldung im Moodle
• regelmäßige Teilnahme am Kurs
• Bearbeitung und Einreichung der Übungsaufgaben
• Erreichen von mindestens 50\% der möglichen Gesamtpunktzahl
• Scheinnote richtet sich nach erreichter Übungspunktzahl (50\% = Note 4,0; 100\% = Note 1,0; dazwischen linear abgestuft.)
\end{itemize}

\minisec{Für die Übungsabgabe sind folgende Punkte einzuhalten:\vspace*{1em}}
\begin{itemize}
• Übungsaufgaben sind bis Freitag vor der folgenden Vorlesung abzugeben, d.\,h. die Mail muss gesendet worden sein und die geforderten Ausdrucke. Sollten Ausdrucke vergessen\,/\, nicht abgegeben werden, dient die Mail als Sicherheit, dass die Aufgaben bearbeitet wurden. Ausdrucke müssen zur Punktvergabe aber auf jeden Fall nachgereicht werden!
• Abgaben in Gruppen bis zu drei Studenten\footnote{Von lat. studens = sich bemühend, damit sind also sowohl weibliche Bemühende als auch männliche Bemühende gemeint.} sind erwünscht
• Abgabe je nach Angabe im Aufgabenzettel: per Mail, handschriftlich, gedruckt (das fertige Dokument, also pdf, oder der Quellcode selbst)
• bei Einsendung per Mail sind nur die angeforderten Dateien zu senden – alles andere überfüllt unnötig die Mailordner
• Maileinsendungen \emph{immer} an: \url{Herpich@stud.uni-heidelberg.de} mit Betreff: \verb|LaTeX-Übungs: Musterfrau, Mustermann| (bei Namensvettern im Moodle: Vornamen mit angeben!)
• Dateinamen nach dem Muster: \verb|uebung1.1_musterfrau_mustermann.tex|. Sind mehrere \verb|tex|-Dateien pro Übung verlangt, darf der Name beliebig erweitert werden.
• handschriftlich geforderte Abgaben \emph{müssen} handschriftlich bearbeitet werden, sonst droht Punktabzug
\end{itemize}

\minisec{Materialien:\vspace*{1em}}
\begin{itemize}
• Folien zur Vorlesung sind als pdf-Dateien im Moodle
• Übungsblätter sind (zumindest vorläufig) im Moodle als pdf
• alle Dateien sind als Quellcode im github zu finden:
•[] \url{http://github.com/alt/LaTeX-course}
• im Moodle stehen von Zeit zu Zeit weitere Materialien, die rein informell gedacht sind, manchmal aber auch Nachtragungen zur Vorlesung. Es lohnt sich immer, nochmal reinzuschauen.
\end{itemize}

\end{document}